\documentclass[openany, a5paper]{utbook}
\usepackage{lltjp-geometry}
\usepackage{geometry}
\usepackage{pxatbegshi}
% \usepackage{showframe}
\usepackage[fontset=windows, heading=true]{ctex}

%% books, chapters
\usepackage{titleps}
\usepackage{zhnumber}
% \newcommand*{\parttitle}{}
\renewcommand{\part}[1]{%
    \refstepcounter{part}%
    \noindent \mbox{\bfseries \kanjiskip=6pt #1}\par%
    \partmark{#1}\par
    \renewcommand{\parttitle}{#1}%
}
\newcommand{\bbook}[1]{%
    \refstepcounter{chapter}%
    \setcounter{section}{0}
    \noindent   \mbox{\bfseries \kanjiskip=6pt #1}\par%
    \chaptermark{#1}%
}
\newcommand{\bchapter}{%
    \refstepcounter{section}%
    \sectionmark{第\zhnum{section}章}%
    % \renewcommand\sectiontitle{第\zhnum{section}章}
    \noindent {\bfseries\sectiontitle} %
}

% page layout
\usepackage{calc}
\newcommand{\ebox}[1]{
\parbox{\maxof{\widthof{#1}}{4zw}}{\parfillskip=0pt #1}}

\addtolength{\voffset}{1.5cm}
\addtolength{\textwidth}{0.16cm}
\addtolength{\textheight}{0.2cm}
\makeatletter
\newcommand\frontchaptertitle{}
\newcommand\mainpageheader{\mbox{\kanjiskip=5pt\parttitle}\quad\quad\quad\ebox{\chaptertitle}\quad\quad\quad\quad\ebox{\sectiontitle}}
\def\mainchapterheader{
\if@mainmatter
  \raisebox{\dimexpr-\height-6pt\relax}[0pt]{%
    \hbox to \textheight{%
      \tate%\hspace{3zh}\small
      \ifnum\value{chapter} > \z@
        \mainpageheader
      \else
        \frontchaptertitle
      \fi\hfill
      \thepage\hspace{4zh}}}%
  \else
  \raisebox{\dimexpr-\height-\headsep\relax}[0pt]{%
    \hbox to \textheight{%
      \tate\hspace{3zh}\small\chaptertitle\hfill
      \thepage\hspace{4zh}}}%
  \fi
}
\newpagestyle{main}{%
    \widenhead{20pt}{20pt}%
    % TODO -- Set the headings of even pages
    \sethead[][][{\mainchapterheader}]{\mainchapterheader}{}{}
}
\makeatother
\pagestyle{main}

% underlines for proper names, place names and titles
\usepackage[normalem]{ulem}
\newlength{\lPNskip}
\newlength{\rPNskip}
\newcommand\PNline{\leavevmode \bgroup \markoverwith{\raise0.75em\vbox{\hrule width0.2em}}\ULon}
\newcommand\ProperNameC[3]{%
    \setlength{\lPNskip}{0.1zw}%
    \setlength{\rPNskip}{0.1zw}%
    \advance\lPNskip#1zw%
    \advance\rPNskip#2zw%
    \hskip\lPNskip\PNline{\kern-\lPNskip#3\kern-\rPNskip}\hskip\rPNskip%
}
\newcommand\ProperName[1]{%
    \ProperNameC{0}{0}{#1}%
}

\newlength{\PNlength}
\newcommand\PlaceNameC[4][0.7zw]{%
    \settowidth{\PNlength}{#4}%
    \setlength{\lPNskip}{0.1zw}%
    \setlength{\rPNskip}{0.1zw}%
    \advance\lPNskip#2zw%
    \advance\rPNskip#3zw%
    \advance\PNlength-\lPNskip
    \advance\PNlength-\rPNskip
    \advance\lPNskip\lPNskip
    \advance\lPNskip\PNlength
    \makebox[0pt]{%
    \hspace{\lPNskip}\setlength{\fboxsep}{0.05zw}%
        \raisebox{#1}{\framebox{\hspace{\PNlength}}}%
    }\nolinebreak%
    #4%
}
\newcommand\PlaceName[2][0.7zw]{%
    \PlaceNameC[#1]{0}{0}{#2}%
}

\newcommand\smallwave{\leavevmode \bgroup \markoverwith{\raise0.45em\hbox{$\tilde{\null\kern0.25zw}$}}\ULon}
\newcommand\BookTitleC[3]{
    \setlength{\lPNskip}{0.1zw}%
    \setlength{\rPNskip}{0.1zw}%
    \advance\lPNskip#1zw%
    \advance\rPNskip#2zw%
    \hskip0.05zw\hskip\lPNskip\smallwave{\kern-\lPNskip#3\kern-\rPNskip}\kern\rPNskip%
}
\newcommand\BookTitle[1]{%
    \BookTitleC{0}{0}{#1}%
}

\usepackage{pxrubrica}
\makeatletter
\pxrr@k@declare@mark{cdots}{%
  \pxrr@dima=\pxrr@ruby@zw\relax
  \hb@xt@\pxrr@dima{%
    \kern-0.2\pxrr@dima
    \pxrr@if@in@tate{}{\lower 0.37\pxrr@dima}%
    \hb@xt@2\pxrr@dima{%
      \pxrr@dima=\f@size\p@
      \fontsize{2\pxrr@dima}{\z@}\selectfont
      \hss
      $\cdots$
      \hss
    }%
    \hss
  }%
}
\kentensizeratio{0.3}\kentenmarkintate{cdots}


%% verse number: Chinese numerals
\usepackage{intcalc}
\newcommand{\zhvnumber}[1]{%
    \ifnum#1<21%
        \zhnumber{#1}
    \else\ifnum\intcalcMod{#1}{10}=0
        \zhnumber{#1}
        \else
        \zhdigits{#1}
        \fi
    \fi
}

%% verse number formating
\newlength{\vnskip}
\usepackage[noresetlinenumannotation, series={A}, noend,noeledsec,nofamiliar,noledgroup]{reledmac}%Not resetting line number annotation
\setlinenumannotationsep{\\[-3pt]}
\makeatletter
\renewcommand{\store@annot@to@absline}[1]{%
    \ifledRcol%
    \ifcsdef{annotR@\the\absline@numR @\the\section@numR}{%
        \csgappto{annotR@\the\absline@numR @\the\section@numR}{\@linenumannotationsep#1}%
        }{%
            \csgdef{annotR@\the\absline@numR @\the\section@numR}{\zhvnumber{#1}}%
        }%
    \else%
        \ifcsdef{annot@\the\absline@num @\the\section@num}{%
        \csgappto{annot@\the\absline@num @\the\section@num}{\@linenumannotationsep\zhvnumber{#1}}%
    }{%
        \csgdef{annot@\the\absline@num @\the\section@num}{\zhvnumber{#1}}%
    }%
    \fi%
}%
\makeatother
\newcommand{\vnwrapper}[1]{\raisebox{0pt}[\height][0pt]{\parbox{3zw}{\raggedleft\bfseries\tiny #1}}}
\newcommand{\bvmark}[1]{%
    \allowbreak\settowidth{\vnskip}{\mbox{\bfseries\tiny\zhvnumber{#1}}}%
    \raisebox{0.75zw}{\makebox[0pt]{\hskip\vnskip\bfseries\tiny\zhvnumber{#1}}}%
}
\newcommand{\bvannot}[1]{%
    \linenumannotation{#1}\nolinebreak%
}
\newcommand{\bv}[1]{%
    \bvmark{#1}%
    \bvannot{#1}%
}
\makeatletter
\Xwraplinenumannotation{\vnwrapper}%We just typeset the annotation, without any formatting
\renewcommand{\linenumrep}[1]{}% We don't typeset the real line number.
\makeatother
\Xnoidenticallinenumannotation% Don't typeset to identical line number annotation
\addtolength{\linenumsep}{-6pt}

%% sidenote/tochu
\renewcommand{\ledlsnotefontsetup}{\linespread{0.8}\raggedright\footnotesize}
\setlength\ledlsnotewidth{44pt}
\addtolength\ledlsnotesep{10pt}
\leftnoteupfalse

%% punctuation
\newcommand{\yuentien}{\nolinebreak\raisebox{0.45zw}{\makebox[0pt]{\kern-0.5zw .}}\allowbreak}
\newcommand{\chuan}{\nolinebreak\raisebox{0.45zw}{\makebox[0pt]{\kern-0.5zw 。}}\allowbreak}
\newcommand{\chientien}{\nolinebreak\raisebox{0.45zw}{\makebox[0pt]{\kern-0.5zw 、}}\allowbreak}
\usepackage{wasysym}
\newcommand{\Chuan}{\makebox[1zw]{\raisebox{0.125zw}{\large\Circle}}}

\usepackage{adjustbox}
\usepackage{stringstrings}
\makeatletter
\newcounter{@@chulength}
\newcommand{\chu}[1]{%
    \@getstringlength{#1}{@@chulength}%
    \ifodd\value{@@chulength}%
    \adjustbox{set height=1zw, set depth=0pt}{%
        \kern0.25zw\raisebox{0.1zw}{\parbox{\intcalcAdd{\intcalcDiv{\value{@@chulength}}{2}}{1}zw}{%
            \scriptsize \baselineskip=6.2pt \lineskiplimit=-\maxdimen \kanjiskip=0.6zw #1}}\kern-0.25zw%
    }%
    \else%
    \adjustbox{set height=1zw, set depth=0pt}{%
        \kern0.25zw\raisebox{0.1zw}{\parbox{\intcalcDiv{\value{@@chulength}}{2}zw}{%
            \scriptsize \baselineskip=6.2pt \lineskiplimit=-\maxdimen \kanjiskip=0.6zw #1}}\kern-0.25zw
    }%
    \fi%
}
\makeatother


\usepackage{pxeveryshi}
\def\pgfsysdriver{pgfsys-dvipdfmx.def}
\pdfpagewidth=\paperwidth
\pdfpageheight=\paperheight
\usepackage{tikz}
\usetikzlibrary{calc}
\usepackage{eso-pic}
\AddToShipoutPictureBG{%
\begin{tikzpicture}[overlay,remember picture]
\newcommand\cuvframeoffsetevenx{1.2cm}
\newcommand\cuvframetop{3.2cm}
\newcommand\cuvframebottom{1.9cm}
\newcommand\cuvframeleftodd{2.1cm}
\newcommand\cuvframerightodd{1.45cm}
\newcommand\cuvframewidth{0.1cm}
\newcommand\cuvframemargin{0.5cm}
\newcommand\cuvframevnsize{0.7cm}

\ifodd\value{page}
\draw[line width=2pt]
    ($ (current page.north west) + (\cuvframeleftodd,-\cuvframetop) $) % top left
    -- 
    ($ (current page.north east) + (-\cuvframerightodd,-\cuvframetop) $) % top right 
    --
    ($ (current page.south east) + (-\cuvframerightodd,\cuvframebottom) $) % bottom right 
    --
    ($ (current page.south west) + (\cuvframeleftodd,\cuvframebottom) $); % bottom left 
\draw[line width=0.5pt]
    ($ (current page.north west) + (\cuvframeleftodd,-\cuvframetop-\cuvframewidth) $)
    -- 
    ($ (current page.north east) + (-\cuvframerightodd-\cuvframewidth,-\cuvframetop-\cuvframewidth) $)
    --
    ($ (current page.south east) + (-\cuvframerightodd-\cuvframewidth,\cuvframebottom+\cuvframewidth) $)
    --
    ($ (current page.south west) + (\cuvframeleftodd,\cuvframebottom+\cuvframewidth) $);
\draw[line width=0.5pt]
    ($ (current page.north west) + (\cuvframeleftodd + \cuvframemargin,-\cuvframetop-\cuvframewidth) $)
    -- 
    ($ (current page.south west) + (\cuvframeleftodd + \cuvframemargin,\cuvframebottom+\cuvframewidth) $);
\draw[line width=0.5pt]
    ($ (current page.north west) + (\cuvframeleftodd + \cuvframemargin,-\cuvframetop-\cuvframevnsize) $)
    -- 
    ($ (current page.north east) + (-\cuvframerightodd-\cuvframewidth,-\cuvframetop-\cuvframevnsize) $);
\else
\draw[line width=2pt]
    ($ (current page.north east) + (-\cuvframeleftodd,-\cuvframetop) + (\cuvframeoffsetevenx, 0) $)
    -- 
    ($ (current page.north west) + (\cuvframerightodd,-\cuvframetop) + (\cuvframeoffsetevenx, 0) $)
    --
    ($ (current page.south west) + (\cuvframerightodd,\cuvframebottom) + (\cuvframeoffsetevenx, 0) $)
    --
    ($ (current page.south east) + (-\cuvframeleftodd,\cuvframebottom) + (\cuvframeoffsetevenx, 0) $);
\draw[line width=0.5pt]
    ($ (current page.north east) + (-\cuvframeleftodd,-\cuvframetop-\cuvframewidth) + (\cuvframeoffsetevenx, 0) $)
    -- 
    ($ (current page.north west) + (\cuvframerightodd+\cuvframewidth,-\cuvframetop-\cuvframewidth) + (\cuvframeoffsetevenx, 0) $)
    --
    ($ (current page.south west) + (\cuvframerightodd+\cuvframewidth,\cuvframebottom+\cuvframewidth) + (\cuvframeoffsetevenx, 0) $)
    --
    ($ (current page.south east) + (-\cuvframeleftodd,\cuvframebottom+\cuvframewidth) + (\cuvframeoffsetevenx, 0) $);
\draw[line width=0.5pt]
    ($ (current page.north east) + (-\cuvframeleftodd - \cuvframemargin,-\cuvframetop-\cuvframewidth) + (\cuvframeoffsetevenx, 0) $)
    -- 
    ($ (current page.south east) + (-\cuvframeleftodd - \cuvframemargin,\cuvframebottom+\cuvframewidth) + (\cuvframeoffsetevenx, 0) $);
\draw[line width=0.5pt]
    ($ (current page.north east) + (-\cuvframeleftodd - \cuvframemargin,-\cuvframetop-\cuvframevnsize) + (\cuvframeoffsetevenx, 0) $)
    -- 
    ($ (current page.north west) + (\cuvframerightodd + \cuvframewidth,-\cuvframetop-\cuvframevnsize) + (\cuvframeoffsetevenx, 0) $);
\fi
\end{tikzpicture}
}


\renewcommand\thepage{\zhnum{page}\relax}


\begin{document}
\pagestyle{main}
\setlength\parindent{1zw}
\kanjiskip=0.0pt plus 0.2pt

\firstlinenum{1}
\linenumincrement{1}
\part{舊約全書}
\bbook{創世記}
\beginnumbering
\pstart
\bchapter%
\bv{1}起初 神創造天地\chuan\ledleftnote{ 神創造天地}
\bv{2}地是空虛混沌\yuentien 淵面黑暗\yuentien  神的靈運行在水面上\chuan% 
\bv{3} 神說\chientien 要有光\chientien 就有了光\chuan%
\bv{4} 神看光是好的\chientien 就把光暗分開了\chuan%
\bv{5} 神稱光為晝\chientien 稱暗為夜\yuentien 有晚上\chientien 有早晨\chientien 這是頭一日\chuan\Chuan%
\bvmark{6} 神\bvannot{6}說\chientien 諸水之間要有空氣\chientien 將水分為上下\chuan 
\bv{7} 神就造出空氣\chientien 將空氣以下的水\chientien 空氣以上的水分開了\yuentien 事就這樣成了\chuan 
\bv{8} 神稱空氣為天\yuentien 有晚上\chientien 有早晨\chientien 是第二日\chuan\Chuan
\bv{9} 神說\chientien 天下的水要聚在一處\chientien 使旱地露出來\yuentien 事就這樣成了\chuan 
\bv{10} 神稱旱地為地\chientien 稱水的聚處為海\yuentien  神看著是好的\chuan 
\bv{11} 神說\chientien 地要發生青草\chientien 和結種子的菜蔬\chientien 並結果子的樹木\chientien 各從其類\chientien 果子都包著核\yuentien 事就這樣成了\chuan 
\bv{12}於是地發生了青草\chientien 和結種子的菜蔬\chientien 各從其類\chientien 並結果子的樹木\chientien 各從其類\chientien 果子都包著核\chuan  神看著是好的\yuentien%
\bv{13}有晚上\chientien 有早晨\chientien 是第三日\chuan\Chuan
\bv{14} 神說\chientien 天上要有光體\chientien 可以分晝夜\chientien 作記號\chientien 定節令\chientien 日子\chientien 年歲\yuentien 
\bv{15}並要發光在天空\chientien 普照在地上\yuentien 事就這樣成了\chuan%
\bv{16}於是 神造了兩個大光\chientien 大的管晝\chientien 小的管夜\yuentien 又造眾星\chuan 
\bv{17}就把這些光擺列在天空\chientien 普照在地上\chientien 
\bv{18}管理晝夜\chientien 分別明暗\yuentien  神看著是好的\yuentien 
\bv{19}有晚上\chientien 有早晨\chientien 是第四日\chuan\Chuan
\bv{20} 神說\chientien 水要多多滋生有生命的物\yuentien 要有雀鳥飛在地面以上\chientien 天空之中\chuan 
\bv{21} 神就造出大魚\chientien 和水中所滋生各樣有生命的動物\chientien 各從其類\yuentien 又造出各樣飛鳥\chientien 各從其類\yuentien  神看著是好的\chuan 
\bv{22} 神就賜福給這一切\chientien 說\chientien 滋生繁多\chientien 充滿海中的水\yuentien 雀鳥也要多生在地上\chuan 
\bv{23}有晚上\chientien 有早晨\chientien 是第五日\chuan\Chuan
\bv{24} 神說\chientien 地要生出活物來\chientien 各從其類\yuentien 牲畜\chientien 昆蟲\chientien 野獸\chientien 各從其類\yuentien 事就這樣成了\chuan 
\bv{25}於是 神造出野獸\chientien 各從其類\yuentien 牲畜\chientien 各從其類\yuentien 地上一切昆蟲\chientien 各從其類\yuentien  神看著是好的\chuan 
\bv{26} 神說\chientien 我們要照著我們的形像\chientien 按著我們的樣式造人\ledleftnote{照自己之像造人}\chientien 使他們管理海裡的魚\chientien 空中的鳥\chientien 地上的牲畜\chientien 和全地\chientien 並地上所爬的一切昆蟲\chuan 
\bv{27} 神就照著自己的形像造人\chientien 乃是照著他的形像造男造女\chuan 
\bvmark{28} 神\bvannot{28}就賜福給他們\chientien 又對他們說\chientien 要生養眾多\chientien 遍滿地面\chientien 治理這地\yuentien 也要管理海裡的魚\chientien 空中的鳥\yuentien 和地上各樣行動的活物\chuan 
\bv{29} 神說\chientien 看哪\chientien 我將遍地上一切結種子的菜蔬\chientien 和一切樹上所結有核的果子\chientien 全賜給你們作食物\chuan 
\bv{30}至於地上的走獸\chientien 和空中的飛鳥\chientien 並各樣爬在地上有生命的物\chientien \kenten{我將}青草\kenten{賜給}他們作食物\yuentien 事就這樣成了\chuan 
\bv{31} 神看著一切所造的都甚好\yuentien 有晚上\chientien 有早晨\chientien 是第六日\chuan 

\pend
\endnumbering
\beginnumbering
\pstart
\bchapter%
\bv{1}天地萬物都造齊了\chuan
\bv{2}到第七日\chientien  神造物的工已經完畢\chientien 就在第七日歇了他一切的工\chientien 安息了\chuan 
\bv{3} 神賜福給第七日\chientien 定為聖日\chientien 因為在這日 神歇了他一切創造的工\chientien 就安息了\chuan\Chuan
\bv{4}創造天地的來歷\chientien 在耶和華 神造天地的日子\chientien 乃是這樣\yuentien 
\bv{5}野地還沒有草木\chientien 田間的菜蔬還沒有長起來\chientien 因為耶和華 神還沒有降雨在地上\chientien 也沒有人耕地\chuan 
\bv{6}但有霧氣從地上騰\chientien 滋潤遍地\chuan 
\bv{7}耶和華 神用地上的塵土造人\chientien 將生氣吹在他鼻孔裡\chientien 他就成了有靈的活人\chientien \kenten{名叫}\ProperNameC{0}{0.5}{\kenten{亞}\kenten{當}}\chuan 
\bv{8}\ledleftnote{立伊甸園}耶和華 神在東方的\PlaceName{伊甸}立了一個園子\chientien 把所造的人安置在那裡\chuan 
\bv{9}耶和華 神使各樣的樹從地裡長出來\chientien 可以悅人的眼目\chientien \kenten{其上的果子}好作食物\yuentien 園子當中又有生命樹\chientien 和分別善惡的樹\chuan 
\bv{10}有河從\PlaceName{伊甸}流出來滋潤那園子\chientien 從那裡分為四道\chuan 
\bv{11}第一道名叫\PlaceNameC{0}{0.5}{比遜}\chientien 就是環繞\PlaceName{哈腓拉}全地的\yuentien 在那裡有金子\chientien 
\bv{12}並且那地的金子是好的\yuentien 在那裡又有珍珠和紅瑪瑙\chuan 
\bv{13}第二道河名叫\PlaceNameC{0}{0.5}{基訓}\chientien 就是環繞\PlaceName{古實}全地的\chuan 
\bv{14}第三道河名叫\PlaceNameC{0}{0.5}{希底結}\chientien 流在\PlaceName{亞述}的東邊\chuan 第四道河就是\PlaceName{伯拉}河\chuan 
\bv{15}耶和華 神將那人安置在\PlaceName{伊甸}園\chientien 使他修理看守\chuan 
\bv{16}耶和華 神吩咐他說\chientien 園中各樣樹上的果子\chientien 你可以隨意喫\yuentien 
\bv{17}只是分別善惡樹上的果子\chientien 你不可喫\chientien 因為你喫的日子必定死\chuan\Chuan
\bv{18}\ledleftnote{爲男人造配偶}耶和華 神說\chientien 那人獨居不好\chientien 我要為他造一個配偶幫助他\chuan 
\bv{19}耶和華 神用土所造成的野地各樣走獸\chientien 和空中各樣飛鳥\chientien 都帶到那人面前看他叫甚麼\yuentien 那人怎樣叫各樣的活物\chientien 那就是他的名字\chuan 
\bv{20}那人便給一切牲畜\chientien 和空中飛鳥\chientien 野地走獸都起了名\yuentien 只是那人沒有遇見配偶幫助他\chuan 
\bv{21}耶和華 神使他沉睡\chientien 他就睡了\yuentien 於是取下他的一條肋骨\chientien 又把肉合起來\chuan 
\bv{22}耶和華 神就用那人身上所取的肋骨\chientien 造成一個女人\chientien 領他到那人跟前\chuan 
\bv{23}那人說\chientien 這是我骨中的骨\chientien 肉中的肉\chientien 可以稱他為女人\chientien 因為他是從男人身上取出來的\chuan 
\bv{24}因此\chientien 人要離開父母\chientien 與妻子連合\chientien 二人成為一體\chuan 
\bv{25}當時夫妻二人\chientien 赤身露體\chientien 並不羞恥\chuan 

\pend
\endnumbering
\beginnumbering
\pstart
\bchapter%
\bv{1}\ledleftnote{始祖被誘惑}耶和華 神所造的\chientien 惟有蛇比田野一切的活物更狡猾\chuan 蛇對女人說\chientien  神豈是真說\chientien 不許你們喫園中所有樹上的果子麼\chuan 
\bv{2}女人對蛇說\chientien 園中樹上的果子我們可以喫\yuentien 
\bv{3}惟有園當中那棵樹上的果子\chientien  神曾說\chientien 你們不可喫\chientien 也不可摸\chientien 免得你們死\chuan 
\bv{4}蛇對女人說\chientien 你們不一定死\chientien 
\bv{5}因為 神知道\chientien 你們喫的日子眼睛就明亮了\chientien 你們便如 神能知道善惡\chuan 
\bv{6}\ledleftnote{違背主命}於是女人見那棵樹\kenten{的果子}好作食物\chientien 也悅人的眼目\chientien 且是可喜愛的\chientien 能使人有智慧\chientien 就摘下果子來喫了\yuentien 又給他丈夫\chientien 他丈夫也喫了\chuan 
\bv{7}他們二人的眼睛就明亮了\chientien 纔知道自己是赤身露體\chientien 便拿無花果樹的葉子\chientien 為自己編作裙子\chuan 
\bv{8}天起了涼風\chientien 耶和華 神在園中行走\chuan 那人和他妻子聽見 神的聲音\chientien 就藏在園裡的樹木中\chientien 躲避耶和華 神的面\chuan 
\bv{9}耶和華 神呼喚那人\chientien 對他說\chientien 你在那裡\chuan 
\bv{10}他說\chientien 我在園中聽見你的聲音\chientien 我就害怕\chientien 因為我赤身露體\yuentien 我便藏了\chuan 
\bv{11}耶和華說\chientien 誰告訴你赤身露體呢\chientien 莫非你喫了我吩咐你不可喫的那樹上的果子麼\chuan 
\bv{12}那人說\chientien 你所賜給我\chientien 與我同居的女人\chientien 他把那樹上的果子給我\chientien 我就喫了\chuan 
\bv{13}耶和華 神對女人說\chientien 你作的是甚麼事呢\chuan 女人說\chientien 那蛇引誘我\chientien 我就喫了\chuan 
\bv{14}耶和華 神對蛇說\chientien 你既作了這事\chientien 就必受咒詛\chientien 比一切的牲畜野獸更甚\chientien 你必用肚子行走\chientien 終身喫土\chuan 
\bv{15}我又要叫你和女人彼此為仇\chientien 你的後裔和女人的後裔\chientien 也彼此為仇\yuentien 女人的後裔要傷你的頭\chientien 你要傷他的腳跟\chuan 
\bv{16}又對女人說\chientien 我必多多加增你懷胎的苦楚\chientien 你生產兒女必多受苦楚\yuentien 你必戀慕你丈夫\chientien 你丈夫必管轄你\chuan 
\bv{17}又對\ProperName{亞當}說\chientien 你既聽從妻子的話\chientien 喫了我所吩咐你不可喫的那樹上的果子\chientien 地必為你的緣故受咒詛\yuentien 你必終身勞苦\chientien 纔能從地裡得喫的\chuan 
\bv{18}地必給你長出荊棘和蒺藜來\chientien 你也要喫田間的菜蔬\chuan 
\bv{19}你必汗流滿面纔得糊口\chientien 直到你歸了土\chientien 因為你是從土而出的\yuentien 你本是塵土\chientien 仍要歸於塵土\chuan 
\bv{20}\ProperNameC{1}{0}{亞當}給他妻子起名叫\ProperNameC{0}{0.5}{夏娃}\chientien 因為他是眾生之母\chuan 
\bv{21}耶和華 神為\ProperName{亞當}和他妻子用皮子作衣服\chientien 給他們穿\chuan\Chuan
\bv{22}耶和華 神說\chientien 那人已經與我們相似\chientien 能知道善惡\yuentien 現在恐怕他伸手又摘生命樹的果子喫\chientien 就永遠活著\yuentien 
\bv{23}耶和華 神便打發他出\PlaceName{伊甸}園去\chientien 耕種他所自出之土\chuan 
\bv{24}\ledleftnote{逐出伊甸}於是把他趕出去了\yuentien 又在伊甸園的東邊安設基路伯\chientien 和四面轉動發火焰的劍\chientien 要把守生命樹的道路\chuan 

\pend
\endnumbering
\beginnumbering
\pstart
\bchapter%
\bv{1}\ledleftnote{生該隱亞伯}有一日\chientien 那人和他妻子\ProperName{夏娃}同房\chientien\ProperName{夏娃}就懷孕\chientien 生了\ProperNameC{0}{0.5}{該隱}\chientien\chu{就是得的意思}便說\chientien 耶和華使我得了一個男子\chuan 
\bv{2}又生了\ProperName{該隱}的兄弟\ProperNameC{0}{0.5}{亞伯}\chuan \ProperName{亞伯}是牧羊的\yuentien \ProperName{該隱}是種地的\chuan 
\bv{3}有一日\chientien\ProperName{該隱}拿地裡的出產為供物獻給耶和華\yuentien 
\bv{4}\ProperNameC{0.5}{0}{亞伯}也將他羊群中頭生的\chientien 和羊的脂油獻上\yuentien 耶和華看中了\ProperName{亞伯}和他的供物\yuentien 
\bv{5}只是看不中\ProperName{該隱}和他的供物\yuentien\ProperName{該隱}就大大的發怒\chientien 變了臉色\chuan 
\bv{6}耶和華對\ProperName{該隱}說\chientien 你為甚麼發怒呢\chientien 你為甚麼變了臉色呢\yuentien 
\bv{7}你若行得好\chientien 豈不蒙悅納\chientien 你若行得不好\chientien 罪就伏在門前\yuentien 他必戀慕你\chientien 你卻要制伏他\chuan 
\bv{8}\ledleftnote{該隱殺其弟}\ProperNameC{0.5}{0}{該隱}與他兄弟\ProperName{亞伯}說話\chientien 二人正在田間\chientien\ProperName{該隱}起來打他兄弟\ProperNameC{0}{0.5}{亞伯}\chientien 把他殺了\chuan\Chuan
\bv{9}耶和華對\ProperName{該隱}說\chientien 你兄弟\ProperName{亞伯}在那裡\yuentien 他說\chientien 我不知道\chientien 我豈是看守我兄弟的嗎\chuan 
\bv{10}耶和華說\chientien 你作了甚麼事呢\chientien 你兄弟的血\chientien 有聲音從地裡向我哀告\chuan 
\bv{11}地開了口\chientien 從你手裡接受你兄弟的血\yuentien 現在你必從這地受咒詛\chuan 
\bv{12}你種地\chientien 地不再給你效力\yuentien 你必流離飄蕩在地上\chuan 
\bv{13}\ProperNameC{1}{0}{該隱}對耶和華說\chientien 我的刑罰太重\chientien 過於我所能當的\chuan 
\bv{14}你如今趕逐我離開這地\chientien 以致不見你面\yuentien 我必流離飄蕩在地上\chientien 凡遇見我的必殺我\chuan 
\bv{15}耶和華對他說\chientien 凡殺\ProperName{該隱}的必遭報七倍\chuan 耶和華就給\ProperName{該隱}立一個記號\chientien 免得人遇見他就殺他\chuan\Chuan
\bv{16}於是\ProperName{該隱}離開耶和華的面\chientien 去住在\PlaceName{伊甸}東邊\PlaceName{挪得}之地\chuan 
\bv{17}\ProperNameC{1}{0}{該隱}與妻子同房\chientien 他妻子就懷孕\chientien 生了\ProperNameC{0}{0.5}{以諾}\chientien\ProperName{該隱}建造了一座城\chientien 就按著他兒子的名將那城叫作以諾\chuan 
\bv{18}\ProperNameC{1}{0}{以諾}生\ProperNameC{0}{0.5}{以拿}\chientien\ProperName{以拿}生\ProperNameC{0}{0.5}{米戶雅利}\chientien\ProperName{米戶雅利}生\ProperNameC{0}{0.5}{瑪土撒利}\chientien \ProperName{瑪土撒利}生\ProperNameC{0}{0.5}{拉麥}\chuan 
\bv{19}\ProperNameC{1}{0}{拉麥}娶了兩個妻\chientien 一個名叫\ProperName{亞}\hss\linebreak\ProperNameC{0}{0.4}{大}\chientien 一個名叫\ProperNameC{0}{0.5}{洗拉}\chuan 
\bv{20}\ProperNameC{1}{0}{亞大}生\ProperNameC{0}{0.5}{雅八}\chientien \ProperName{雅八}就是住帳棚牧養牲畜之人的祖師\chuan 
\bv{21}\ProperNameC{1}{0}{雅八}的兄弟名叫\ProperNameC{0}{0.5}{猶八}\yuentien 他是一切彈琴吹簫之人的祖師\chuan 
\bv{22}\ProperNameC{1}{0}{洗拉}又生了\ProperNameC{0}{0.5}{土八該隱}\chientien 他是打造各樣銅鐵利器的\chientien \chu{或作是銅匠鐵匠的祖師}\ProperName{土八該隱}的妹子是\ProperNameC{0}{0.5}{拿瑪}\chuan 
\bv{23}\ProperNameC{1}{0}{拉麥}對他兩個妻子說\chientien\ProperNameC{0}{0.5}{亞大}\chientien\ProperNameC{0}{0.5}{洗拉}\chientien 聽我的聲音\chientien \ProperName{拉麥}的妻子細聽我的話語\chientien 壯年人傷我\chientien 我把他殺了\chientien 少年人損我\chientien 我把他害了\yuentien \chu{或作我殺壯士卻傷自己我害幼童卻損本身}
\bv{24}若殺\ProperNameC{0}{0.5}{該隱}\chientien 遭報七倍\chientien 殺\ProperNameC{0}{0.5}{拉麥}\chientien 必遭報七十七倍\chuan\Chuan
\bv{25}\ProperNameC{1}{0}{亞當}又與妻子同房\chientien 他就生了一個兒子\chientien 起名叫\ProperNameC{0}{0.5}{塞特}\chientien 意思說\chientien  神另給我立了一個兒子代替\ProperName{亞}\hss\linebreak\ProperNameC{0}{0.5}{伯}\chientien 因為\ProperName{該隱}殺了他\chuan 
\bv{26}\ProperNameC{1}{0}{塞特}也生了一個兒子\chientien 起名叫\ProperNameC{0}{0.5}{以挪士}\chuan 那時候人纔求告耶和華的名\chuan 

\pend
\endnumbering
\beginnumbering
\pstart
\bchapter%
\bv{1}\ProperNameC{0.5}{0}{亞當}的後代\chientien\ledleftnote{亞當之後裔}記在下面\chuan 當 神造人的日子\chientien 是照著自己的樣式造的\chientien 
\bv{2}並且造男造女\chientien 在他們被造的日子{\chientien}\hspace{0pt}{ }神賜福給他們\chientien 稱他們為人\chuan 
\bv{3}\ProperNameC{0.5}{0}{亞當}活到一百三十歲\chientien 生了一個兒子\chientien 形像樣式和自己相似\chientien 就給他起名叫\ProperNameC{0}{0.5}{塞特}\chuan 
\bv{4}\ProperNameC{0.5}{0}{亞當}生\ProperName{塞特}之後\chientien 又在世八百年\yuentien 並且生兒養女\chuan 
\bv{5}\ProperNameC{0.5}{0}{亞當}共活了九百三十歲就死了\chuan\Chuan
\bv{6}\ProperNameC{0.5}{0}{塞特}活到一百零五歲\chientien 生了\ProperNameC{0}{0.5}{以挪士}\chuan 
\bv{7}\ProperNameC{0.5}{0}{塞特}生\ProperName{以挪士}之後\chientien 又活了八百零七年\yuentien 並且生兒養女\chuan 
\bv{8}\ProperNameC{0.5}{0}{塞}\ProperName{特}共活了九百一十二歲就死了\chuan\Chuan
\bv{9}\ProperNameC{0.5}{0}{以挪士}活到九十歲\chientien 生了\ProperNameC{0}{0.5}{該南}\chuan 
\bv{10}\ProperNameC{0.5}{0}{以挪士}生\ProperName{該南}之後\chientien 又活了八百一十五年\yuentien 並且生兒養女\chuan 
\bv{11}\ProperNameC{1}{0}{以挪士}共活了九百零五歲就死了\chuan\Chuan
\bv{12}\ProperNameC{1}{0}{該南}活到七十歲\chientien 生了\ProperNameC{0}{0.5}{瑪勒列}\chuan 
\bv{13}\ProperNameC{1}{0}{該南}生\ProperName{瑪\allowbreak 勒列}之後\chientien 又活了八百四十年\yuentien 並且生兒養女\chuan 
\bv{14}\ProperNameC{1}{0}{該南}共活了九百一十歲就死了\chuan\Chuan
\bv{15}\ProperNameC{1}{0}{瑪勒列}活到六十五歲\chientien 生了\ProperNameC{0}{0.5}{雅列}\chuan 
\bv{16}\ProperNameC{1}{0}{瑪勒列}生\ProperName{雅列}之後\chientien 又活了八百三十年\yuentien 並且生兒養女\chuan 
\bv{17}\ProperNameC{1}{0}{瑪勒列}共活了八百九十五歲就死了\chuan\Chuan
\bv{18}\ProperNameC{1}{0}{雅列}活到一百六十二歲\chientien 生了\ProperNameC{0}{0.5}{以諾}\chuan 
\bv{19}\ProperNameC{1}{0}{雅列}生\ProperName{以諾}之後\chientien 又活了八百年\yuentien 並且生兒養女\chuan 
\bv{20}\ProperNameC{1}{0}{雅列}共活了九百六十二歲就死了\chuan\Chuan
\bv{21}\ledleftnote{以諾被主接去}\ProperNameC{1}{0}{以諾}活到六十五歲\chientien 生了\ProperNameC{0}{0.5}{瑪土撒拉}\chuan 
\bv{22}\ProperNameC{1}{0}{以諾}生\ProperName{瑪土撒拉}之後\chientien 與 神同行三百年\yuentien 並且生兒養女\chuan 
\bv{23}\ProperNameC{1}{0}{以諾}共活了三百六十五歲\chuan 
\bv{24}\ProperNameC{1}{0}{以諾}與 神同行\chientien  神將他取去\chientien 他就不在世了\chuan\Chuan
\bv{25}瑪\ProperName{土撒拉}活到一百八十七歲\chientien 生了\ProperNameC{0}{0.5}{拉麥}\chuan 
\bv{26}\ProperNameC{1}{0}{瑪土撒拉}生\ProperName{拉麥}之後\chientien 又活了七百八十二年\yuentien 並且生兒養女\chuan 
\bv{27}瑪\ProperName{土撒拉}共活了九百六十九歲就死了\chuan\Chuan
\bv{28}\ProperNameC{1}{0}{拉麥}活到一百八十二歲\chientien 生了一個兒子\chientien 
\bv{29}給他起名叫\ProperNameC{0}{0.5}{挪亞}\chientien 說\chientien 這個兒子必為我們的操作\chientien 和手中的勞苦\chientien 安慰我們\yuentien 這操作勞苦是因為耶和華咒詛地\chuan 
\bv{30}\ProperNameC{1}{0}{拉麥}生\ProperName{挪亞}之後\chientien 又活了五百九十五年\yuentien 並且生兒養女\chuan 
\bv{31}\ProperNameC{1}{0}{拉麥}共活了七百七十七歲就死了\chuan 
\bv{32}\ProperNameC{1}{0}{挪亞}五百歲生了\ProperNameC{0}{0.5}{閃}\chientien\ProperNameC{0}{0.5}{含}\chientien\ProperNameC{0}{0.5}{雅弗}\chuan 

\pend
\endnumbering
\beginnumbering
\pstart
\bchapter%
\bv{1}當人在世上多起來\chientien 又生女兒的時候\chientien 
\bv{2} 神的兒子們看見人的女子美貌\chientien 就隨意挑選\chientien 娶來為妻\chuan\Chuan
\bv{3}耶和華說\chientien 人既屬乎血氣\chientien 我的靈就不永遠住在他裡面\chientien 然而他的日子還可到一百二十年\chuan 
\bv{4}那時候有偉人在地上\chientien 後來 神的兒子們\chientien 和人的女子們交合生子\chientien 那就是上古英武有名的人\chuan 
\bv{5}\ledleftnote{耶和華後悔造人於地}耶和華見人在地上罪惡很大\chientien 終日所思想的盡都是惡\yuentien 
\bv{6}耶和華就後悔造人在地上\chientien 心中憂傷\chuan 
\bv{7}耶和華說\chientien 我要將所造的人\chientien 和走獸\chientien 並昆蟲\chientien 以及空中的飛鳥\chientien 都從地上除滅\chientien 因為我造他們後悔了\chuan 
\bv{8}惟有\ProperName{挪亞}在耶和華眼前蒙恩\chuan\Chuan
\bv{9}\ProperNameC{0.5}{0}{挪亞}的後代\chientien 記在下面\chuan\ProperName{挪亞}是個義人\chientien 在當時的世代是個完全人\yuentien\ProperName{挪亞}與 神同行\chuan 
\bv{10}\ProperNameC{0.5}{0}{挪亞}生了三個兒子\chientien 就是\ProperNameC{0}{0.5}{含}\chientien\ProperNameC{0}{0.5}{雅弗}\chuan 
\bv{11}世界在 神面前敗壞\yuentien 地上滿了強暴\chuan 
\bv{12} 神觀看世界\chientien 見是敗壞了\yuentien 凡有血氣的人\chientien 在地上都敗壞了行為\chuan 
\bv{13} 神就對\ProperName{挪亞}說\chientien 凡有血氣的人\chientien 他的盡頭已經來到我面前\chientien 因為地上滿了他們的強暴\chientien 我要把他們和地一併毀滅\chuan 
\bv{14}\ledleftnote{ 神命挪亞造方舟}你要用歌斐木造一隻方舟\chientien 分一間一間的造\chientien 裡外抹上松香\chuan 
\bv{15}方舟的造法乃是這樣\chientien 要長三百肘\chientien 寬五十肘\chientien 高三十肘\chuan 
\bv{16}方舟上邊要留透光處\chientien 高一肘\yuentien 方舟的門要開在旁邊\yuentien 方舟要分上\chientien 中\chientien 下三層\chuan 
\bv{17}看哪\chientien 我要使洪水氾濫在地上\chientien 毀滅天下\chientien 凡地上有血肉\chientien 有氣息的活物\chientien 無一不死\chuan 
\bv{18}我卻要與你立約\chientien 你同你的妻\chientien 與兒子\chientien 兒婦\chientien 都要進入方舟\chuan 
\bv{19}凡有血肉的活物\chientien 每樣兩個\chientien 一公一母\chientien 你要帶進方舟\chientien 好在你那裡保全生命\chuan 
\bv{20}飛鳥各從其類\chientien 牲畜各從其類\chientien 地上的昆蟲各從其類\chientien 每樣兩個\chientien 要到你那裡\chientien 好保全生命\chuan 
\bv{21}你要拿各樣食物積蓄起來\chientien 好作你和他們的食物\chuan 
\bv{22}\ProperNameC{1}{0}{挪亞}就這樣行\yuentien 凡 神所吩咐的\chientien 他都照樣行了\chuan 

\pend
\endnumbering
\beginnumbering
\pstart
\bchapter%
\bv{1}\ledleftnote{挪亞進方舟}耶和華對\ProperName{挪亞}說\chientien 你和你的全家都要進入方舟\chientien 因為在這世代中\chientien 我見你在我面前是義人\chuan 
\bv{2}凡潔淨的畜類\chientien 你要帶七公七母\yuentien 不潔淨的畜類\chientien 你要帶一公一母\yuentien 
\bv{3}空中的飛鳥\chientien 也要帶七公七母\chientien 可以留種\chientien 活在全地上\chientien 
\bv{4}因為再過七天\chientien 我要降雨在地上四十晝夜\chientien 把我所造的各種活物\chientien 都從地上除滅\chuan 
\bv{5}\ProperNameC{0.5}{0}{挪亞}就遵著耶和華所吩咐的行了\chuan\Chuan
\bv{6}當洪水氾濫在地上的時候\chientien\ProperName{挪亞}整六百歲\chuan 
\bv{7}\ProperNameC{0.5}{0}{挪亞}就同他的妻\chientien 和兒子\chientien 兒婦\chientien 都進入方舟\chientien 躲避洪水\chuan 
\bv{8}潔淨的畜類\chientien 和不潔淨的畜類\chientien 飛鳥並地上一切的昆蟲\chientien 
\bv{9}都是一對一對的\chientien 有公有母\chientien 到\ProperName{挪亞}那裡進入方舟\chientien 正如 神所吩咐\ProperName{挪亞}的\chuan 
\bv{10}過了那七天\chientien 洪水氾濫在地上\chuan 
\bv{11}當\ProperName{挪亞}六百歲\chientien 二月十七日那一天\chientien 大淵的泉源\chientien 都裂開了\chientien 天上的窗戶\chientien 也敞開了\chuan 
\bv{12}四十晝夜降大雨在地上\chuan\Chuan
\bv{13}正當那日\chientien \ProperName{挪亞}和他三個兒子\ProperNameC{0}{0.5}{閃}\chientien\ProperNameC{0}{0.5}{含}\chientien\ProperNameC{0}{0.5}{雅弗}\chientien 並\ProperName{挪亞}的妻子\chientien 和三個兒婦\chientien 都進入方舟\chuan 
\bv{14}他們和百獸\chientien 各從其類\chientien 一切牲畜\chientien 各從其類\chientien 爬在地上的昆蟲\chientien 各從其類\chientien 一切禽鳥\chientien 各從其類\chientien 都進入方舟\chuan 
\bv{15}凡有血肉\chientien 有氣息的活物\chientien 都一對一對的到\ProperName{挪亞}那裡\chientien 進入方舟\chuan 
\bv{16}凡有血肉進入方舟的\chientien 都是有公有母\chientien 正如 神所吩咐\ProperName{挪亞}的\yuentien 耶和華就把他關在方舟裡頭\chuan 
\bv{17}\ledleftnote{洪水氾濫四十日}洪水氾濫在地上四十天\chientien 水往上長\chientien 把方舟從地上漂起\chuan 
\bv{18}水勢浩大\chientien 在地上大大的往上長\chientien 方舟在水面上漂來漂去\chuan 
\bv{19}水勢在地上極其浩大\chientien 天下的高山都淹沒了\chuan 
\bv{20}水勢比山高過十五肘\chientien 山嶺都淹沒了\chuan 
\bv{21}凡在地上有血肉的動物\chientien 就是飛鳥\chientien 牲畜\chientien 走獸\chientien 和爬在地上的昆蟲\chientien 以及所有的人都死了\chuan 
\bv{22}凡在旱地上\chientien 鼻孔有氣息的生靈都死了\chuan 
\bv{23}凡地上各類的活物\chientien 連人帶牲畜\chientien 昆蟲\chientien 以及空中的飛鳥\chientien 都從地上除滅了\chientien 只留下\ProperName{挪亞}和那些與他同在方舟裡的\chuan 
\bv{24}水勢浩大\chientien 在地上共一百五十天\chuan 


\pend
\endnumbering
\beginnumbering
\pstart
\bchapter%
\bv{1}\ledleftnote{洪水消落} 神記念\ProperNameC{0}{0.5}{挪亞}\chientien 和\ProperName{挪亞}方舟裡的一切走獸牲畜\yuentien  神叫風吹地\chientien 水勢漸落\chuan 
\bv{2}淵源和天上的窗戶\chientien 都閉塞了\chientien 天上的大雨也止住了\chuan 
\bv{3}水從地上漸退\yuentien 過了一百五十天\chientien 水就漸消\chuan 
\bv{4}七月十七日\chientien 方舟停在\PlaceName{亞拉臘}山上\chuan 
\bv{5}水又漸消\chientien 到十月初一日\chientien 山頂都現出來了\chuan\Chuan
\bv{6}\ledleftnote{挪亞放烏鴉與鴿出方舟}過了四十天\chientien\ProperName{挪亞}開了方舟的窗戶\chientien 
\bv{7}放出一隻烏鴉去\chientien 那烏鴉飛來飛去\chientien 直到地上的水都乾了\chuan 
\bv{8}他又放出一隻鴿子去\chientien 要看看水從地上退了沒有\chuan 
\bv{9}但遍地上都是水\chientien 鴿子找不著落腳之地\chientien 就回到方舟\ProperName{挪亞}那裡\chientien\ProperName{挪亞}伸手把鴿子接進方舟來\chuan 
\bv{10}他又等了七天\chientien 再把鴿子從方舟放出去\yuentien 
\bv{11}到了晚上\chientien 鴿子回到他那裡\chientien 嘴裡叼著一個新擰下來的橄欖葉子\chientien\ProperName{挪亞}就知道地上的水退了\chuan 
\bv{12}他又等了七天\chientien 放出鴿子去\chientien 鴿子就不再回來了\chuan\Chuan
\bv{13}到\ProperName{挪亞}六百零一歲\chientien 正月初一日\chientien 地上的水都乾了\yuentien\ProperName{挪亞}撤去方舟的蓋觀看\chientien 便見地面上乾了\chuan 
\bv{14}到了二月二十七日\chientien 地就都乾了\chuan 
\bv{15}\ledleftnote{挪亞和全家出方舟} 神對\ProperName{挪亞}說\chientien 
\bv{16}你和你的妻子\chientien 兒子\chientien 兒婦\chientien 都可以出方舟\chuan 
\bv{17}在你那裡凡有血肉的活物\chientien 就是飛鳥\chientien 牲畜\chientien 和一切爬在地上的昆蟲\chientien 都要帶出來\chientien 叫他在地上多多滋生\chientien 大大興旺\chuan 
\bv{18}於是\ProperName{挪亞}和他的妻子\chientien 兒子\chientien 兒婦\chientien 都出來了\chuan 
\bv{19}一切走獸\chientien 昆蟲\chientien 飛鳥\chientien 和地上所有的動物\chientien 各從其類\chientien 也都出了方舟\chuan\Chuan
\bv{20}\ledleftnote{挪亞築壇獻祭}\ProperNameC{1}{0}{挪亞}為耶和華築了一座壇\chientien 拿各類潔淨的牲畜\chientien 飛鳥\chientien 獻在壇上為燔祭\chuan 
\bv{21}耶和華聞那馨香之氣\chientien 就心裡說\chientien 我不再因人的緣故咒詛地\chientien (人從小時心裡懷著惡念)也不再按著我纔行的\chientien 滅各種的活物了\chuan 
\bv{22}地還存留的時候\chientien 稼穡\chientien 寒暑\chientien 冬夏\chientien 晝夜\chientien 就永不停息了\chuan 

\pend
\endnumbering
\beginnumbering
\pstart
\bchapter%
\bv{1}\ledleftnote{ 神賜福給挪亞} 神賜福給\ProperName{挪亞}和他的兒子\chientien 對他們說\chientien 你們要生養眾多\chientien 遍滿了地\chuan 
\bv{2}凡地上的走獸\chientien 和空中的飛鳥\chientien 都必驚恐\chientien 懼怕你們\yuentien 連地上一切的昆蟲\chientien 並海裡一切的魚\chientien 都交付你們的手\chuan 
\bv{3}凡活著的動物\chientien 都可以作你們的食物\chientien 這一切我都賜給你們如同菜蔬一樣\chuan 
\bv{4}惟獨肉帶著血\chientien 那就是他的生命\chientien 你們不可喫\chuan 
\bv{5}流你們血害你們命的\chientien 無論是獸\chientien 是人\chientien 我必討他的罪\chientien 就是向各人的弟兄也是如此\chuan 
\bv{6}凡流人血的\chientien 他的血也必被人所流\chientien 因為 神造人\chientien 是照自己的形像造的\chuan
\bv{7}你們要生養眾多\chientien 在地上昌盛繁茂\chuan\Chuan
\bv{8} 神曉諭\ProperName{挪亞}和他的兒子說\chientien 
\bv{9}我與你們和你們的後裔立約\yuentien 
\bv{10}並與你們這裡的一切活物\chientien 就是飛鳥\chientien 牲畜\chientien 走獸\yuentien 凡從方舟裡出來的活物立約\chuan 
\bv{11}我與你們立約\chientien 凡有血肉的\chientien 不再被洪水滅絕\chientien 也不再有洪水毀壞地了\chuan 
\bv{12}\ledleftnote{立虹爲記} 神說\chientien 我與你們\chientien 並你們這裡的各樣活物所立的永約\chientien 是有記號的\chuan 
\bv{13}我把虹放在雲彩中\chientien 這就可作我與地立約的記號了\chuan 
\bv{14}我使雲彩蓋地的時候\chientien 必有虹現在雲彩中\yuentien 
\bv{15}我便記念我與你們\chientien 和各樣有血肉的活物所立的約\chientien 水就再不氾濫毀壞一切有血肉的物了\chuan 
\bv{16}虹必現在雲彩中\chientien 我看見\chientien 就要記念我與地上各樣有血肉的活物所立的永約\chuan 
\bv{17} 神對\ProperName{挪亞}說\chientien 這就是我與地上一切有血肉之物立約的記號了\chuan\Chuan
\bv{18}出方舟\ProperName{挪亞}的兒子\chientien 就是\ProperNameC{0}{0.5}{閃}\chientien\ProperNameC{0}{0.5}{含}\chientien\ProperNameC{0}{0.5}{雅弗}\yuentien\ProperName{含}是\ProperName{迦南}的父親\chuan 
\bv{19}這是\ProperName{挪亞}的三個兒子\yuentien 他們的後裔分散在全地\chuan\Chuan
\bv{20}\ProperNameC{1}{0}{挪亞}作起農夫來\chientien 栽了一個葡萄園\chuan 
\bv{21}他喝了園中的酒便醉了\yuentien 在帳棚裡赤著身子\chuan 
\bv{22}\ProperNameC{1}{0}{迦南}的父親\ProperNameC{0}{0.5}{含}\chientien 看見他父親赤身\chientien 就到外邊告訴他兩個弟兄\chuan 
\bv{23}於是\ProperName{閃}和\ProperNameC{0}{0.5}{雅弗}\chientien 拿件衣服搭在肩上\chientien 倒退著進去\chientien 給他父親蓋上\chientien 他們背著臉就看不見父親的赤身\chuan 
\bv{24}\ledleftnote{迦南受詛咒}\ProperNameC{1}{0}{挪亞}醒了酒\chientien 知道小兒子向他所作的事\chientien 
\bv{25}就說\chientien\ProperName{迦南}當受咒詛\chientien 必給他弟兄作奴僕的奴僕\chuan 
\bv{26}又說\chientien 耶和華閃的 神\chientien 是應當稱頌的\chientien 願\ProperName{迦南}作\ProperName{閃}的奴僕\chuan 
\bv{27}願 神使\ProperName{雅弗}擴張\chientien 使他住在閃的帳棚裡\yuentien 又願\ProperName{迦南}作他的奴僕\chuan\Chuan
\bv{28}洪水以後\chientien\ProperName{挪亞}又活了三百五十年\chuan 
\bv{29}\ProperNameC{1}{0}{挪亞}共活了九百五十歲就死了\chuan 

\pend
\endnumbering
\beginnumbering
\pstart
\bchapter%
\bv{1}\ledleftnote{閃含雅弗之後裔}\ProperNameC{0.5}{0}{挪亞}的兒子\ProperNameC{0}{0.5}{閃}\chientien\ProperNameC{0}{0.5}{含}\chientien\ProperNameC{0}{0.5}{雅弗}的後代\chientien 記在下面\yuentien 洪水以後\chientien 他們都生了兒子\chuan\Chuan
\bv{2}\ProperNameC{0.5}{0}{雅弗}的兒子是\ProperNameC{0}{0.5}{歌\allowbreak 篾}\chientien\ProperNameC{0}{0.5}{瑪各}\chientien\ProperNameC{0}{0.5}{瑪代}\chientien\ProperNameC{0}{0.5}{雅完}\chientien\ProperNameC{0}{0.5}{土巴}\chientien\ProperNameC{0}{0.5}{米設}\chientien\ProperNameC{0}{0.5}{提拉}\chientien 
\bv{3}\ProperNameC{0.5}{0}{歌篾}的兒子是\ProperNameC{0}{0.5}{亞實基拿}\chientien\ProperNameC{0}{0.5}{利法}\chientien\ProperNameC{0}{0.5}{陀迦瑪}\yuentien 
\bv{4}\ProperNameC{0.5}{0}{雅完}的兒子是\ProperNameC{0}{0.5}{以利沙}\chientien\ProperNameC{0}{0.5}{他施}\chientien\ProperName{基}\hss\linebreak\ProperNameC{0}{0.5}{提}\chientien\ProperNameC{0}{0.5}{多單}\chientien 
\bv{5}這些人的後裔\chientien 將各國的地土\chientien 海島\chientien 分開居住\chientien 各隨各的方言宗族立國\chuan\Chuan
\bv{6}含的兒子是\ProperNameC{0}{0.5}{古實}\chientien\ProperName{麥}\hss\linebreak\ProperNameC{0}{0.5}{西}\chientien\ProperNameC{0}{0.5}{弗}\chientien\ProperNameC{0}{0.5}{迦南}\chientien 
\bv{7}\ProperNameC{0.5}{0}{古實}的兒子是\ProperNameC{0}{0.5}{西巴}\chientien\ProperNameC{0}{0.5}{哈腓拉}\chientien\ProperNameC{0}{0.5}{撒弗他}\chientien\ProperNameC{0}{0.5}{拉瑪}\chientien\ProperNameC{0}{0.5}{撒弗提迦}\chientien\ProperName{拉瑪}的兒子是\ProperNameC{0}{0.5}{示巴}\chientien\ProperNameC{0}{0.5}{底但}\chuan 
\bv{8}\ProperNameC{0.5}{0}{古實}又生\ProperNameC{0}{0.5}{寧錄}\chientien\hss\linebreak 他為世上英雄之首\yuentien 
\bv{9}他在耶和華面前是個英勇的獵戶\chientien 所以俗語說\chientien 像\ProperName{寧錄}在耶和華面前是個英勇的獵戶\chuan 
\bv{10}他國的起頭是\PlaceNameC{0}{0.5}{巴別}\chientien\PlaceNameC{0}{0.5}{以力}\chientien\PlaceNameC{0}{0.5}{亞甲}\chientien\PlaceNameC{0}{0.5}{甲尼}\chientien 都在\PlaceName{示拿}地\chuan 
\bv{11}他從那地出來往\PlaceName{亞述}去\chientien 建造\PlaceNameC{0}{0.5}{尼尼微}\chientien\PlaceNameC{0}{0.5}{利河伯}\chientien\PlaceNameC{0}{0.5}{迦拉}\chientien 
\bv{12}和\PlaceNameC{0.1}{0.5}{尼尼微}\chientien\PlaceName{迦拉}中間的\PlaceNameC{0}{0.5}{利鮮}\chientien 這就是那大城\chuan 
\bv{13}\ProperNameC{1}{0}{麥西}生\PlaceName{路低}人\chientien\PlaceName{亞拿米}人\chientien\PlaceName{利哈比}人\chientien\PlaceName{拿弗土希}人\chientien 
\bv{14}\PlaceNameC{1.1}{0}{帕斯}\PlaceName{魯細}人\chientien\PlaceName{迦斯路希}人\chientien\PlaceName{迦斐託}人\chientien 從\PlaceName{迦斐託}出來的有\PlaceName{非利士}人\chuan\Chuan
\bv{15}\ProperNameC{1}{0}{迦南}生長子\ProperNameC{0}{0.5}{西頓}\chientien 又生\ProperNameC{0}{0.5}{赫}\chientien 
\bv{16}和\PlaceName{耶布斯}人\chientien\PlaceName{亞摩利}人\chientien\PlaceName{革迦撒}人\chientien 
\bv{17}\PlaceNameC{1.1}{0}{希未}人\chientien\PlaceName{亞基}人\chientien\PlaceName{西尼}人\chientien 
\bv{18}\PlaceNameC{1.1}{0}{亞瓦底}人\chientien\PlaceName{洗瑪利}人\chientien\PlaceName{哈馬}人\chientien 後來\PlaceName{迦南}的諸族分散了\chuan 
\bv{19}\PlaceNameC{1.1}{0}{迦南}的境界是從\PlaceName{西頓}向\PlaceName{基拉耳}的路上\chientien 直到\PlaceNameC{0}{0.5}{迦薩}\chientien 又向\PlaceNameC{0}{0.5}{所多瑪}\chientien\PlaceNameC{0}{0.5}{蛾摩拉}\chientien\PlaceNameC{0}{0.5}{押瑪}\chientien\PlaceName{洗扁}的路上\chientien 直到\PlaceNameC{0}{0.5}{拉沙}\yuentien 
\bv{20}這就是\ProperName{含}的後裔\chientien 各隨他們的宗族\chientien 方言\chientien 所住的地土\chientien 邦國\chuan\Chuan
\bv{21}\ProperNameC{1}{0}{雅弗}的哥哥\ProperNameC{0}{0.4}{閃}\chientien 是\ProperName{希伯}子孫之祖\yuentien 他也生了兒子\chuan 
\bv{22}閃的兒子是\ProperNameC{0}{0.5}{以攔}\chientien\ProperNameC{0}{0.5}{亞述}\chientien\ProperNameC{0}{0.5}{亞法撒}\chientien\ProperNameC{0}{0.5}{路德}\chientien\ProperNameC{0}{0.5}{亞蘭}\chuan 
\bv{23}\ProperNameC{1}{0}{亞蘭}的兒子是\ProperNameC{0}{0.5}{烏斯}\chientien\ProperNameC{0}{0.5}{戶勒}\chientien\ProperNameC{0}{0.5}{基帖}\chientien\ProperNameC{0}{0.5}{瑪施}\chuan 
\bv{24}\ProperNameC{1}{0}{亞法撒}生\ProperNameC{0}{0.5}{沙拉}\chientien\ProperNameC{0}{0.5}{沙拉}生\ProperName{希}\hss\linebreak\ProperNameC{0}{0.5}{伯}\chientien 
\bv{25}\ProperNameC{1}{0}{希伯}生了兩個兒子\chientien 一個名叫\ProperNameC{0}{0.5}{法勒}\chientien\chu{法勒就是分的意思}因為那時人就分地居住\yuentien\ProperName{法勒}的兄弟名叫\ProperNameC{0}{0.5}{約坍}\chuan 
\bv{26}\ProperNameC{1}{0}{約坍}生\ProperNameC{0}{0.5}{亞摩答}\chientien\ProperNameC{0}{0.5}{沙列}\chientien\ProperNameC{0}{0.5}{哈薩瑪非}\chientien\ProperNameC{0}{0.5}{耶拉}\chientien 
\bv{27}\ProperNameC{1}{0.5}{哈多蘭}\chientien\ProperNameC{0}{0.5}{烏薩}\chientien\ProperNameC{0}{0.5}{德拉}\chientien 
\bv{28}\ProperNameC{1}{0.5}{俄巴路}\chientien\ProperNameC{0}{0.5}{亞比瑪利}\chientien\ProperNameC{0}{0.5}{示巴}\chientien 
\bv{29}\ProperNameC{1}{0.5}{阿斐}\chientien\ProperNameC{0}{0.5}{哈腓拉}\chientien\ProperNameC{0}{0.5}{約巴}\chientien 這都是\ProperName{約坍}\hss\linebreak 的兒子\chuan 
\bv{30}他們所住的地方\chientien 是從\PlaceNameC{0}{0.5}{米沙}\chientien 直到\PlaceName{西發}東邊的山\chuan 
\bv{31}這就是\ProperName{閃}的子孫\chientien 各隨他們的宗族\chientien 方言\chientien 所住的地土\chientien 邦國\chuan\Chuan
\bv{32}這些都是\ProperName{挪亞}三個兒子的宗族\chientien 各隨他們的支派立國\chientien 洪水以後\chientien 他們在地上分為邦國\chuan 

\pend
\endnumbering
\beginnumbering
\pstart
\bchapter%
\bv{1}那時\chientien 天下人的口音言語\chientien 都是一樣\chuan 
\bv{2}他們往東邊遷移的時候\chientien 在\PlaceName{示拿}地遇見一片平原\chientien 就住在那裡\chuan 
\bv{3}他們彼此商量說\chientien 來罷\chientien 我們要作磚\chientien 把磚燒透了\chuan 他們就拿磚當石頭\chientien 又拿石漆當灰泥\chuan 
\bv{4}他們說\chientien 來罷\chientien 我們要建造一座城\chientien 和一座塔\chientien 塔頂通天\chientien 為要傳揚我們的名\chientien 免得我們分散在全地上\chuan 
\bv{5}耶和華降臨要看看世人所建造的城和塔\chuan 
\bv{6}耶和華說\chientien 看哪\chientien 他們成為一樣的人民\chientien 都是一樣的言語\chientien 如今既作起這事來\chientien 以後他們所要作的事\chientien 就沒有不成就的了\chuan 
\bv{7}\ledleftnote{變亂口音}我們下去\chientien 在那裡變亂他們的口音\chientien 使他們的言語\chientien 彼此不通\chuan 
\bv{8}於是耶和華使他們從那裡分散在全地上\yuentien 他們就停工\chientien 不造那城了\chuan 
\bv{9}因為耶和華在那裡變亂天下人的言語\chientien 使眾人分散在全地上\chientien 所以那城名叫\PlaceNameC{0}{0.5}{巴別}\chuan\chu{就是變亂的意思}\Chuan
\bv{10}\ProperNameC{0.5}{0}{閃}的後代記在下面\yuentien 洪水以後二年\chientien \ProperName{閃}一百歲生了\ProperNameC{0}{0.5}{亞法撒}\chuan 
\bv{11}閃生\ProperName{亞法撒}之後\chientien 又活了五百年\yuentien 並且生兒養女\chuan 
\bv{12}\ProperNameC{1}{0}{亞法撒}活到三十五歲\chientien 生了\ProperNameC{0}{0.5}{沙拉}\yuentien 
\bv{13}\ProperNameC{1}{0}{亞法撒}生\ProperName{沙拉}之後\chientien 又活了四百零三年\yuentien 並且生兒養女\chuan 
\bv{14}\ProperNameC{1}{0}{沙拉}活到三十歲\chientien 生了\ProperNameC{0}{0.5}{希伯}\yuentien 
\bv{15}沙\ProperName{拉}生\ProperName{希伯}之後\chientien 又活了四百零三年\yuentien 並且生兒養女\chuan 
\bv{16}\ProperNameC{1}{0}{希伯}活到三十四歲\chientien 生了\ProperNameC{0}{0.5}{法勒}\yuentien 
\bv{17}\ProperNameC{1}{0}{希伯}生\ProperName{法勒}之後\chientien 又活了四百三十年\yuentien 並且生兒養女\chuan 
\bv{18}\ProperNameC{1}{0}{法勒}活到三十歲\chientien 生了拉吳\yuentien 
\bv{19}\ProperNameC{1}{0}{法勒}生\ProperName{拉吳}之後\chientien 又活了二百零九年\yuentien 並且生兒養女\chuan 
\bv{20}\ProperNameC{1}{0}{拉吳}活到三十二歲\chientien 生了\ProperNameC{0}{0.5}{西鹿}\yuentien 
\bv{21}\ProperNameC{1}{0}{拉吳}生\ProperName{西鹿}之後\chientien 又活了二百零七年\yuentien 並且生兒養女\chuan 
\bv{22}\ProperNameC{1}{0}{西鹿}活到三十歲\chientien 生了\ProperNameC{0}{0.5}{拿鶴}\yuentien 
\bv{23}\ProperNameC{1}{0}{西鹿}生\ProperName{拿鶴}之後\chientien 又活了二百年\yuentien 並且生兒養女\chuan 
\bv{24}\ProperNameC{1}{0}{拿鶴}活到二十九歲\chientien 生了\ProperNameC{0}{0.5}{他拉}\yuentien 
\bv{25}拿\ProperName{鶴}生\ProperName{他拉}之後\chientien 又活了一百一十九年\yuentien 並且生兒養女\chuan 
\bv{26}\ProperNameC{1}{0}{他拉}活到七十歲\chientien 生了\ProperNameC{0}{0.5}{亞伯蘭}\chientien\ProperNameC{0}{0.5}{拿鶴}\chientien\ProperNameC{0}{0.5}{哈蘭}\chuan\Chuan
\bv{27}他\ProperName{拉}的後代\chientien 記在下面\yuentien 他拉生\ProperNameC{0}{0.5}{亞伯蘭}\chientien\ProperNameC{0}{0.5}{拿鶴}\chientien\ProperNameC{0}{0.5}{哈蘭}\yuentien\ProperName{哈蘭}生\ProperNameC{0}{0.5}{羅得}\chuan 
\bv{28}\ProperNameC{1}{0}{哈蘭}死在他的本地迦勒底的\PlaceNameC{0}{0.5}{吾珥}\chientien 在他父\hss\linebreak 親\ProperName{他拉}之先\chuan 
\bv{29}\ProperNameC{1}{0.5}{亞伯蘭}\chientien\ProperNameC{0}{0.5}{拿鶴}\chientien 各娶了妻\yuentien\ProperName{亞伯蘭}的妻子名叫\ProperNameC{0}{0.5}{撒萊}\yuentien\ProperName{拿鶴}的妻子名叫\ProperNameC{0}{0.5}{密迦}\chientien 是\ProperName{哈蘭}的女兒\yuentien \ProperName{哈}\hss\linebreak\ProperName{蘭}是\ProperName{密迦}和\ProperName{亦迦}的父親\chuan 
\bv{30}\ProperNameC{1}{0}{撒萊}不生育\chientien 沒有孩子\chuan 
\bv{31}\ledleftnote{亞伯蘭離吾珥往迦南}他拉帶著他兒子\ProperNameC{0}{0.5}{亞伯蘭}\chientien 和他孫子\ProperName{哈蘭}的兒子\ProperNameC{0}{0.5}{羅得}\chientien\hss\linebreak 並他兒婦\ProperName{亞伯蘭}的妻子\ProperNameC{0}{0.5}{撒萊}\chientien 出了\PlaceName{迦勒底}的\PlaceNameC{0}{0.5}{吾珥}\chientien 要往\PlaceName{迦南}地去\chientien 他們走到\PlaceName{哈蘭}就住在那裡\chuan 
\bv{32}他拉共活了二百零五歲\chientien 就死在\PlaceNameC{0}{0.5}{哈蘭}\chuan 

\pend
\endnumbering
\beginnumbering
\pstart
\bchapter%
\bv{1}\ledleftnote{應許萬族因亞伯蘭得福}耶和華對\ProperName{亞伯蘭}說\chientien 你要離開本地\chientien 本族\chientien 父家\chientien 往我所要指示你的地去\chuan 
\bv{2}我必叫你成為大國\yuentien 我必賜福給你\chientien 叫你的名為大\chientien 你也要叫別人得福\yuentien 
\bv{3}為你祝福的\chientien 我必賜福與他\chientien 那咒詛你的\chientien 我必咒詛他\chientien 地上的萬族都要因你得福\chuan 
\bv{4}\ProperNameC{0.5}{0}{亞伯蘭}就照著耶和華的吩咐去了\yuentien\ProperName{羅得}也和他同去\yuentien \ProperName{亞伯蘭}出\PlaceName{哈}\PlaceName{蘭}的時候\chientien 年七十五歲\chuan 
\bv{5}\ProperNameC{0.5}{0}{亞伯蘭}將他妻子\ProperNameC{0}{0.5}{撒萊}\chientien 和姪兒\ProperNameC{0}{0.5}{羅得}\chientien 連他們在哈蘭所積蓄的財物\chientien 所得的人口\chientien 都帶往\PlaceName{迦南}地去\yuentien 他們就到了\PlaceName{迦南}地\chuan 
\bv{6}\ProperNameC{0.5}{0}{亞伯蘭}經過那地\chientien 到了\PlaceName{示劍}地方\PlaceName{摩利}橡樹那裡\yuentien 那時\PlaceName{迦南}人住在那地\chuan 
\bv{7}耶和華向\ProperName{亞伯蘭}顯現\chientien 說\chientien 我要把這地賜給你的後裔\yuentien \ProperName{亞伯蘭}就在那裡為向他顯現的耶和華築了一座壇\chuan 
\bv{8}從那裡他又遷到\PlaceName{伯特利}東邊的山\chientien 支搭帳棚\yuentien 西邊是\PlaceNameC{0}{0.5}{伯特利}\chientien 東邊是\PlaceNameC{0}{0.5}{艾}\yuentien 他在那裡又為耶和華築了一座壇\chientien 求告耶和華的名\chuan 
\bv{9}後來\ProperName{亞伯蘭}又漸漸遷往南地去\chuan\Chuan
\bv{10}\ledleftnote{亞伯蘭因饑荒下埃及}那地遭遇饑荒\chientien 因饑荒甚大\chientien\ProperName{亞\allowbreak 伯蘭}就下\PlaceName{埃及}去\chientien 要在那裡暫居\chuan 
\bv{11}將近\PlaceName{埃及}就對他妻子\ProperName{撒萊}說\chientien 我知道你是容貌俊美的婦人\yuentien 
\bv{12}\PlaceNameC{1.1}{0}{埃及}人看見你必說\chientien 這是他的妻子\chientien 他們就要殺我\chientien 卻叫你存活\chuan 
\bv{13}求你說\chientien 你是我的妹子\chientien 使我因你得平安\chientien 我的命也因你存活\chuan 
\bv{14}及至\ProperName{亞伯蘭}到了\PlaceNameC{0}{0.5}{埃及}\chientien \PlaceName{埃及}人看見那婦人極其美貌\chuan 
\bv{15}\ProperNameC{1}{0}{法老}的臣宰看見了他\chientien 就在\ProperName{法老}面前誇獎他\yuentien 那婦人就被帶進\ProperName{法老}的宮去\chuan 
\bv{16}\ProperNameC{1}{0}{法老}因這婦人就厚待\ProperNameC{0}{0.5}{亞伯蘭}\chientien \ProperName{亞伯蘭}得了許多牛\chientien 羊\chientien 駱駝\chientien 公驢\chientien 母驢\chientien 僕婢\chuan 
\bv{17}耶和華因\ProperName{亞伯蘭}妻子\ProperName{撒萊}的緣故\chientien 降大災與\ProperName{法老}和他的全家\chuan 
\bv{18}\ledleftnote{法老責備亞伯蘭}\ProperNameC{1}{0}{法老}就召了\ProperName{亞伯蘭}來\chientien 說\chientien 你這向我作的是甚麼事呢\chientien 為甚麼沒有告訴我他是你的妻子\yuentien 
\bv{19}為甚麼說\chientien 他是你的妹子\chientien 以致我把他取來要作我的妻子\yuentien 現在你的妻子在這裡\chientien 可以帶他走罷\chuan 
\bv{20}於是\ProperName{法老}吩咐人將\ProperName{亞伯蘭}和他妻子\chientien 並他所有的都送走了\chuan 

\pend
\endnumbering
\beginnumbering
\pstart
\bchapter%
\bv{1}\ProperNameC{0.5}{0}{亞伯蘭}帶著他的妻子與\ProperNameC{0}{0.5}{羅得}\chientien 並一切所有的\chientien 都從\PlaceName{埃及}上南地去\chuan 
\bv{2}\ProperNameC{0.5}{0}{亞伯蘭}的金\chientien 銀\chientien 牲畜極多\chuan 
\bv{3}他從南地漸漸往\PlaceName{伯特利}去\chientien 到了\PlaceName{伯特利}和\PlaceName{艾}的中間\chientien 就是從前支搭帳棚的地方\chientien 
\bv{4}也是他起先築壇的地方\yuentien 他又在那裡求告耶和華的名\chuan 
\bv{5}與\ProperName{亞伯蘭}同行的羅得\chientien 也有牛群\chientien 羊群\chientien 帳棚\chuan 
\bv{6}那地容不下他們\chientien 因為他們的財物甚多\chientien 使他們不能同居\chuan 
\bv{7}當時\PlaceName{迦南}人\chientien 與\PlaceName{比利洗}人\chientien 在那地居住\yuentien \ProperName{亞伯蘭}的牧人\chientien 和\ProperName{羅得}的牧人相爭\chuan 
\bv{8}\ProperNameC{0.5}{0}{亞伯蘭}就對\ProperName{羅得}說\chientien 你我不可相爭\chientien 你的牧人和我的牧人也不可相爭\chientien 因為我們是骨肉\chuan\hss\linebreak\chu{原文作弟兄}
\bv{9}遍地不都在你眼前麼\yuentien 請你離開我\chientien 你向左\chientien 我就向右\chientien 你向右\chientien 我就向左\chuan 
\bv{10}\ledleftnote{亞伯蘭與羅得分離}\ProperNameC{1}{0}{羅得}舉目看見\PlaceName{約但}河的全平原\chientien 直到\PlaceNameC{0}{0.5}{瑣珥}\chientien 都是滋潤的\chientien 那地在耶和華未滅\PlaceNameC{0}{0.5}{所多瑪}\chientien \PlaceName{蛾摩拉}以先\chientien 如同耶和華的園子\chientien 也像\PlaceName{埃}\PlaceName{及}地\chuan 
\bv{11}於是\ProperName{羅得}選擇\PlaceName{約但}河的全平原\chientien 往東遷移\yuentien 他們就彼此分離了\chuan 
\bv{12}\ProperNameC{1}{0}{亞伯蘭}住在\PlaceName{迦南}地\chientien\ProperName{羅得}住在平原的城邑\chientien 漸漸挪移帳棚\chientien 直到\PlaceNameC{0}{0.5}{所多瑪}\chuan 
\bv{13}\PlaceNameC{1.1}{0}{所多瑪}人在耶和華面前罪大惡極\chuan\Chuan
\bv{14}\ProperNameC{1}{0}{羅得}離別\ProperName{亞伯蘭}以後\chientien 耶和華對\ProperName{亞伯蘭}說\chientien 從你所在的地方\chientien 你舉目向東西南北觀看\yuentien 
\bv{15}凡你所看見的一切地\chientien 我都要賜給你和你的後裔\chientien 直到永遠\yuentien 
\bv{16}我也要使你的後裔如同地上的塵沙那樣多\chientien 人若能數算地上的塵沙\chientien 纔能數算你的後裔\chuan 
\bv{17}你起來\chientien 縱橫走遍這地\chientien 因為我必把這地賜給你\chuan 
\bv{18}\ProperNameC{1}{0}{亞伯蘭}就搬了帳棚\chientien 來到\PlaceNameC{0}{0.1}{希伯崙}\PlaceNameC{0.1}{0}{幔利}的橡樹那裡居住\chientien 在那裡為耶和華築了一座壇\chuan 

\pend
\endnumbering
\beginnumbering
\pstart
\bchapter%
\bv{1}\ledleftnote{四王與五王戰}當\ProperName{暗拉非}作\PlaceName{示拿}王\chientien\ProperName{亞略}作\PlaceName{以拉撒}王\chientien\ProperName{基大老瑪}作\PlaceName{以攔}王\chientien\ProperName{提達}作\PlaceName{戈印}王的時候\chientien 
\bv{2}他們都攻打\PlaceName{所多瑪}王\ProperNameC{0}{0.5}{比拉}\chientien\PlaceName{蛾摩拉}王\ProperNameC{0}{0.5}{比沙}\chientien\PlaceName{押瑪}王\ProperNameC{0}{0.5}{示納}\chientien\PlaceName{洗扁}王\ProperNameC{0}{0.5}{善以別}\chientien 和\PlaceName{比拉}王\yuentien\PlaceName{比拉}就是\PlaceNameC{0}{0.5}{瑣珥}\chuan 
\bv{3}這五王都在\PlaceName{西訂}谷會合\yuentien\PlaceName{西訂}谷就是鹽海\chuan 
\bv{4}他們已經事奉\ProperName{基大老瑪}十二年\chientien 到十三年就背叛了\chuan 
\bv{5}十四年\ProperName{基大老\allowbreak 瑪}和同盟的王\chientien 都來在\PlaceNameC{0}{0.5}{亞特律加寧}\chientien 殺敗了\PlaceName{利乏音}人\chientien 在\PlaceName{哈麥}殺敗了\PlaceName{蘇西}人\chientien 在\PlaceName{沙微基列亭}殺敗了\PlaceName{以}\PlaceName{米}人\chientien 
\bv{6}在\PlaceName{何利}人的\PlaceName{西珥}山殺敗了\PlaceName{何利}人\chientien 一直殺到靠近曠野的\PlaceNameC{0}{0.5}{伊勒巴蘭}\chuan 
\bv{7}他們回到\PlaceNameC{0}{0.5}{安密巴}\chientien 就是\PlaceName{加低}\PlaceNameC{0}{0.5}{斯}\chientien 殺敗了\PlaceName{亞瑪力}全地的人\chientien 以及住在\PlaceName{哈洗遜他瑪}的\PlaceName{亞摩利}人\chuan 
\bv{8}於是\PlaceName{所多瑪}王\chientien\PlaceName{蛾摩拉}王\chientien\PlaceName{押瑪}王\chientien\PlaceName{洗扁}王\chientien 和\PlaceName{比拉}王\chientien (\PlaceName{比拉}就是\PlaceName{瑣珥})都出來\chientien 在\PlaceName{西訂}谷擺陣\chientien 與他們交戰\yuentien 
\bv{9}就是與\PlaceName{以攔}王\ProperNameC{0}{0.5}{基大老瑪}\chientien\PlaceName{戈印}王\ProperNameC{0}{0.5}{提達}\chientien\PlaceName{示拿}王\ProperNameC{0}{0.5}{暗拉非}\chientien\PlaceName{以拉撒}王\ProperName{亞略}交戰\yuentien 乃是四王與五王交戰\chuan 
\bv{10}\PlaceNameC{0.5}{0}{西訂}谷有許多石漆坑\yuentien \PlaceName{所多瑪}王\chientien 和\PlaceName{蛾}\PlaceName{摩拉}王逃跑\chientien 有掉在坑裡的\chientien 其餘的人都往山上逃跑\chuan 
\bv{11}四王就把\PlaceName{所多瑪}和\PlaceName{蛾摩拉}所有的財物\chientien 並一切的糧食\chientien 都擄掠去了\chuan 
\bv{12}又把\ProperName{亞伯蘭}的姪兒\ProperNameC{0}{0.5}{羅得}\chientien 和\ProperName{羅得}的財物擄掠去了\yuentien 當時\chientien\ProperName{羅得}正住在\PlaceNameC{0}{0.5}{所多瑪}\chuan\Chuan
\bv{13}有一個逃出來的人\chientien\ledleftnote{亞伯蘭救回羅得}告訴\PlaceName{希伯來}人\ProperNameC{0}{0.5}{亞伯蘭}\yuentien\ProperName{亞伯蘭}正住在\PlaceName{亞摩利}人\ProperName{幔利}的橡樹那裡\yuentien\ProperName{幔利}和\ProperNameC{0}{0.5}{以實各}\chientien 並\ProperName{亞乃}都是弟兄\chientien 曾與\ProperName{亞伯蘭}聯盟\chuan
\bv{14}\ProperNameC{1}{0}{亞伯蘭}聽見他姪兒\chu{原文作弟兄}被擄去\chientien 就率領他家裡生養的精練壯丁三百一十八人\chientien 直追到\PlaceNameC{0}{0.5}{但}\yuentien 
\bv{15}便在夜間\chientien 自己同僕人分隊殺敗敵人\chientien 又追到\PlaceName{大馬色}左邊的\PlaceNameC{0}{0.5}{何把}\yuentien 
\bv{16}將被擄掠的一切財物奪回來\chientien 連他姪兒\ProperName{羅得}和他的財物\chientien 以及婦女人民\chientien 也都奪回來\chuan\Chuan
\bv{17}\ProperNameC{1}{0}{亞伯蘭}殺敗\ProperNameC{0}{0.5}{基大老瑪}\chientien 和與他同盟的王回來的時候\chientien\PlaceName{所多瑪}王出來\chientien 在\PlaceName{沙微}谷迎接他\chientien\PlaceName{沙微}谷就是{王}谷\chuan 
\bv{18}\ledleftnote{麥基洗德爲亞伯蘭祝福}又有\PlaceName{撒冷}王\ProperNameC{0}{0.5}{麥基洗\allowbreak 德}\chientien 帶著餅和酒\chientien 出來迎接\yuentien 他是至高 神的祭司\chuan 
\bv{19}他為\ProperName{亞伯蘭}祝福\chientien 說\chientien 願天地的主\chientien 至高的 神\chientien 賜福與\ProperNameC{0}{0.5}{亞伯蘭}\yuentien 
\bv{20}至高的 神把敵人交在你手裡\chientien 是應當稱頌的\chuan\ProperName{亞伯蘭}就把所得的\chientien 拿出十分之一來\chientien 給\ProperNameC{0}{0.5}{麥基\allowbreak 洗德}\chuan 
\bv{21}\PlaceNameC{1.1}{0}{所多瑪}王對\ProperName{亞伯蘭}說\chientien 你把人口給我\chientien 財物你自己拿去罷\chuan 
\bv{22}\ProperNameC{1}{0}{亞伯蘭}對\PlaceName{所多瑪}王說\chientien 我已經向天地的主\chientien 至高的 神耶和華起誓\yuentien 
\bv{23}凡是你的東西\chientien 就是一根線\chientien 一根鞋帶\chientien 我都不拿\chientien 免得你說\chientien 我使\ProperName{亞伯蘭}富足\yuentien 
\bv{24}只有僕人所喫的\chientien 並與我同行的\ProperNameC{0}{0.5}{亞乃}\chientien\ProperNameC{0}{0.5}{以實各}\chientien\ProperNameC{0}{0.5}{幔利}\chientien 所應得的分\chientien 可以任憑他們拿去\chuan

\pend
\endnumbering
\beginnumbering
\pstart
\bchapter%
\bv{1}這事以後\chientien 耶和華在異象中有話對\ProperName{亞伯蘭}說\chientien \ProperName{亞伯蘭}你不要懼怕\chientien 我是你的盾牌\chientien 必大大的賞賜你\chuan 
\bv{2}\ProperNameC{0.5}{0}{亞伯蘭}說\chientien 主耶和華阿\chientien 我既無子\chientien 你還賜我甚麼呢\yuentien 並且要承受我家業的\chientien 是\PlaceName{大馬色}人\ProperNameC{0}{0.5}{以\allowbreak 利以謝}\chuan 
\bv{3}\ProperNameC{0.5}{0}{亞伯蘭}又說\chientien 你沒有給我兒子\chientien 那生在我家中的人\chientien 就是我的後嗣\chuan 
\bv{4}耶和華又有話對他說\chientien 這人必不成為你的後嗣\chientien 你本身所生的\chientien 纔成為你的後嗣\chuan 
\bv{5}\ledleftnote{ 神應許亞伯蘭之後裔多如衆星}於是領他走到外邊\chientien 說\chientien 你向天觀看\chientien 數算眾星\chientien 能數得過來麼\yuentien 又對他說\chientien 你的後裔將要如此\chuan 
\bv{6}\ProperNameC{0.5}{0}{亞伯蘭}信耶和華\chientien 耶和華就以此為他的義\chuan 
\bv{7}耶和華又對他說\chientien 我是耶和華\chientien 曾領你出了\PlaceName{迦勒底}的\PlaceNameC{0}{0.5}{吾珥}\chientien 為要將這地賜你為業\chuan 
\bv{8}\ProperNameC{0.5}{0}{亞伯蘭}說\chientien 主耶和華阿\chientien 我怎能知道必得這地為業呢\chuan 
\bv{9}他說\chientien 你為我取一隻三年的母牛\chientien 一隻三年的母山羊\chientien 一隻三年的公綿羊\chientien 一隻斑鳩\chientien 一隻雛鴿\chuan 
\bv{10}\ProperNameC{0.5}{0}{亞伯蘭}就取了這些來\chientien 每樣劈開分成兩半\chientien 一半對著一半的擺列\chientien 只有鳥沒有劈開\chuan 
\bv{11}有鷙鳥下來落在那死畜的肉上\chientien\ProperName{亞伯蘭}就把他嚇飛了\chuan\Chuan
\bv{12}日頭正落的時候\chientien\ProperName{亞伯蘭}沉沉的睡了\yuentien 忽然有驚人的大黑暗落在他身上\chuan 
\bv{13}耶和華對\ProperName{亞伯蘭}說\chientien 你要的確知道\chientien 你的後裔必寄居別人的地\chientien 又服事那地的人\yuentien 那地的人要苦待他們四百年\yuentien 
\bv{14}並且他們所要服事的那國\chientien 我要懲罰後來他們必帶著許多財物\chientien 從那裡出來\chuan 
\bv{15}但你要享大壽數\chientien 平平安安的歸到你列祖那裡\chientien 被人埋葬\chuan 
\bv{16}到了第四代\chientien 他們必回到此地\chientien 因為\PlaceName{亞摩利}人的罪孽\chientien 還沒有滿盈\chuan 
\bv{17}日落天黑\chientien 不料有冒煙的爐\chientien 並燒著的火把\chientien 從那些肉塊中經過\chuan 
\bv{18}當那日耶和華與\ProperName{亞伯蘭}立約\chientien 說\chientien 我已賜給你的後裔\chientien 從\PlaceName{埃及}河直到\PlaceName{伯拉}大河之地\yuentien 
\bv{19}就是\PlaceName{基尼}人\chientien\PlaceName{基尼洗}人\chientien\PlaceName{甲摩尼}人\chientien 
\bv{20}赫人\chientien\PlaceName{比利洗}人\chientien\PlaceName{利乏音}人\chientien 
\bv{21}\PlaceNameC{1.1}{0}{亞摩利}人\chientien\PlaceName{迦南}人\chientien\PlaceName{革迦撒}人\chientien\PlaceName{耶布斯}人\chientien 之地\chuan 

\pend
\endnumbering
\beginnumbering
\pstart
\bchapter%
\bv{1}\ProperNameC{0.5}{0}{亞伯蘭}的妻子\ProperName{撒萊}不給他生兒女\yuentien\ProperName{撒萊}有一個使女名叫\ProperNameC{0}{0.5}{夏甲}\chientien 是\PlaceName{埃及}人\chuan 
\bv{2}\ProperNameC{0.5}{0}{撒萊}對{亞伯\allowbreak 蘭}說\chientien 耶和華使我不能生育\chientien 求你和我的使女同房\chientien 或者我可以因他得孩子\chuan\chu{得孩子原文作被建立}\ProperName{亞伯蘭}聽從了\ProperName{撒萊}的話\yuentien 
\bv{3}於是\ProperName{亞伯蘭}的妻子\ProperNameC{0}{0.5}{撒萊}\chientien 將使女\PlaceName{埃及}人\ProperName{夏甲}給了丈夫為妾\yuentien 那時\ProperName{亞伯蘭}在\PlaceName{迦南}已經住了十年\chuan 
\bv{4}\ProperNameC{0.5}{0}{亞伯蘭}與\ProperName{夏甲}同房\chientien\ProperName{夏甲}就懷了孕\yuentien 他見自己有孕\chientien 就小看他的主母\chuan 
\bv{5}\ProperNameC{0.5}{0}{撒萊}對\ProperName{亞伯蘭}說\chientien 我因你受屈\chientien 我將我的使女放在你懷中\chientien 他見自己有了孕就小看我\chientien 願耶和華在你我中間判斷\chuan 
\bv{6}\ledleftnote{撒萊苦待夏甲}\ProperNameC{0.5}{0}{亞伯蘭}對\ProperName{撒\allowbreak 萊}說\chientien 使女在你手下\chientien 你可以隨意待他\yuentien\ProperName{撒萊}苦待他\chientien 他就從\ProperName{撒萊}面前逃走了\chuan\Chuan
\bv{7}耶和華的使者在曠野\chientien\PlaceName{書珥}路上的水泉旁遇見他\chientien 
\bv{8}對他說\chientien\ProperName{撒萊}的使女\ProperNameC{0}{0.5}{夏甲}\chientien 你從那裡來\chientien 要往那裡去\yuentien\ProperName{夏甲}說\chientien 我從我的主母\ProperName{撒萊}面前逃出來\chuan 
\bv{9}耶和華的使者對他說\chientien 你回到你主母那裡\chientien 服在他手下\chuan 
\bv{10}又說\chientien 我必使你的後裔極其繁多\chientien 甚至不可勝數\chuan 
\bv{11}並說\chientien 你如今懷孕要生一個兒子\chientien 可以給他起名叫\ProperNameC{0}{0.5}{以實瑪利}\chientien 因為耶和華聽見了你的苦情\chuan\chu{以實瑪利就是 神聽見的意思}
\bv{12}他為人必像野驢\yuentien 他的手要攻打人\chientien 人的手也要攻打他\chientien 他必住在眾弟兄的東邊\chuan 
\bv{13}\ProperNameC{1}{0}{夏甲}就稱那對他說話的耶和華為看顧人的 神\yuentien 因而說\chientien 在這裡我也看見那看顧我的麼\chuan 
\bv{14}所以這井名叫\PlaceNameC{0}{0.5}{庇耳拉海萊}\yuentien 這井正在\PlaceNameC{0}{0.5}{加低斯}\chientien 和\PlaceName{巴列}中間\chuan\Chuan
\bv{15}後來\ProperName{夏甲}給\ProperName{亞伯蘭}生了一個兒子\yuentien\ProperName{亞伯蘭}給他起名叫\ProperNameC{0}{0.5}{以實瑪利}\chuan 
\bv{16}\ProperNameC{1}{0}{夏甲}給\ProperName{亞伯蘭}生\ProperName{以實瑪利}的時候\chientien\ProperName{亞伯蘭}年八十六歲\chuan 

\pend
\endnumbering
\beginnumbering
\pstart
\bchapter%
\bv{1}\ledleftnote{ 神易亞伯蘭名爲亞伯拉罕}\ProperNameC{0.5}{0}{亞伯蘭}年九十九歲的時候\chientien 耶和華向他顯現\chientien 對他說\chientien 我是全能的 神\chientien 你當在我面前作完全人\chuan 
\bv{2}我就與你立約\chientien 使你\kenten{的後裔}極其繁多\chuan 
\bv{3}\ProperNameC{0.5}{0}{亞伯蘭}俯伏在地\chientien  神又對他說\chientien 
\bv{4}我與你立約\chientien 你要作多國的父\chuan 
\bv{5}從此以後\chientien 你的名不再叫\ProperNameC{0}{0.5}{亞伯蘭}\chientien 要叫\ProperNameC{0}{0.5}{亞伯拉罕}\chientien 因為我已立你作多國的父\chuan 
\bv{6}我必使你\kenten{的後裔}極其繁多\chientien 國度從你而立\chientien 君王從你而出\chuan 
\bv{7}我要與你並你世世代代的後裔堅立我的約\chientien 作永遠的約\chientien 是要作你和你後裔的 神\chuan 
\bv{8}我要將你現在寄居的地\chientien 就是\PlaceName{迦南}全地\chientien 賜給你和你的後裔\chientien 永遠為業\yuentien 我也必作他們的 神\chuan\Chuan
\bv{9}\ledleftnote{始定割禮} 神又對\ProperName{亞伯拉罕}說\chientien 你和你的後裔必世世代代遵守我的約\chuan 
\bv{10}你們所有的男子\chientien 都要受割禮\chientien 這就是我與你\chientien 並你的後裔所立的約\chientien 是你們所當遵守的\chuan 
\bv{11}你們都要受割禮\chu{受割禮原文作割陽}\linebreak\chu{皮十四二十三二十四二十五節同}這是我與你們立約的證據\chuan 
\bv{12}你們世世代代的男子\chientien 無論是家裡生的\chientien 是在你後裔之外用銀子從外人買的\chientien 生下來第八日\chientien 都要受割禮\chuan 
\bv{13}你家裡生的\chientien 和你用銀子買的\chientien 都必須受割禮\yuentien 這樣\chientien 我的約就立在你們肉體上\chientien 作永遠的約\chuan 
\bv{14}但不受割禮的男子\chientien 必從民中剪除\chientien 因他背了我的約\chuan\Chuan
\bv{15} 神又對亞伯拉罕說\chientien\ledleftnote{易撒萊名爲撒拉}你的妻子\ProperNameC{0}{0.5}{撒萊}\chientien 不可再叫\ProperNameC{0}{0.5}{撒萊}\chientien 他的名要叫\ProperNameC{0}{0.5}{撒拉}\chuan 
\bv{16}我必賜福給他\chientien 也要使你從他得一個兒子\chientien 我要賜福給他\chientien 他也要作多國之\kenten{母}\chientien 必有百姓的君王從他而出\chuan 
\bv{17}\ProperNameC{1}{0}{亞伯拉罕}就俯伏在地喜笑\chientien 心裡說\chientien 一百歲的人\chientien 還能得孩子麼\yuentien\ProperName{撒拉}已經九十歲了\chientien 還能生養麼\chuan 
\bv{18}\ProperNameC{1}{0}{亞伯拉罕}對 神說\chientien 但願\ProperName{以實瑪利}活在你面前\chuan 
\bv{19}\ledleftnote{應許生以撒} 神說\chientien 不然\chientien 你妻子\ProperName{撒拉}要給你生一個兒子\chientien 你要給他起名叫\ProperNameC{0}{0.5}{以撒}\chientien 我要與他堅定所立的約\chientien 作他後裔永遠的約\chuan 
\bv{20}至於\ProperNameC{0}{0.5}{以實瑪利}\chientien 我也應允你\chientien 我必賜福給他\chientien 使他昌盛極其繁多\chientien 他必生十二個族長\chientien 我也要使他成為大國\chuan 
\bv{21}到明年這時節\chientien\ProperName{撒拉}必給你生\ProperName{以撒}\chientien 我要與他堅定所立的約\chuan\Chuan
\bv{22} 神和\ProperName{亞伯拉罕}說完了話\chientien 就離開他上升去了\chuan 
\bv{23}\ledleftnote{始受割禮}正當那日\chientien\ProperName{亞伯拉罕}遵著 神的命\chientien 給他的兒子\ProperName{以實瑪利}和家裡的一切男子\chientien 無論是在家裡生的\chientien 是用銀子買的\chientien 都行了割禮\chuan 
\bv{24}\ProperNameC{1}{0}{亞伯拉罕}受割禮的時候\chientien 年九十九歲\chuan 
\bv{25}他兒子\ProperName{以實瑪利}受割禮的時候\chientien 年十三歲\chuan 
\bv{26}正當那日\chientien\ProperName{亞伯拉罕}和他兒子\ProperNameC{0}{0.5}{以實瑪利}\chientien 一同受了割禮\chuan 
\bv{27}家裡所有的人\chientien 無論是在家裡生的\chientien 是用銀子從外人買的\chientien 也都一同受了割禮\chuan 

\pend
\endnumbering
\beginnumbering
\pstart
\bchapter%
\bv{1}\ledleftnote{亞伯拉罕接待天使}耶和華在\ProperName{幔利}橡樹那裡\chientien 向\ProperName{亞伯拉罕}顯現出來\yuentien 那時正熱\chientien\ProperName{亞伯拉罕}坐在帳棚門口\chuan 
\bv{2}舉目觀看\chientien 見有三個人在對面站著\yuentien 他一見\chientien 就從帳棚門口跑去迎接他們\chientien 俯伏在地\chientien 
\bv{3}說\chientien 我主\chientien 我若在你眼前蒙恩\chientien 求你不要離開僕人往前去\chuan 
\bv{4}容我拿點水來\chientien 你們洗洗腳\chientien 在樹下歇息歇息\yuentien 
\bv{5}我再拿一點餅來\chientien 你們可以加添心力\chientien 然後往前去\chientien 你們既到僕人這裡來\chientien 理當如此\chuan 他們說\chientien 就照你說的行罷\chuan 
\bv{6}\ProperNameC{0.5}{0}{亞伯拉罕}急忙進帳棚見撒拉說\chientien 你速速拿三細亞細麵調和作餅\chuan 
\bv{7}\ProperNameC{0.5}{0}{亞伯拉罕}又跑到牛群裡\chientien 牽了一隻又嫩又好的牛犢來\chientien 交給僕人\chientien 僕人急忙預備好了\chuan 
\bv{8}\ProperNameC{0.5}{0}{亞伯拉罕}又取了奶油和奶\chientien 並預備好的牛犢來\chientien 擺在他們面前\chientien 自己在樹下站在旁邊\chientien 他們就喫了\chuan\Chuan
\bv{9}\ledleftnote{應許撒拉生子}他們問\ProperName{亞伯拉罕}說\chientien 你妻子\ProperName{撒拉}在那裡\chientien 他說\chientien 在帳棚裡\chuan 
\bv{10}三人中有一位說\chientien 到明年這時候\chientien 我必要回到你這裡\chientien 你的妻子\ProperName{撒拉}必生一個兒子\yuentien\ProperName{撒拉}在那人後邊的帳棚門口\chientien 也聽見了這話\chuan 
\bv{11}\ProperNameC{1}{0}{亞伯拉罕}和\ProperName{撒拉}年紀老邁\chientien\ProperName{撒拉}的月經已斷絕了\chuan 
\bv{12}\ProperNameC{1}{0}{撒拉}心裡暗笑\chientien 說\chientien 我既已衰敗\chientien 我主也老邁\chientien 豈能有這喜事呢\chuan 
\bv{13}耶和華對\ProperName{亞伯拉罕}說\chientien\ProperName{撒拉}為甚麼暗笑\chientien 說\chientien 我既已年老\chientien 果真能生養嗎\chuan 
\bv{14}耶和華豈有難成的事麼\yuentien 到了日期\chientien 明年這時候\chientien 我必回到你這裡\chientien\ProperName{撒拉}必生一個兒子\chuan 
\bv{15}\ProperNameC{1}{0}{撒拉}就害怕\chientien 不承認\chientien 說\chientien 我沒有笑\yuentien 那位說\chientien 不然\chientien 你實在笑了\chuan\Chuan
\bv{16}三人就從那裡起行\chientien 向\PlaceName{所多瑪}觀看\chientien\ProperName{亞伯拉罕}也與他們同行\chientien 要送他們一程\chuan 
\bv{17}耶和華說\chientien 我所要作的事\chientien 豈可瞞著\ProperName{亞伯拉罕}呢\yuentien 
\bv{18}\ProperNameC{1}{0}{亞伯拉罕}必要成為強大的國\chientien 地上的萬國都必因他得福\chuan 
\bv{19}我眷顧他\chientien 為要叫他吩咐他的眾子\chientien 和他的眷屬\chientien 遵守我的道\chientien 秉公行義\chientien 使我所應許\ProperName{亞伯拉罕}的話都成就了\chuan 
\bv{20}\ledleftnote{ 神將滅所多瑪蛾摩拉}耶和華說\chientien\PlaceName{所多瑪}和\PlaceName{蛾摩拉}的罪惡甚重\chientien 聲聞於我\chuan 
\bv{21}我現在要下去\chientien 察看他們所行的\chientien 果然盡像那達到我耳中的聲音一樣麼\yuentien 若是不然\chientien 我也必知道\chuan\Chuan
\bv{22}二人轉身離開那裡\chientien 向\PlaceName{所多瑪}去\chientien 但\ProperName{亞伯拉罕}仍舊站在耶和華面前\chuan 
\bv{23}\ledleftnote{亞伯拉罕爲所多瑪祈求}\ProperNameC{1}{0}{亞伯拉罕}近前來說\chientien 無論善惡\chientien 你都要剿滅麼\chuan 
\bv{24}假若那城裡有五十個義人\chientien 你還剿滅那地方麼\chuan 不為城裡這五十個義人饒恕其中的人麼\chuan 
\bv{25}將義人與惡人同殺\chientien 將義人與惡人一樣看待\chientien 這斷不是你所行的\yuentien 審判全地的主\chientien 豈不行公義麼\chuan 
\bv{26}耶和華說\chientien 我若在\PlaceName{所多}\PlaceName{瑪}城裡見有五十個義人\chientien 我就為他們的緣故\chientien 饒恕那地方的眾人\chuan 
\bv{27}\ProperNameC{1}{0}{亞伯拉罕}說\chientien 我雖然是灰塵\chientien 還敢對主說話\yuentien 
\bv{28}假若這五十個義人短了五個\chientien 你就因為短了五個毀滅全城麼\yuentien 他說\chientien 我在那裡若見有四十五個\chientien 也不毀滅那城\chuan 
\bv{29}\ProperNameC{1}{0}{亞伯拉罕}又對他說\chientien 假若在那裡見有四十個怎麼樣呢\yuentien 他說\chientien 為這四十個的緣故\chientien 我也不作這事\chuan 
\bv{30}\ProperNameC{1}{0}{亞伯拉罕}說\chientien 求主不要動怒\chientien 容我說\yuentien 假若在那裡見有三十個怎麼樣呢\yuentien 他說\chientien 我在那裡若見有三十個\chientien 我也不作這事\chuan 
\bv{31}\ProperNameC{1}{0}{亞伯拉罕}說\chientien 我還敢對主說話\chientien 假若在那裡見有二十個怎麼樣呢\yuentien 他說\chientien 為這二十個的緣故\chientien 我也不毀滅那城\chuan 
\bv{32}\ProperNameC{1}{0}{亞伯拉罕}說\chientien 求主不要動怒\chientien 我再說這一次\chientien 假若在那裡見有十個呢\yuentien 他說\chientien 為這十個的緣故\chientien 我也不毀滅那城\chuan 
\bv{33}耶和華與\ProperName{亞伯拉罕}說完了話就走了\yuentien\ProperName{亞伯拉罕}也回到自己的地方去了\chuan 

\pend
\endnumbering
\beginnumbering
\pstart
\bchapter%
\bv{1}\ledleftnote{羅得接待二天使}那兩個天使晚上到了\PlaceNameC{0}{0.5}{所多瑪}\yuentien\ProperName{羅得}正坐在\PlaceName{所多瑪}城門口\yuentien 看見他們\chientien 就起來迎接\chientien 臉伏於地下拜\chientien 
\bv{2}說\chientien 我主阿\chientien 請你們到僕人家裡洗洗腳\chientien 住一夜\chientien 清早起來再走\yuentien 他們說\chientien 不\chientien 我們要在街上過夜\chuan 
\bv{3}\ProperNameC{0.5}{0}{羅得}切切的請他們\chientien 他們這纔進去到他屋裡\yuentien\ProperName{羅得}為他們預備筵席\chientien 烤無酵餅\chientien 他們就喫了\chuan 
\bv{4}他們還沒有躺下\chientien\PlaceName{所多瑪城}裡各處的人\chientien 連老帶少\chientien 都來圍住那房子\yuentien 
\bv{5}呼叫\ProperName{羅得}說\chientien 今日晚上到你這裡來的人在哪裡呢\yuentien 把他們帶出來\chientien 任我們所為\chuan 
\bv{6}\ProperNameC{0.5}{0}{羅得}出來\chientien 把門關上\chientien 到眾人那裡\chientien 
\bv{7}說\chientien 眾弟兄請你們不要作這惡事\chuan 
\bv{8}我有兩個女兒\chientien 還是處女\chientien 容我領出來任憑你們的心願而行\chientien 只是這兩個人既然到我舍下\chientien 不要向他們作甚麼\chuan 
\bv{9}眾人說\chientien 退去罷\yuentien 又說\chientien 這個人來寄居\chientien 還想要作官哪\yuentien 現在我們要害你比害他們更甚\chientien 眾人就向前擁擠\ProperNameC{0}{0.5}{羅得}\chientien 要攻破房門\chuan 
\bv{10}\ledleftnote{天使救援羅得}只是那二人伸出手來\chientien 將羅得拉進屋去\chientien 把門關上\yuentien 
\bv{11}並且使門外的人\chientien 無論老少\chientien 眼都昏迷\yuentien 他們摸來摸去\chientien 總尋不著房門\chuan\Chuan
\bv{12}二人對\ProperName{羅得}說\chientien 你這裡還有甚麼人麼\yuentien 無論是女婿\chientien 是兒女\chientien 和這城中一切屬你的人\chientien 你都要將他們從這地方帶出去\chuan 
\bv{13}我們要毀滅這地方\chientien 因為城內罪惡的聲音\chientien 在耶和華面前甚大\chientien 耶和華差我們來\chientien 要毀滅這地方\chuan 
\bv{14}羅得就出去\chientien 告訴娶了他女兒的女婿們\chientien\chu{娶了}\hss\linebreak\chu{或作將要娶}說\chientien 你們起來離開這地方\chientien 因為耶和華要毀滅這城\yuentien 他女婿們卻以為他說的是戲言\chuan 
\bv{15}天明了\chientien 天使催逼\ProperName{羅得}說\chientien 起來\chientien 帶著你的妻子\chientien 和你在這裡的兩個女兒出去\chientien 免得你因這城裡的罪惡\chientien 同被剿滅\chuan 
\bv{16}\ledleftnote{引領羅得避災}但\ProperName{羅得}遲延不走\yuentien 二人因為耶和華憐恤羅得\chientien 就拉著他的手\chientien 和他妻子的手\chientien 並他兩個女兒的手\chientien 把他們領出來\chientien 安置在城外\chuan 
\bv{17}領他們出來以後\chientien 就說\chientien 逃命罷\yuentien 不可回頭看\chientien 也不可在平原站住\chientien 要往山上逃跑\chientien 免得你被剿滅\chuan 
\bv{18}羅得對他們說\chientien 我主阿\chientien 不要如此\yuentien 
\bv{19}你僕人已經在你眼前蒙恩\chientien 你又向我顯出莫大的慈愛\chientien 救我的性命\chientien 我不能逃到山上去\chientien 恐怕這災禍臨到我\chientien 我便死了\chuan 
\bv{20}看哪\chientien 這座城又小又近\chientien 容易逃到\chientien 這不是一個小的麼\yuentien 求你容我逃到那裡\chientien 我的性命就得存活\chuan 
\bv{21}天使對他說\chientien 這事我也應允你\chientien 我不傾覆你所說的這城\chientien 
\bv{22}你要速速的逃到那城\chientien 因為你還沒有到那裡我不能作甚麼\chuan 因此那城名叫\PlaceNameC{0}{0.5}{瑣珥}\chuan\chu{瑣珥就是小的}\hss\linebreak\chu{意思}\Chuan
\bv{23}羅得到了\PlaceNameC{0}{0.5}{瑣珥}\chientien 日頭已經出來了\chuan 
\bv{24}\ledleftnote{毀滅所多瑪蛾摩拉}當時耶和華將硫磺與火\chientien 從天上耶和華那裡\chientien 降與\PlaceName{所多瑪}和\PlaceName{蛾}\PlaceNameC{0}{0.5}{摩拉}\chientien 
\bv{25}把那些城\chientien 和全平原\chientien 並城裡所有的居民\chientien 連地上生長的\chientien 都毀滅了\chuan 
\bv{26}\ProperNameC{1}{0}{羅得}的妻子在後邊回頭一看\chientien 就變成了一根鹽柱\chuan 
\bv{27}\ProperNameC{1}{0}{亞伯拉罕}清早起來\chientien 到了他從前站在耶和華面前的地方\chientien 
\bv{28}向\PlaceNameC{0}{0.5}{所多瑪}\chientien 和\PlaceNameC{0}{0.5}{蛾摩拉}\chientien 與平原的全地觀看\yuentien 不料\chientien 那地方煙氣上騰\chientien 如同燒窰一般\chuan\Chuan
\bv{29}當 神毀滅平原諸城的時候\chientien 他記念\ProperNameC{0}{0.5}{亞伯\allowbreak 拉罕}\chientien 正在傾覆羅得所住之城的時候\chientien 就打發\ProperName{羅得}從傾覆之中出來\chuan\Chuan
\bv{30}\ProperNameC{1}{0}{羅得}因為怕住在\PlaceNameC{0}{0.5}{瑣珥}\chientien 就同他兩個女兒從瑣珥上去住在山裡\yuentien 他和兩個女兒住在一個洞裡\chuan 
\bv{31}大女兒對小女兒說\chientien 我們的父親老了\chientien 地上又無人按著世上的常規\chientien 進到我們這裡\yuentien 
\bv{32}來\chientien 我們可以叫父親喝酒\chientien 與他同寢\yuentien 這樣\chientien 我們好從他存留後裔\chuan 
\bv{33}於是那夜他們叫父親喝酒\chientien 大女兒就進去和他父親同寢\yuentien 他幾時躺下\chientien 幾時起來\chientien 父親都不知道\chuan 
\bv{34}第二天\chientien 大女兒對小女兒說\chientien 我昨夜與父親同寢\chientien 今夜我們再叫他喝酒\chientien 你可以進去與他同寢\yuentien 這樣\chientien 我們好從父親存留後裔\yuentien 
\bv{35}於是那夜他們又叫父親喝酒\chientien 小女兒起來與他父親同寢\yuentien 他幾時躺下\chientien 幾時起來\chientien 父親都不知道\chuan 
\bv{36}這樣\chientien\ProperName{羅得}的兩個女兒\chientien 都從他父親懷了孕\chuan 
\bv{37}大女兒生了兒子\chientien 給他起名叫\ProperNameC{0}{0.5}{摩押}\chientien 就是現今\PlaceName{摩押}人的始祖\chuan 
\bv{38}小女兒也生了兒子\chientien 給他起名叫\ProperNameC{0}{0.5}{便亞米}\chientien 就是現今\PlaceName{亞捫}人的始祖\chuan 

\pend
\endnumbering
\beginnumbering
\pstart
\bchapter%
\bv{1}\ProperNameC{0.5}{0}{亞伯拉罕}從那裡向南地遷去\chientien 寄居在\PlaceName{加低斯}和\PlaceName{書珥}中間的\PlaceNameC{0}{0.5}{基拉耳}\chuan 
\bv{2}\ProperNameC{0.5}{0}{亞伯拉罕}稱他的妻\ProperName{撒拉}為妹子\chientien\PlaceName{基拉耳}王\ProperName{亞比米勒}差人把\ProperName{撒拉}取了去\chuan 
\bv{3}但夜間 神來在夢中\chientien 對\ProperName{亞比米勒}說\chientien 你是個死人哪\chientien 因為你取了那女人來\chientien 他原是別人的妻子\chuan 
\bv{4}\ProperNameC{0.5}{0}{亞比米勒}卻還沒有親近\ProperNameC{0}{0.5}{撒拉}\yuentien 他說\chientien 主阿\chientien 連有義的國你也要毀滅麼\chuan 
\bv{5}那人豈不是自己對我說\chientien 他是我的妹子麼\yuentien 就是女人也自己說\chientien 他是我的哥哥\yuentien 我作這事\chientien 是心正手潔的\chuan 
\bv{6} 神在夢中對他說\chientien 我知道你作這事是心中正直\chientien 我也攔阻了你\chientien 免得你得罪我\chientien 所以我不容你沾著他\chuan 
\bv{7}現在你把這人的妻子歸還他\chientien 因為他是先知\chientien 他要為你禱告\chientien 使你存活\yuentien 你若不歸還他\chientien 你當知道\chientien 你和你所有的人\chientien 都必要死\chuan\Chuan
\bv{8}\ledleftnote{亞比米勒責亞伯拉罕}\ProperNameC{0.5}{0}{亞比米勒}清早起來\chientien 召了眾臣僕來\chientien 將這些事都說給他們聽\chientien 他們都甚懼怕\chuan 
\bv{9}\ProperNameC{0.5}{0}{亞比米勒}召了\ProperName{亞伯拉罕}來\chientien 對他說\chientien 你怎麼向我這樣行呢\chientien 我在甚麼事上得罪了你\chientien 你竟使我和我國裡的人陷在大罪裡\yuentien 你向我行不當行的事了\chuan 
\bv{10}\ProperNameC{0.5}{0}{亞比米勒}又對\ProperName{亞伯拉罕}說\chientien 你見了甚麼纔作這事呢\chuan 
\bv{11}\ProperNameC{1}{0}{亞伯拉罕}說\chientien 我以為這地方的人總不懼怕 神\chientien 必為我妻子的緣故殺我\chuan 
\bv{12}況且他也實在是我的妹子\chientien 他與我是同父異母\chientien 後來作了我的妻子\chuan 
\bv{13}當 神叫我離開父家飄流在外的時候\chientien 我對他說\chientien 我們無論走到甚麼地方\chientien 你可以對人說\chientien 他是我的哥哥\yuentien 這就是你待我的恩典了\chuan 
\bv{14}\ProperNameC{1}{0}{亞比米勒}把牛羊\chientien 僕婢賜給亞伯拉罕\chientien 又把他的妻子撒拉歸還他\chuan 
\bv{15}\ProperNameC{1}{0}{亞比米勒}又說\chientien 看哪\chientien 我的地都在你面前\chientien 你可以隨意居住\chuan 
\bv{16}又對\ProperName{撒拉}說\chientien 我給你哥哥一千銀子\chientien 作為你在閤家人面前遮羞的\chientien\chu{羞原文作眼}你就在眾人面前沒有不是了\chuan 
\bv{17}\ProperNameC{1}{0}{亞伯拉罕}禱告 神\chientien  神就醫好了\ProperName{亞比米勒}和他的妻子\chientien 並他的眾女僕\chientien 他們便能生育\chuan 
\bv{18}因耶和華為\ProperName{亞伯拉罕}的妻子\ProperName{撒拉}的緣故\chientien 已經使\ProperName{亞比米勒}家中的婦人\chientien 不能生育\chuan 

\pend
\endnumbering
\beginnumbering
\pstart
\bchapter%
\bv{1}耶和華按著先前的話\chientien 眷顧\ProperNameC{0}{0.5}{撒拉}\chientien 便照他所說的給\ProperName{撒拉}成就\chuan 
\bv{2}\ledleftnote{以撒生}當\ProperName{亞伯拉罕}年老的時候\chientien\ProperName{撒拉}懷了孕\yuentien 到 神所說的日期\chientien 就給\ProperName{亞伯拉罕}生了一個兒子\chuan 
\bv{3}\ProperNameC{0.5}{0}{亞伯拉罕}給\ProperName{撒拉}所生的兒子起名叫\ProperNameC{0}{0.5}{以撒}\chuan 
\bv{4}\ProperNameC{0.5}{0}{以撒}生下來第八日\chientien\ProperName{亞伯拉罕}照著 神所吩咐的\chientien 給\ProperName{以撒}行了割禮\chuan 
\bv{5}他兒子\ProperName{以撒}生的時候\chientien\ProperName{亞\allowbreak 伯拉罕}年一百歲\chuan 
\bv{6}\ProperNameC{0.5}{0}{撒拉}說\chientien  神使我喜笑\chientien 凡聽見的必與我一同喜笑\chuan 
\bv{7}又說\chientien 誰能預先對\ProperName{亞伯拉罕}說\chientien\ProperName{撒\allowbreak 拉}要乳養嬰孩呢\chientien 因為在他年老的時候\chientien 我給他生了一個兒子\chuan\Chuan
\bv{8}\ledleftnote{夏甲與其子被逐}孩子漸長\chientien 就斷了奶\yuentien\ProperName{以撒}斷奶的日子\chientien\ProperName{亞伯拉罕}設擺豐盛的筵席\chuan 
\bv{9}當時\chientien\ProperName{撒拉}看見\PlaceName{埃及}人\ProperName{夏甲}給\ProperName{亞伯拉罕}所生的兒子戲笑\chientien 
\bv{10}就對\ProperName{亞伯拉\allowbreak 罕}說\chientien 你把這使女\chientien 和他兒子趕出去\chientien 因為這使女的兒子\chientien 不可與我的兒子\ProperNameC{0}{0.5}{以撒}\chientien 一同承受產業\chuan 
\bv{11}\ProperNameC{1}{0}{亞伯拉\allowbreak 罕}因他兒子的緣故很憂愁\chuan 
\bv{12} 神對\ProperName{亞伯拉罕}說\chientien 你不必為這童子和你的使女憂愁\chientien 凡撒拉對你說的話\chientien 你都該聽從\yuentien 因為從\ProperName{以撒}生的\chientien 纔要稱為你的後裔\chuan 
\bv{13}至於使女的兒子\chientien 我也必使他\kenten{的後裔}成立一國\chientien 因為他是你所生的\chuan 
\bv{14}\ProperNameC{1}{0}{亞伯拉罕}清早起來\chientien 拿餅和一皮袋水\chientien 給了\ProperNameC{0}{0.5}{夏甲}\chientien 搭在他的肩上\chientien 又把孩子\kenten{交給他}\chientien 打發他走\yuentien\ProperName{夏甲}就走了\chientien 在\PlaceName{別是巴}的曠野走迷了路\chuan 
\bv{15}皮袋的水用盡了\chientien\ProperName{夏甲}就把孩子撇在小樹底下\chientien 
\bv{16}自己走開約有一箭之遠\chientien 相對而坐\chientien 說\chientien 我不忍見孩子死\chientien 就相對而坐\chientien 放聲大哭\chuan 
\bv{17}\ledleftnote{天使安慰夏甲} 神聽見童子的聲音\yuentien  神的使者從天上呼叫\ProperName{夏甲}說\chientien\ProperNameC{0}{0.5}{夏甲}\chientien 你為何這樣呢\chientien 不要害怕\chientien  神已經聽見童子的聲音了\chuan 
\bv{18}起來\chientien 把童子抱在懷中\chientien\chu{懷原文作手}我必使他\kenten{的後裔}成為大國\chuan 
\bv{19} 神使\ProperName{夏甲}的眼睛明亮\chientien 他就看見一口水井\chientien 便去將皮袋盛滿了水\chientien 給童子喝\chuan 
\bv{20} 神保佑童子\chientien 他就漸長\chientien 住在曠野\chientien 成了弓箭手\chuan 
\bv{21}他住在\PlaceName{巴蘭}的曠野\chientien 他母親從\PlaceName{埃及}地給他娶了一個妻子\chuan\Chuan
\bv{22}當那時候\chientien\ProperName{亞比米勒}同他軍長非各\chientien 對\ProperName{亞伯拉罕}說\chientien 凡你所行的事\chientien 都有 神的保佑\chuan 
\bv{23}我願你如今在這裡指著 神對我起誓\chientien 不要欺負我與我的兒子\chientien 並我的子孫\chientien 我怎樣厚待了你\chientien 你也要照樣厚待我\chientien 與你所寄居這地的民\chuan 
\bv{24}\ProperNameC{1}{0}{亞伯拉罕}說\chientien 我情願起誓\chuan 
\bv{25}從前\ProperName{亞比米勒}的僕人\chientien 霸佔了一口水井\chientien\ProperName{亞伯拉罕}為這事指責\ProperNameC{0}{0.5}{亞比米勒}\chuan 
\bv{26}\ProperNameC{1}{0}{亞比米勒}說\chientien 誰作這事我不知道\chientien 你也沒有告訴我\chientien 今日我纔聽見了\chuan 
\bv{27}\ledleftnote{亞伯拉罕與亞比米勒立\linebreak 約}\ProperNameC{1}{0}{亞伯拉罕}把羊和牛給了\ProperNameC{0}{0.5}{亞比米勒}\chientien 二人就彼此立約\chuan 
\bv{28}\ProperNameC{1}{0}{亞伯拉罕}把七隻母羊羔另放在一處\chuan 
\bv{29}\ProperNameC{1}{0}{亞比米勒}問\ProperName{亞伯拉罕}說\chientien 你把這七隻母羊羔另放在一處\chientien 是甚麼意思呢\chuan 
\bv{30}他說\chientien 你要從我手裡受這七隻母羊羔\chientien 作我挖這口井的證據\chuan 
\bv{31}所以他給那地方起名叫\PlaceNameC{0}{0.5}{別是巴}\chientien 因為他們二人在那裡起了誓\chuan\chu{別是巴就是盟誓的井}
\bv{32}他們在\PlaceName{別是巴}立了約\yuentien\ProperName{亞比米勒}就同他軍長非各\chientien 起身回\PlaceName{非利士}地去了\chuan 
\bv{33}\ProperNameC{1}{0}{亞伯拉罕}在別是巴栽上一棵垂絲柳樹\chientien 又在那裡求告耶和華永生 神的名\chuan 
\bv{34}\ProperNameC{1}{0}{亞伯拉罕}在\PlaceName{非利士}人的地寄居了多日\chuan 

\pend
\endnumbering
\beginnumbering
\pstart
\bchapter%
\bv{1}這些事以後\chientien  神要試驗\ProperNameC{0}{0.5}{亞伯拉罕}\chientien 就呼叫他說\chientien\ProperNameC{0}{0.5}{亞伯拉罕}\chientien 他說\chientien 我在這裡\chuan 
\bv{2} 神說\chientien 你帶著你的兒子\chientien 就是你獨生的兒子\chientien 你所愛的\ProperNameC{0}{0.5}{以撒}\chientien 往\PlaceName{摩利亞}地去\chientien 在我所要指示你的山上\chientien 把他獻為燔祭\chuan 
\bv{3}\ProperNameC{0.5}{0}{亞伯拉罕}清早起來\chientien 備上驢\chientien 帶著兩個僕人和他兒子\ProperNameC{0}{0.5}{以撒}\chientien 也劈好了燔祭的柴\chientien 就起身往 神所指示他的地方去了\chuan 
\bv{4}到了第三日\chientien\ProperName{亞伯拉罕}舉目遠遠的看見那地方\chuan 
\bv{5}\ProperNameC{0.5}{0}{亞伯拉罕}對他的僕人說\chientien 你們和驢在此等候\chientien 我與童子往那裡去拜一拜\chientien 就回到你們這裡來\chuan 
\bv{6}\ProperNameC{0.5}{0}{亞伯拉罕}把燔祭的柴放在他兒子\ProperName{以撒}身上\chientien 自己手裡拿著火與刀\yuentien 於是二人同行\chuan 
\bv{7}\ProperNameC{0.5}{0}{以撒}對他父親\ProperName{亞伯拉罕}說\chientien 父親哪\chuan\ProperName{亞伯拉罕}說\chientien 我兒\chientien 我在這裡\chuan\ProperName{以撒}說\chientien 請看\chientien 火與柴都有了\chientien 但燔祭的羊羔在哪裡呢\chuan 
\bv{8}\ProperNameC{0.5}{0}{亞伯拉罕}說\chientien 我兒\chientien  神必自己預備作燔祭的羊羔\yuentien 於是二人同行\chuan\Chuan
\bv{9}他們到了 神所指示的地方\chientien\ProperName{亞伯拉罕}在那裡築壇\chientien 把柴擺好\chientien 捆綁他的兒子\ProperNameC{0}{0.5}{以撒}\chientien 放在壇的柴上\chuan 
\bv{10}\ProperNameC{0.5}{0}{亞伯拉罕}就伸手拿刀\chientien 要殺他的兒子\chuan 
\bv{11}耶和華的使者從天上呼叫他說\chientien\ProperNameC{0}{0.5}{亞伯拉\allowbreak 罕}\chientien\ProperNameC{0}{0.5}{亞伯拉罕}\chientien 他說\chientien 我在這裡\chuan 
\bv{12}天使說\chientien 你不可在這童子身上下手\chientien 一點不可害他\yuentien 現在我知道你是敬畏 神的了\chientien 因為你沒有將你的兒子\chientien 就是你獨生的兒子\chientien 留下不給我\chuan 
\bv{13}\ProperNameC{1}{0}{亞伯拉罕}舉目觀看\chientien 不料\chientien 有一隻公羊\chientien 兩角扣在稠密的小樹中\chientien\ProperNameC{1}{0}{亞伯拉罕}就取了那隻公羊來\chientien 獻為燔祭\chientien 代替他的兒子\chuan 
\bv{14}\ProperNameC{1}{0}{亞伯拉罕}給那地方起名叫耶和華以勒\chientien\chu{意思就是耶和華必預備}直到今日人還說\chientien 在耶和華的山上必有預備\chuan 
\bv{15}耶和華的使者第二次從天上呼叫\ProperName{亞伯拉罕}說\chientien 
\bv{16}耶和華說你既行了這事\chientien 不留下你的兒子\chientien 就是你獨生的兒子\chientien 我便指著自己起誓說\chientien 
\bv{17}論福\chientien 我必賜大福給你\chientien 論子孫\chientien 我必叫你的子孫多起來\chientien 如同天上的星\chientien 海邊的沙\chientien 你子孫必得著仇敵的城門\chuan 
\bv{18}並且地上萬國都必因你的後裔得福\chientien 因為你聽從了我的話\yuentien 
\bv{19}於是\ProperName{亞伯拉罕}回到他僕人那裡\chientien 他們一同起身往\PlaceName{別是巴}去\chientien\ProperName{亞伯拉罕}就住在\PlaceNameC{0}{0.5}{別是巴}\chuan\Chuan
\bv{20}這事以後\chientien 有人告訴\ProperName{亞伯拉罕}說\chientien\ProperName{密迦}給你兄弟\ProperName{拿鶴}生了幾個兒子\chientien 
\bv{21}長子是\ProperNameC{0}{0.5}{烏斯}\chientien 他的兄弟是\ProperNameC{0}{0.5}{布斯}\chientien 和\ProperName{亞蘭}的父親\ProperNameC{0}{0.5}{基母利}\yuentien 
\bv{22}並\ProperName{基薛}\chientien\ProperNameC{0}{0.5}{哈\allowbreak 瑣}\chientien\ProperNameC{0}{0.5}{必達}\chientien\ProperNameC{0}{0.5}{益拉}\chientien\ProperNameC{0}{0.5}{彼土利}\chientien (\ProperName{彼土利}生\ProperName{利百加})
\bv{23}這八個人\chientien 都是\ProperName{密迦}給\ProperName{亞伯拉罕}的兄弟\ProperName{拿鶴}生的\chuan 
\bv{24}\ProperNameC{1}{0}{拿鶴}的\hss\linebreak 妾名叫\ProperNameC{0}{0.5}{流瑪}\chientien 生了\ProperNameC{0}{0.5}{提八}\chientien\ProperNameC{0}{0.5}{迦含}\chientien\ProperNameC{0}{0.5}{他轄}\chientien 和\ProperNameC{0}{0.5}{瑪迦}\chuan 

\pend
\endnumbering
\beginnumbering
\pstart
\bchapter%
\bv{1}\ProperNameC{0.5}{0}{撒拉}享壽一百二十七歲\chientien 這是\ProperName{撒拉}一生的歲數\chuan 
\bv{2}\ProperNameC{0.5}{0}{撒拉}死在\PlaceName{迦南}地的\PlaceNameC{0}{0.5}{基列亞巴}\chientien 就是\PlaceNameC{0}{0.5}{希伯崙}\chientien\ProperName{亞伯拉罕}為他哀慟哭號\chuan 
\bv{3}後來\ProperName{亞伯拉罕}從死人面前起來\chientien 對\PlaceName{赫}人說\chientien 
\bv{4}我在你們中間是外人\chientien 是寄居的\chientien 求你們在這裡給我一塊地\chientien 我好埋葬我的死人\chientien 使他不在我眼前\chuan 
\bv{5}\PlaceNameC{0.6}{0}{赫}人回答\ProperName{亞伯拉罕}說\chientien 
\bv{6}我主請聽\chientien 你在我們中間是一位尊大的王子\chientien 只管在我們最好的墳地裡埋葬你的死人\chientien 我們沒有一人不容你在他的墳地裡埋葬你的死人\chuan 
\bv{7}\ProperNameC{0.5}{0}{亞伯拉罕}就起來\chientien 向那地的\PlaceName{赫}人下拜\chuan 
\bv{8}對他們說\chientien 你們若有意叫我埋葬我的死人\chientien 使他不在我眼前\chientien 就請聽我的話\chientien 為我求\ProperName{瑣轄}的兒子\ProperNameC{0}{0.5}{以弗崙}\chientien 
\bv{9}把田頭上那\PlaceName{麥比拉}洞給我\chientien 他可以按著足價賣給我\chientien 作我在你們中間的墳地\chuan 
\bv{10}當時\ProperName{以弗崙}正坐在\PlaceName{赫}人中間\yuentien 於是\PlaceName{赫}人\ProperNameC{0}{0.5}{以弗崙}\chientien 在城門出入的\PlaceName{赫}人\chientien 面前對\ProperName{亞伯拉罕}說\chientien 
\bv{11}不然\chientien 我主請聽\chientien 我送給你這塊田\chientien 連田間的洞\chientien 也送給你\chientien 在我同族的人面前都給你\chientien 可以埋葬你的死人\chuan 
\bv{12}\ProperNameC{1}{0}{亞伯拉罕}就在那地的人民面前下拜\chuan 
\bv{13}在他們面前對\ProperName{以弗\allowbreak 崙}說\chientien 你若應允\chientien 請聽我的話\chientien 我要把田價給你\chientien 求你收下\chientien 我就在那裡埋葬我的死人\chuan 
\bv{14}\ProperNameC{1}{0}{以弗崙}回答\ProperName{亞伯\allowbreak 拉罕}說\chientien 
\bv{15}我主請聽\chientien 值四百舍客勒銀子的一塊田\chientien 在你我中間還算甚麼呢\chientien 只管埋葬你的死人罷\chuan 
\bv{16}\ProperNameC{1}{0}{亞伯\allowbreak 拉罕}聽從了\ProperNameC{0}{0.5}{以弗崙}\chientien 照著他在\PlaceName{赫}人面前所說的話\chientien 把買賣通用的銀子\chientien 平了四百舍客勒給\ProperNameC{0}{0.5}{以弗崙}\chuan\Chuan
\bv{17}\ledleftnote{撒拉埋葬}於是\PlaceNameC{0}{0.5}{麥比拉}\chientien \PlaceNameC{0}{0.5}{幔利}前\chientien\ProperName{以弗崙}的那塊田\chientien 和其中的洞\chientien 並田間四圍的樹木\chientien 
\bv{18}都定準歸與\ProperNameC{0}{0.5}{亞伯拉罕}\chientien 乃是他在\PlaceName{赫}人面前\chientien 並城門出入的人面前買妥的\chuan 
\bv{19}此後\chientien\ProperName{亞伯拉罕}把他妻子\ProperName{撒拉}埋葬在\PlaceName{迦南}地\PlaceName{幔利}前的\PlaceName{麥}\PlaceName{比拉}田間的洞裡\chientien\PlaceName{幔利}就是\PlaceNameC{0}{0.5}{希伯崙}\chuan 
\bv{20}從此\chientien 那塊田\chientien 和田間的洞\chientien 就藉著\PlaceName{赫}人定準\chientien 歸與\ProperName{亞伯拉罕}作墳地\chuan{}\relax%
\pend
\endnumbering
\beginnumbering
\pstart
\bchapter%
\bv{1}\ledleftnote{亞伯拉罕遣僕爲子娶妻}\ProperNameC{0.5}{0}{亞伯拉罕}年紀老邁\chientien 向來在一切事上\chientien 耶和華都賜福給他\chuan 
\bv{2}\ProperNameC{0.5}{0}{亞伯拉罕}對管理他全業最老的僕人說\chientien 請你把手放在我大腿底下\chuan 
\bv{3}我要叫你指著耶和華天地的主起誓\chientien 不要為我兒子娶這\PlaceName{迦南}地中的女子為妻\chuan 
\bv{4}你要往我本地本族去\chientien 為我的兒子\ProperName{以撒}娶一個妻子\chuan 
\bv{5}僕人對他說\chientien 倘若女子不肯跟我到這地方來\chientien 我必須將你的兒子帶回你原出之地麼\chuan 
\bv{6}\ProperNameC{0.5}{0}{亞伯拉罕}對他說\chientien 你要謹慎\chientien 不要帶我的兒子回那裡去\yuentien 
\bv{7}耶和華天上的主\chientien 曾帶領我離開父家和本族的地\chientien 對我說話向我起誓\chientien 說\chientien 我要將這地賜給你的後裔\yuentien 他必差遣使者在你面前\chientien 你就可以從那裡為我兒子娶一個妻子\chuan 
\bv{8}倘若女子不肯跟你來\chientien 我使你起的誓就與你無干了\chientien 只是不可帶我的兒子回那裡去\chuan 
\bv{9}僕人就把手放在他主人\ProperName{亞伯拉罕}的大腿底下\chientien 為這事向他起誓\chuan\Chuan
\bv{10}那僕人從他主人的駱駝裡取了十匹駱駝\chientien 並帶些他主人各樣的財物\chientien 起身往\PlaceName{米所波大米}去\chientien 到了\ProperName{拿鶴}的城\chuan 
\bv{11}天將晚\chientien 眾女子出來打水的時候\chientien 他便叫駱駝跪在城外的水井那裡\chuan 
\bv{12}他說\chientien 耶和華我主人\ProperName{亞伯拉罕}的 神阿\chientien 求你施恩給我主人\ProperNameC{0}{0.5}{亞伯拉罕}\chientien 使我今日遇見好機會\chuan 
\bv{13}我現今站在井旁\chientien 城內居民的女子們正出來打水\chuan 
\bv{14}我向那一個女子說\chientien 請你拿下水瓶來\chientien 給我水喝\chuan 他若說\chientien 請喝\chientien 我也給你的駱駝喝\chientien 願那女子就作你所預定給你僕人\ProperName{以撒}的妻\chientien 這樣\chientien 我便知道你施恩給我主人了\chuan 
\bv{15}話還沒有說完\chientien 不料\chientien\ProperName{利百加}肩頭上扛著水瓶出來\chientien\ProperName{利百加}是\ProperName{彼土利}所生的\chientien\ProperName{彼土利}是\ProperName{亞伯\allowbreak 拉罕}兄弟\ProperName{拿鶴}妻子\ProperName{密迦}的兒子\chuan 
\bv{16}那女子容貌極其俊美\chientien 還是處女\chientien 也未曾有人親近他\chientien 他下到井旁打滿了瓶\chientien 又上來\yuentien 
\bv{17}僕人跑上前去迎著他說\chientien 求你將瓶裡的水給我一點喝\chuan 
\bv{18}女子說\chientien 我主請喝\yuentien 就急忙拿下瓶來\chientien 托在手上給他喝\chuan 
\bv{19}女子給他喝了\chientien 就說\chientien 我再為你的駱駝打水\chientien 叫駱駝也喝足\chuan 
\bv{20}他就急忙把瓶裡的水倒在槽裡\chientien 又跑到井旁打水\chientien 就為所有的駱駝打上水來\chuan 
\bv{21}那人定睛看他\chientien 一句話也不說\chientien 要曉得耶和華賜他通達的道路沒有\chuan 
\bv{22}駱駝喝足了\chientien 那人就拿一個金環\chientien 重半舍客勒\chientien 兩個金鐲\chientien 重十舍客勒\chientien 給了那女子\chientien 
\bv{23}說\kern0.5zw\chientien\kern-0.5zw 請告訴我\chientien 你是誰的女兒\chientien 你父親家裡有我們住宿的地方沒有\chuan 
\bv{24}女子說\chientien 我是\ProperName{密迦}與\ProperName{拿鶴}之子\ProperName{彼土利}的女兒\yuentien 
\bv{25}又說\chientien 我們家裡足有糧草\chientien 也有住宿的地方\chuan 
\bv{26}那人就低頭向耶和華下拜\chientien 
\bv{27}說\kern0.5zw\chientien\kern-0.5zw 耶和華我主人\ProperName{亞伯拉罕}的 神是應當稱頌的\chientien 因他不斷的以慈愛誠實待我主人\yuentien 至於我\chientien 耶和華在路上引領我\chientien 直走到我主人的兄弟家裡\chuan\Chuan
\bv{28}\ledleftnote{拉班迎接僕人}女子跑回去\chientien 照著這些話告訴他母親和他家裡的人\chuan 
\bv{29}利\kern0.5zw\bv{30}\kern-0.5zw\ProperNameC{0}{0.3}{百}\ProperNameC{0.5}{0}{加}有一個哥哥\chientien 名叫\ProperNameC{0}{0.5}{拉班}\chientien 看見金環\chientien 又看見金鐲在他妹子的手上\chientien 並聽見他妹子\ProperName{利百加}的話\chientien 說\chientien 那人對我如此如此說\chientien\ProperName{拉班}就跑出來往井旁去\chientien 到那人跟前\chientien 見他仍站在駱駝旁邊的井旁那裡\yuentien%
\bv{31}便對他說\chientien 你這蒙耶和華賜福的\chientien 請進來\chientien 為甚麼站在外邊\chientien 我已經收拾了房屋\chientien 也為駱駝預備了地方\chuan 
\bv{32}那人就進了\ProperName{拉班}的家\yuentien\ProperName{拉班}卸了駱駝\chientien 用草料餵上\chientien 拿水給那人和跟隨的人洗腳\chuan 
\bv{33}\ledleftnote{僕人述說來意}把飯擺在他面前\chientien 叫他喫\chientien 他卻說\chientien 我不喫\chientien 等我說明白我的事情再喫\chientien 拉班說\chientien 請說\chuan 
\bv{34}他說\chientien 我是\ProperName{亞伯拉罕}的僕人\chuan 
\bv{35}耶和華大大地賜福給我主人\chientien 使他昌大\yuentien 又賜給他羊群\chientien 牛群\chientien 金銀\chientien 僕婢\chientien 駱駝\chientien 和驢\chuan 
\bv{36}我主人的妻子\ProperName{撒拉}年老的時候\chientien 給我主人生了一個兒子\yuentien 我主人也將一切所有的都給了這個兒子\chuan 
\bv{37}我主人叫我起誓說\chientien 你不要為我兒子娶\PlaceName{迦南}地的女子為妻\yuentien 
\bv{38}你要往我父家\chientien 我本族那裡去\chientien 為我的兒子娶一個妻子\chuan 
\bv{39}我對我主人說\chientien 恐怕女子不肯跟我來\chuan 
\bv{40}他就說\chientien 我所事奉的耶和華必要差遣他的使者與你同去\chientien 叫你的道路通達\yuentien 你就得以在我父家\chientien 我本族那裡\chientien 給我的兒子娶一個妻子\chuan 
\bv{41}只要你到了我本族那裡\chientien 我使你起的誓\chientien 就與你無干\chientien 他們若不把女子交給你\chientien 我使你起的誓也與你無干\chuan 
\bv{42}我今日到了井旁\chientien 便說\chientien 耶和華我主人\ProperName{亞伯拉罕}的 神阿\chientien 願你叫我所行的道路通達\yuentien 
\bv{43}我如今站在井旁\chientien 對那一個出來打水的女子說\chientien 請你把你瓶裡的水給我一點喝\yuentien 
\bv{44}他若說\chientien 你只管喝\chientien 我也為你的駱駝打水\yuentien 願那女子就作耶和華給我主人兒子所預定的妻\chuan 
\bv{45}我心裡的話還沒有說完\chientien\ProperName{利百加}就出來\chientien 肩頭上扛著水瓶\chientien 下到井旁打水\yuentien 我便對他說\chientien 請你給我水喝\chuan 
\bv{46}他就急忙從肩頭上拿下瓶來\chientien 說\chientien 請喝\chientien 我也給你的駱駝喝\yuentien 我便喝了\yuentien 他又給我的駱駝喝了\chuan 
\bv{47}我問他說\chientien 你是誰的女兒\chientien 他說\chientien 我是\ProperName{密迦}與\ProperName{拿鶴}之子\ProperName{彼土利}的女兒\chientien 我就把環子戴在他鼻子上\chientien 把鐲子戴在他兩手上\chuan 
\bv{48}隨後我低頭向耶和華下拜\chientien 稱頌耶和華我主人\ProperName{亞伯拉罕}的 神\chientien 因為他引導我走合式的道路\chientien 使我得著我主人兄弟的孫女\chientien 給我主人的兒子為妻\chuan 
\bv{49}現在你們若願以慈愛誠實待我主人\chientien 就告訴我\yuentien 若不然\chientien 也告訴我\chientien 使我可以或向左\chientien 或向右\chuan\Chuan
\bv{50}\ledleftnote{拉班彼土利允諾}\ProperNameC{1}{0}{拉班}和\ProperName{彼土利}回答說\chientien 這事乃出於耶和華\chientien 我們不能向你說好說歹\chuan 
\bv{51}看哪\chientien\ProperName{利百加}在你面前\chientien 可以將他帶去\chientien 照著耶和華所說的\chientien 給你主人的兒子為妻\chuan 
\bv{52}亞伯拉罕的僕人聽見他們這話\chientien 就向耶和華俯伏在地\chuan 
\bv{53}當下僕人拿出金器\chientien 銀器\chientien 和衣服送給\ProperNameC{0}{0.5}{利百加}\chientien 又將寶物送給他哥哥\chientien 和他母親\chuan 
\bv{54}僕人和跟從他的人\chientien 喫了喝了\chientien 住了一夜\chientien 早晨起來\chientien 僕人就說\chientien 請打發我回我主人那裡去罷\chuan 
\bv{55}\ProperNameC{1}{0}{利百加}的哥哥和他母親說\chientien 讓女子同我們再住幾天\chientien 至少十天\chientien 然後他可以去\chuan 
\bv{56}僕人說\chientien 耶和華既賜給我通達的道路\chientien 你們不要耽誤我\chientien 請打發我走\chientien 回我主人那裡去罷\chuan 
\bv{57}他們說\chientien 我們把女子叫來問問他\chientien 
\bv{58}就叫了\ProperName{利百加}來\chientien 問他說\chientien 你和這人同去麼\chientien\ProperName{利百加}說\chientien 我去\chuan 
\bv{59}於是他們打發妹子利百加和他的乳母\chientien 同\ProperName{亞伯拉罕}的僕人\chientien 並跟從僕人的\chientien 都走了\chuan 
\bv{60}他們就給\ProperName{利百加}祝福\chientien 說\chientien 我們的妹子阿\chientien 願你作千萬人的母\chientien 願你的後裔\chientien 得著仇敵的城門\chuan 
\bv{61}\ProperNameC{1}{0}{利百加}和他的使女們起來\chientien 騎上駱駝\chientien 跟著那僕人\yuentien 僕人就帶著利百加走了\chuan 
\bv{62}那時\chientien\ProperName{以撒}住在南地\chientien 剛從\PlaceName{庇耳拉海萊}回來\chuan 
\bv{63}天將晚\chientien\ProperName{以撒}出來在田間默想\yuentien 舉目一看\chientien 見來了些駱駝\chuan 
\bv{64}\ProperNameC{1}{0}{利百加}舉目看見\ProperNameC{0}{0.5}{以撒}\chientien 就急忙下了駱駝\yuentien 
\bv{65}問那僕人說\chientien 這田間走來迎接我們的是誰\chientien 僕人說\chientien 是我的主人\yuentien\ProperName{利百加}就拿帕子蒙上臉\chuan 
\bv{66}僕人就將所辦的一切事\chientien 都告訴\ProperNameC{0}{0.5}{以撒}\chuan 
\bv{67}\ledleftnote{以撒娶利百加爲妻}\ProperNameC{1}{0}{以撒}便領\ProperName{利百加}進了他母親\ProperName{撒拉}的帳棚\chientien 娶了他為妻\yuentien 並且愛他\chuan\ProperName{以撒}自從他母親不在了\chientien 這纔得了安慰\chuan

\pend
\endnumbering
\beginnumbering
\pstart
\bchapter%
\bv{1}\ledleftnote{亞伯拉罕繼娶基土拉}\ProperNameC{0.5}{0}{亞伯拉罕}又娶了一妻\chientien 名叫\ProperNameC{0}{0.5}{基土拉}\chientien 
\bv{2}\ProperNameC{0.5}{0}{基土}拉給他生了\ProperNameC{0}{0.5}{心蘭}\chientien\ProperNameC{0}{0.5}{約珊}\chientien\ProperNameC{0}{0.5}{米但}\chientien\ProperNameC{0}{0.5}{米甸}\chientien\ProperNameC{0}{0.5}{伊施巴}\chientien 和\ProperNameC{0}{0.5}{書亞}\chuan 
\bv{3}\ProperNameC{0.5}{0}{約珊}生了\ProperNameC{0}{0.5}{示巴}\chientien 和\ProperNameC{0}{0.5}{底但}\chuan\ProperName{底但}的子孫\chientien 是\PlaceName{亞書利}族\chientien\PlaceName{利都是}族\chientien 和\PlaceName{利烏米}族\chuan
\bv{4}\ProperNameC{0.5}{0}{米甸}的兒子是\ProperNameC{0}{0.5}{以法}\chientien\ProperNameC{0}{0.5}{以弗}\chientien\ProperNameC{0}{0.5}{哈諾}\chientien\ProperNameC{0}{0.5}{亞比大}\chientien 和\ProperNameC{0}{0.5}{以勒大}\chuan 這都是\ProperName{基土拉}的子孫\chuan 
\bv{5}\ProperNameC{0.5}{0}{亞伯拉罕}將一切所有的都給了\ProperNameC{0}{0.5}{以撒}\chuan 
\bv{6}\ProperNameC{0.5}{0}{亞伯拉罕}把財物分給他庶出的眾子\chientien 趁著自己還在世的時候\chientien 打發他們離開他的兒子\ProperName{以撒}往東方去\chuan 
\bv{7}\ledleftnote{亞伯拉罕壽終}亞伯拉罕一生的年日\chientien 是一百七十五歲\chuan 
\bv{8}\ProperNameC{0.5}{0}{亞伯拉罕}壽高年邁\chientien 氣絕而死\chientien 歸到他列祖\chu{原文作本民}那裡\yuentien 
\bv{9}他兩個兒子\ProperNameC{0}{0.5}{以撒}\chientien\ProperNameC{0}{0.5}{以\allowbreak 實瑪利}\chientien 把他埋葬在\PlaceName{麥比拉}洞裡\yuentien 這洞在\PlaceName{幔利}前\chientien\PlaceName{赫}人\ProperName{瑣轄}的兒子\ProperName{以弗崙}的田中\chientien 
\bv{10}就是\ProperName{亞伯拉罕}向\PlaceName{赫}人買的那塊田\yuentien\ProperName{亞伯拉罕}和他妻子\ProperNameC{0}{0.5}{撒拉}\chientien 都葬在那裡\chuan 
\bv{11}\ProperNameC{1}{0}{亞伯拉罕}死了以後\chientien  神賜福給他的兒子\ProperNameC{0}{0.5}{以撒}\yuentien\ProperName{以撒}靠近\PlaceName{庇耳拉海萊}居住\chuan\Chuan
\bv{12}\ledleftnote{以實瑪利之後裔}\ProperNameC{1}{0}{撒拉}的使女\PlaceName{埃及}人\ProperNameC{0}{0.5}{夏甲}\chientien 給\ProperName{亞伯拉罕}所生的兒子\chientien 是\ProperNameC{0}{0.5}{以實瑪利}\chuan 
\bv{13}\ProperNameC{1}{0}{以實瑪\allowbreak 利}兒子們的名字\chientien 按著他們的家譜\chientien 記在下面\yuentien\ProperName{以實瑪利}的長子是\ProperNameC{0}{0.5}{尼拜約}\chientien 又有\ProperNameC{0}{0.5}{基達}\chientien\ProperNameC{0}{0.5}{亞德別}\chientien\ProperNameC{0}{0.5}{米比衫}\chientien 
\bv{14}米\ProperNameC{0}{0.5}{施瑪}\chientien\ProperNameC{0}{0.5}{度瑪}\chientien\ProperNameC{0}{0.5}{瑪撒}\chientien 
\bv{15}\ProperNameC{1}{0.5}{哈大}\chientien\ProperNameC{0}{0.5}{提瑪}\chientien\ProperNameC{0}{0.5}{伊突}\chientien\ProperNameC{0}{0.5}{拿非施}\chientien\ProperNameC{0}{0.5}{基底瑪}\yuentien 
\bv{16}這是\ProperName{以實瑪利}眾子的名字\chientien 照著他們的村莊\chientien 營寨\chientien 作了十二族的族長\chuan 
\bv{17}\ProperNameC{1}{0}{以實瑪利}享壽一百三十七歲\chientien 氣絕而死\chientien 歸到他列祖\chu{原文作本民}那裡\chuan 
\bv{18}他子孫的住處在他眾弟兄東邊\chientien 從\PlaceName{哈腓拉}直到\PlaceName{埃及}前的\PlaceNameC{0}{0.5}{書珥}\chientien 正在\PlaceName{亞述}的道上\chuan\Chuan
\bv{19}\ProperNameC{1}{0}{亞伯拉罕}的兒子\ProperName{以撒}的後代\chientien 記在下面\yuentien\ProperName{亞伯拉罕}生\ProperNameC{0}{0.5}{以撒}\chientien 
\bv{20}\ProperNameC{1}{0}{以撒}娶\ProperName{利百加}為妻的時候\chientien 正四十歲\yuentien\ProperName{利百加}是\PlaceName{巴旦亞蘭}地的\PlaceName{亞蘭}人\chientien\ProperName{彼土利}的女兒\chientien 是\PlaceName{亞蘭}人\ProperName{拉班}的妹子\chuan 
\bv{21}\ProperNameC{1}{0}{以撒}因他妻子不生育\chientien 就為他祈求耶和華\chientien 耶和華應允他的祈求\chientien 他的妻子\ProperName{利百加}就懷了孕\chuan 
\bv{22}孩子們在他腹中彼此相爭\chientien 他就說\chientien 若是這樣\chientien 我為甚麼活著呢\yuentien\chu{或作我為甚麼如此呢}他就去求問耶和華\chuan 
\bv{23}耶和華對他說\chientien 兩國在你腹內\chientien 兩族要從你身上出來\chientien 這族必強於那族\chientien 將來大的要服事小的\chuan 
\bv{24}\ledleftnote{以掃雅各生}生產的日子到了\chientien 腹中果然是雙子\chuan 
\bv{25}先產的身體發紅\chientien 渾身有毛\chientien 如同皮衣\yuentien 他們就給他起名叫\ProperNameC{0}{0.5}{以掃}\chuan\chu{以掃就是有毛的意思}
\bv{26}隨後又生了\ProperName{以掃}的兄弟\chientien 手抓住以掃的腳跟\chientien 因此給他起名叫\ProperNameC{0}{0.5}{雅各}\chuan\chu{雅各就是抓住的意思}\ProperName{利\allowbreak 百加}生下兩個兒子的時候\chientien\ProperName{以撒}年正六十歲\chuan\Chuan
\bv{27}兩個孩子漸漸長大\chientien\ProperName{以掃}善於打獵\chientien 常在田野\yuentien\ProperName{雅各}為人安靜\chientien 常住在帳棚裡\chuan 
\bv{28}\ProperNameC{1}{0}{以撒}愛\ProperNameC{0}{0.5}{以掃}\chientien 因為常喫他的野味\yuentien\ProperName{利百加}卻愛\ProperNameC{0}{0.5}{雅各}\chuan 
\bv{29}有一天\chientien\ProperName{雅各}熬湯\chientien\ProperName{以掃}從田野回來累昏了\chuan 
\bv{30}\ProperNameC{1}{0}{以掃}對\ProperName{雅各}說\chientien 我累昏了\chientien 求你把這紅湯給我喝\yuentien 因此\ProperName{以掃}又叫\ProperNameC{0}{0.5}{以東}\chuan\chu{以東就是紅的意思}
\bv{31}\ProperNameC{1}{0}{雅各}說\chientien 你今日把長子的名分賣給我罷\chuan 
\bv{32}\ProperNameC{1}{0}{以掃}說\chientien 我將要死\chientien 這長子的名分於我有甚麼益處呢\chuan 
\bv{33}\ProperNameC{1}{0}{雅各}說\chientien 你今日對我起誓罷\yuentien\ProperName{以掃}就對他起了誓\chientien 把長子的名分賣給\ProperNameC{0}{0.5}{雅各}\chuan 
\bv{34}\ledleftnote{以掃賣長子之分}於是\ProperName{雅各}將餅和紅豆湯給了\ProperNameC{0}{0.5}{以掃}\chientien\ProperName{以掃}喫了喝了\chientien 便起來走了\yuentien 這就是\ProperName{以掃}輕看了他長子的名分\chuan 

\pend
\endnumbering
\beginnumbering
\pstart
\bchapter%
\bv{1}在\ProperName{亞伯拉罕}的日子\chientien 那地有一次饑荒\yuentien 這時又有饑荒\chientien 以撒就往\PlaceName{基拉耳}去\chientien 到\PlaceName{非利士}人的王\ProperName{亞比米勒}那裡\chuan 
\bv{2}耶和華向\ProperName{以撒}顯現\chientien 說\chientien 你不要下\PlaceName{埃及}去\chientien 要住在我所指示你的地\chuan 
\bv{3}你寄居在這地\chientien 我必與你同在\chientien 賜福給你\chientien 因為我要將這些地都賜給你和你的後裔\chientien 我必堅定我向你父\ProperName{亞伯拉罕}所起的誓\chuan 
\bv{4}我要加增你的後裔\chientien 像天上的星那樣多\yuentien 又要將這些地都賜給你的後裔\yuentien 並且地上萬國必因你的後裔得福\yuentien 
\bv{5}都因\ProperName{亞伯拉罕}聽從我的話\chientien 遵守我的吩咐\chientien 和我的命令\chientien 律例\chientien 法度\chuan 
\bv{6}\ledleftnote{以撒居基拉耳}\ProperNameC{0.5}{0}{以撒}就住在\PlaceName{基拉}\PlaceNameC{0}{0.5}{耳}\chuan 
\bv{7}那地方的人問到他的妻子\chientien 他便說\chientien 那是我的妹子\yuentien 原來他怕說\chientien 是我的妻子\yuentien\kenten{他心裡想}\chientien 恐怕這地方的人\chientien 為\ProperName{利百加}的緣故殺我\chientien 因為他容貌俊美\chuan 
\bv{8}他在那裡住了許久\yuentien 有一天\PlaceName{非利士}人的王\ProperNameC{0}{0.5}{亞比米勒}\chientien 從窗戶裡往外觀看\chientien 見\ProperName{以撒}和他的妻子\ProperName{利百加}戲玩\chuan 
\bv{9}\ledleftnote{受亞比米勒之責}\ProperNameC{0}{0.5}{亞比米勒}召了\ProperName{以撒}來\chientien 對他說\chientien 他實在是你的妻子\yuentien 你怎麼說\chientien 他是你的妹子\yuentien\ProperName{以撒}說\chientien 我心裡想\chientien 恐怕我因他而死\chuan 
\bv{10}\ProperNameC{0.5}{0}{亞比米勒}說\chientien 你向我們作的是甚麼事呢\chientien 民中險些有人和你的妻同寢\chientien 把我們陷在罪裡\chuan 
\bv{11}於是\ProperName{亞比米勒}曉諭眾民說\chientien 凡沾著這個人\chientien 或是他妻子的\chientien 定要把他治死\chuan\Chuan
\bv{12}\ProperNameC{1}{0}{以撒}在那地耕種\chientien 那一年有百倍的收成\chuan 耶和華賜福給他\chuan 
\bv{13}他就昌大\chientien 日增月盛\chientien 成了大富戶\chuan 
\bv{14}他有羊群\chientien 牛群\chientien 又有許多僕人\yuentien\PlaceName{非利士}人就嫉妒他\chuan 
\bv{15}當他父親\ProperName{亞伯拉罕}在世的日子\chientien 他父親的僕人所挖的井\chientien\PlaceName{非利士}人全都塞住\chientien 填滿了土\chuan 
\bv{16}\ProperNameC{1}{0}{亞比米勒}對以撒說\chientien 你離開我們去罷\chientien 因為你比我們強盛得多\chuan 
\bv{17}\ProperNameC{1}{0}{以撒}就離開那裡\chientien 在\PlaceName{基拉耳}谷支搭帳棚\chientien 住在那裡\chuan\Chuan
\bv{18}當他父親\ProperName{亞伯拉罕}在世之日所挖的水井\chientien 因\PlaceName{非利士}人在\ProperName{亞伯拉罕}死後塞住了\chientien\ProperName{以撒}就重新挖出來\chientien 仍照他父親所叫的\chientien 叫那些井的名字\chuan 
\bv{19}\ProperNameC{1}{0}{以撒}的僕人在谷中挖\kenten{井}\chientien 便得了一口活水井\chuan 
\bv{20}\PlaceNameC{1.1}{0}{基拉耳}的牧人與\ProperName{以撒}的牧人爭競\chientien 說\chientien 這水是我們的\yuentien\linebreak\ProperName{以撒}就給那井起名叫\PlaceNameC{0}{0.5}{埃色}\chientien 因為他們和他相爭\chuan\chu{埃色就是相爭的意思}
\bv{21}\ProperNameC{1}{0}{以撒}的僕人又挖了一口井\chientien 他們又為這井爭競\chientien 因此\ProperName{以撒}給這井起名叫\PlaceNameC{0}{0.5}{西提拿}\chuan\chu{西提拿就是為敵的意思}
\bv{22}\ProperNameC{1}{0}{以撒}離開那裡\chientien 又挖了一口井\chientien 他們不為這井爭競了\chientien 他就給那井起名叫\PlaceNameC{0}{0.5}{利河伯}\yuentien\chu{就是寬闊的意思}他說\chientien 耶和華現在給我們寬闊之地\chientien 我們必在這地昌盛\chuan\Chuan
\bv{23}\ProperNameC{1}{0}{以撒}從那裡上\PlaceName{別是巴}去\chuan 
\bv{24}當夜耶和華向他顯現\chientien 說\chientien 我是你父親\ProperName{亞伯拉罕}的 神\chientien 不要懼怕\chientien 因為我與你同在\chientien 要賜福給你\chientien 並要為我僕人\ProperName{亞伯拉罕}的緣故\chientien 使你的後裔繁多\chuan 
\bv{25}\ProperNameC{1}{0}{以撒}就在那裡築了一座壇\chientien 求告耶和華的名\chientien 並且支搭帳棚\yuentien 他的僕人便在那裡挖了一口井\chuan\Chuan
\bv{26}\ledleftnote{以撒與亞比米勒結盟}\ProperNameC{1}{0.5}{亞比米勒}\chientien 同他的朋友\ProperNameC{0}{0.5}{亞戶撒}\chientien 和他的軍長\ProperNameC{0}{0.5}{非各}\chientien 從\PlaceName{基拉耳}來見\ProperNameC{0}{0.5}{以撒}\chuan 
\bv{27}\ProperNameC{1}{0}{以撒}對他們說\chientien 你們既然恨我\chientien 打發我走了\chientien 為甚麼到我這裡來呢\chuan 
\bv{28}他們說\chientien 我們明明的看見耶和華與你同在\chientien 便說\chientien 不如我們兩下彼此起誓\chientien 彼此立約\chuan 
\bv{29}使你不害我們\chientien 正如我們未曾害你\chientien 一味的厚待你\chientien 並且打發你平平安安的走\yuentien 你是蒙耶和華賜福的了\chuan 
\bv{30}\ProperNameC{1}{0}{以撒}就為他們設擺筵席\chientien 他們便喫了喝了\chuan 
\bv{31}他們清早起來彼此起誓\yuentien\ProperName{以撒}打發他們走\chientien 他們就平平安安的離開他走了\chuan 
\bv{32}那一天\ProperName{以撒}的僕人來\chientien 將挖井的事告訴他\chientien 說\chientien 我們得了水了\chuan 
\bv{33}他就給那井起名叫\PlaceNameC{0}{0.5}{示巴}\yuentien 因此那城叫作\PlaceNameC{0}{0.5}{別是巴}\chientien 直到今日\chuan\Chuan
\bv{34}\ledleftnote{以掃娶二妻}\ProperNameC{1}{0}{以掃}四十歲的時候\chientien 娶了\PlaceName{赫}人\ProperName{比利}的女兒\ProperNameC{0}{0.5}{猶滴}\chientien 與\PlaceName{赫}人\ProperName{以倫}的女兒\ProperName{巴實抹}為妻\yuentien 
\bv{35}他們常使\ProperName{以撒}和\ProperName{利百加}心裡愁煩\chuan 

\pend
\endnumbering
\beginnumbering
\pstart
\bchapter%
\bv{1}\ProperNameC{0.5}{0}{以撒}年老\chientien 眼睛昏花\chientien 不能看見\chientien 就叫了他大兒子\ProperName{以掃}來\chientien 說\chientien 我兒\chientien\ProperName{以掃}說\chientien 我在這裡\chuan 
\bv{2}他說\chientien 我如今老了\chientien 不知道那一天死\chuan 
\bv{3}現在拿你的器械\chientien 就是箭囊\chientien 和弓\chientien 往田野去為我打獵\yuentien 
\bv{4}照我所愛的作成美味\chientien 拿來給我喫\chientien 使我在未死之先\chientien 給你祝福\chuan\Chuan
\bv{5}\ProperNameC{0.5}{0}{以撒}對他兒子\ProperName{以掃}說話\chientien\ProperName{利百加}也聽見了\chuan 以掃往田野去打獵\chientien 要得野味帶來\chuan 
\bv{6}\ProperNameC{0.5}{0}{利百加}就對他兒子\ProperName{雅各}說\chientien 我聽見你父親對你哥哥\ProperName{以掃}說\chientien 
\bv{7}你去把野獸帶來\chientien 作成美味給我喫\chientien 我好在未死之先\chientien 在耶和華面前給你祝福\chuan
\bv{8}現在我兒\chientien 你要照著我所吩咐你的\chientien 聽從我的話\chuan 
\bv{9}你到羊群裡去\chientien 給我拿兩隻肥山羊羔來\chientien 我便照你父親所愛的\chientien 給他作成美味\chuan 
\bv{10}你拿到你父親那裡給他喫\chientien 使他在未死之先\chientien 給你祝福\chuan 
\bv{11}\ProperNameC{1}{0}{雅各}對他母親\ProperName{利百加}說\chientien 我哥哥\ProperName{以掃}渾身是有毛的\chientien 我身上是光滑的\yuentien 
\bv{12}倘若我父親摸著我\chientien 必以我為欺哄人的\chientien 我就招咒詛\chientien 不得祝福\chuan 
\bv{13}他母親對他說\chientien 我兒\chientien 你招的咒詛歸到我身上\chientien 你只管聽我的話\chientien 去把羊羔給我拿來\chuan 
\bv{14}他便去拿來\chientien 交給他母親\chientien 他母親就照他父親所愛的\chientien 作成美味\chuan 
\bv{15}\ProperNameC{1}{0}{利百加}又把家裡所存大兒子\ProperName{以掃}上好的衣服\chientien 給他小兒子\ProperName{雅各}穿上\chuan 
\bv{16}又用山羊羔皮\chientien 包在\ProperName{雅各}的手上\chientien 和頸項的光滑處\yuentien 
\bv{17}就把所作的美味和餅\chientien 交在他兒子\ProperName{雅各}的手裡\chuan\Chuan
\bv{18}\ledleftnote{雅各欺父}\ProperNameC{1}{0}{雅各}到他父親那裡說\chientien 我父親\chientien 他說\chientien 我在這裡\chientien 我兒\chientien 你是誰\chuan 
\bv{19}\ProperNameC{1}{0}{雅各}對他父親說\chientien 我是你的長子\ProperNameC{0}{0.5}{以掃}\chientien 我已照你所吩咐我的行了\yuentien 請起來坐著\chientien 喫我的野味\chientien 好給我祝福\chuan 
\bv{20}\ProperNameC{1}{0}{以撒}對他兒子說\chientien 我兒\chientien 你如何找得這麼快呢\yuentien 他說\chientien 因為耶和華你的 神使我遇見好機會得著的\chuan 
\bv{21}\ProperNameC{1}{0}{以撒}對\ProperName{雅各}說\chientien 我兒\chientien 你近前來\chientien 我摸摸你\chientien 知道你真是我的兒子以掃不是\chuan 
\bv{22}\ProperNameC{1}{0}{雅各}就挨近他父親\ProperNameC{0}{0.5}{以撒}\yuentien 以撒摸著他說\chientien 聲音是\ProperName{雅各}的聲音\chientien 手卻是\ProperName{以掃}的手\chuan 
\bv{23}\ProperNameC{1}{0}{以撒}就辨不出他來\chientien 因為他手上有毛\chientien 像他哥哥\ProperName{以掃}的手一樣\chientien 就給他祝福\chuan 
\bv{24}又說\chientien 你真是我兒子\ProperName{以掃}麼\chientien 他說\chientien 我是\chuan 
\bv{25}\ProperNameC{1}{0}{以撒}說\chientien 你遞給我\chientien 我好喫我兒子的野味\chientien 給你祝福\chuan\ProperName{雅各}就遞給他\chientien 他便喫了\yuentien 又拿酒給他\chientien 他也喝了\chuan 
\bv{26}他父親\ProperName{以撒}對他說\chientien 我兒\chientien 你上前來與我親嘴\chuan 
\bv{27}他就上前與父親親嘴\yuentien 他父親一聞他衣服上的香氣\chientien 就給他祝福\chientien 說\chientien 我兒的香氣如同耶和華賜福之田地的香氣一樣\chuan 
\bv{28}願 神賜你天上的甘露\chientien 地上的肥土\chientien 並許多五穀新酒\yuentien 
\bv{29}願多民事奉你\chientien 多國跪拜你\yuentien 願你作你弟兄的主\chientien 你母親的兒子向你跪拜\yuentien 凡咒詛你的\chientien 願他受咒詛\yuentien 為你祝福的\chientien 願他蒙福\chuan\Chuan
\bv{30}\ProperNameC{1}{0}{以撒}為\ProperName{雅各}祝福已畢\chientien 雅各從他父親那裡纔出來\chientien 他哥哥\ProperName{以掃}正打獵回來\chuan 
\bv{31}也作了美味\chientien 拿來給他父親\chientien 說\chientien 請父親起來\chientien 喫你兒子的野味\chientien 好給我祝福\chuan 
\bv{32}他父親\ProperName{以撒}對他說\chientien 你是誰\chientien 他說\chientien 我是你的長子\ProperNameC{0}{0.5}{以掃}\chuan 
\bv{33}\ProperNameC{1}{0}{以撒}就大大的戰兢\chientien 說\chientien 你未來之先\chientien 是誰得了野味拿來給我呢\chientien 我已經喫了\chientien 為他祝福\chientien 他將來也必蒙福\chuan 
\bv{34}\ledleftnote{以掃痛哭求父祝福}\ProperNameC{1}{0}{以掃}聽了他父親的話\chientien 就放聲痛哭\chientien 說\chientien 我父阿\chientien 求你也為我祝福\chuan 
\bv{35}\ProperNameC{1}{0}{以撒}說\chientien 你兄弟已經用詭計來將你的福分奪去了\chuan 
\bv{36}\ProperNameC{1}{0}{以掃}說\chientien 他名\ProperName{雅各}豈不是正對麼\yuentien 因為他欺騙了我兩次\chientien 他從前奪了我長子的名分\chientien 你看\chientien 他現在又奪了我的福分\chuan\ProperName{以掃}又說\chientien 你沒有留下為我可祝的福麼\chuan 
\bv{37}\ProperNameC{1}{0}{以撒}回答\ProperName{以掃}說\chientien 我已立他為你的主\chientien 使他的弟兄都給他作僕人\chientien 並賜他五穀新酒可以養生\chientien 我兒\chientien 現在我還能為你作甚麼呢\chuan 
\bv{38}\ProperNameC{1}{0}{以掃}對他父親說\chientien 父阿\chientien 你只有一樣可祝的福麼\chientien 我父阿\chientien 求你也為我祝福\yuentien\ProperName{以掃}就放聲而哭\chuan 
\bv{39}他父親\ProperName{以撒}說\chientien 地上的肥土必為你所住\chientien 天上的甘露必為你所得\yuentien 
\bv{40}你必倚靠刀劍度日\chientien 又必事奉你的兄弟\chientien 到你強盛的時候\chientien 必從你頸項上掙開他的軛\chuan\Chuan
\bv{41}\ProperNameC{1}{0}{以掃}因他父親給\ProperName{雅各}祝的福\chientien 就怨恨\ProperNameC{0}{0.5}{雅各}\chientien 心裡說\chientien 為我父親居喪的日子近了\chientien 到那時候\chientien 我要殺我的兄弟\ProperNameC{0}{0.5}{雅各}\chuan\ledleftnote{以掃怨恨雅各欲殺之}
\bv{42}有人把\ProperName{利百加}大兒子\ProperName{以掃}的話告訴\ProperName{利百加}\chientien 他就打發人去\chientien 叫了他小兒子\ProperName{雅各}來\chientien 對他說\chientien 你哥哥\ProperName{以掃}\kenten{想要}殺你\chientien 報仇雪恨\chuan 
\bv{43}現在我兒\chientien 你要聽我的話\chientien 起來逃往\PlaceName{哈蘭}我哥哥\ProperName{拉班}那裡去\yuentien 
\bv{44}同他住些日子\chientien 直等你哥哥的怒氣消了\yuentien 
\bv{45}你哥哥向你消了怒氣\chientien 忘了你向他所作的事\chientien 我便打發人去把你從那裡帶回來\yuentien 為甚麼一日喪你們二人呢\chuan\Chuan 
\bv{46}\ProperNameC{1}{0}{利百加}對\ProperName{以撒}說\chientien 我因這\PlaceName{赫}人的女子\chientien 連性命都厭煩了\yuentien 倘若\ProperName{雅各}也娶\PlaceName{赫}人的女子為妻\chientien 像這些一樣\chientien 我活著還有甚麼益處呢\chuan 

\pend
\endnumbering
\beginnumbering
\pstart
\bchapter%
\bv{1}\ledleftnote{以撒遣雅各娶妻於舅家}\ProperNameC{0.5}{0}{以撒}叫了\ProperName{雅各}來\chientien 給他祝福\yuentien 並囑咐他說\chientien 你不要娶\PlaceName{迦南}的女子為妻\chuan 
\bv{2}你起身往\PlaceName{巴旦}\PlaceName{亞蘭}去\chientien 到你外祖\ProperName{彼土利}家裡\chientien 在你母舅\ProperName{拉班}的女兒中\chientien 娶一女為妻\chuan 
\bv{3}願全能的 神賜福給你\chientien 使你生養眾多\chientien 成為多族\chientien 
\bv{4}將應許\ProperName{亞伯拉罕}的福賜給你\chientien 和你的後裔\chientien 使你承受你所寄居的地為業\chientien 就是 神賜給\ProperName{亞伯拉罕}的地\chuan 
\bv{5}\ProperNameC{0.5}{0}{以撒}打發\ProperName{雅各}走了\chientien 他就往\PlaceName{巴旦亞蘭}去\chientien 到\PlaceName{亞蘭}人\ProperName{彼土利}的兒子\ProperName{拉班}那裡\yuentien\ProperName{拉班}是\ProperName{雅各}\ProperName{以掃}的母舅\chuan 
\bv{6}\ProperNameC{0.5}{0}{以掃}見\ProperName{以撒}已經給\ProperName{雅各}祝福\chientien 而且打發他往\PlaceName{巴旦亞蘭}去\chientien 在那裡娶妻\chientien 並見祝福的時候囑咐他說\chientien 不要娶\PlaceName{迦南}的女子為妻\yuentien 
\bv{7}又見\ProperName{雅各}聽從他父母的話\chientien 往\PlaceName{巴旦亞蘭}去了\yuentien 
\bv{8}\ProperNameC{0.5}{0}{以掃}就曉得他父親\ProperName{以撒}看不中\PlaceName{迦南}的女子\chientien 
\bv{9}便往\ProperName{以實瑪利}那裡去\chientien 在他二妻之外\chientien 又娶了\ProperName{瑪哈拉}為妻\yuentien 他是\ProperName{亞伯\allowbreak 拉罕}兒子\ProperName{以實瑪利}的女兒\chientien\ProperName{尼拜約}的妹子\chuan\Chuan
\bv{10}\ledleftnote{雅各夢中得指示}\ProperNameC{0.5}{0}{雅各}出了\PlaceName{別是巴}向\PlaceName{哈蘭}走去\chuan 
\bv{11}到了一個地方\chientien 因為太陽落了\chientien 就在那裡住宿\yuentien 便拾起那地方的一塊石頭\chientien 枕在頭下\chientien 在那裡躺臥睡了\chuan 
\bv{12}夢見一個梯子立在地上\chientien 梯子的頭頂著天\chientien 有 神的使者在梯子上\chientien 上去下來\chuan 
\bv{13}耶和華站在梯子以上\chientien\chu{或作站在他旁邊}說\chientien 我是耶和華你祖\ProperName{亞伯拉罕}的 神\chientien 也是\ProperName{以撒}的 神\chientien 我要將你現在所躺臥之地賜給你\chientien 和你的後裔\yuentien 
\bv{14}你的後裔必像地上的塵沙那樣多\chientien 必向東西南北開展\yuentien 地上萬族必因你和你的後裔得福\yuentien 
\bv{15}我也與你同在\chientien 你無論往那裡去\chientien 我必保佑你\chientien 領你歸回這地\chientien 總不離棄你\chientien 直到我成全了向你所應許的\chuan 
\bv{16}\ProperNameC{1}{0}{雅各}睡醒了\chientien 說\chientien 耶和華真在這裡\chientien 我竟不知道\chuan 
\bv{17}就懼怕說\chientien 這地方何等可畏\chientien 這不是別的\chientien 乃是 神的殿\chientien 也是天的門\chuan\Chuan
\bv{18}\ProperNameC{1}{0}{雅各}清早起來\chientien 把所枕的石頭立作柱子\chientien 澆油在上面\chuan 
\bv{19}他就給那地方起名\chientien 叫\PlaceNameC{0}{0.5}{伯特利}\yuentien\chu{就是 神殿的意思}但那地方起先名叫\PlaceNameC{0}{0.5}{路斯}\chuan 
\bv{20}\ledleftnote{雅各許願}\ProperNameC{1}{0}{雅各}許願\chientien 說\chientien  神若與我同在\chientien 在我所行的路上保佑我\chientien 又給我食物喫\chientien 衣服穿\chientien 
\bv{21}使我平平安安的回到我父親的家\chientien 我就必以耶和華為我的 神\chientien 
\bv{22}我所立為柱子的石頭\chientien 也必作 神的殿\yuentien 凡你所賜給我的\chientien 我必將十分之一獻給你\chuan 

\pend
\endnumbering
\beginnumbering
\pstart
\bchapter%
\bv{1}\ProperNameC{0.5}{0}{雅各}起行\chientien 到了東方人之地\yuentien 
\bv{2}看見田間有一口井\chientien 有三群羊臥在井旁\yuentien 因為人飲羊群\chientien 都是用那井裡的水\yuentien 井口上的石頭是大的\chuan 
\bv{3}常有羊群在那裡聚集\yuentien 牧人把石頭轉離井口飲羊\chientien 隨後又把石頭放在井口的原處\chuan 
\bv{4}\ProperNameC{0.5}{0}{雅各}對牧人說\chientien 弟兄們\chientien 你們是那裡來的\yuentien 他們說\chientien 我們是\PlaceName{哈蘭}來的\chuan 
\bv{5}他問他們說\chientien\ProperName{拿鶴}的孫子\ProperNameC{0}{0.5}{拉班}\chientien 你們認識麼\yuentien 他們說\chientien 我們認識\chuan 
\bv{6}\ProperNameC{0.5}{0}{雅各}說\chientien 他平安麼\yuentien 他們說\chientien 平安\yuentien 看哪\chientien 他女兒\ProperName{拉結}領著羊來了\chuan 
\bv{7}\ProperNameC{0.5}{0}{雅各}說\chientien 日頭還高\chientien 不是羊群聚集的時候\chientien 你們不如飲羊再去放一放\chuan 
\bv{8}他們說\chientien 我們不能\chientien 必等羊群聚齊\chientien 人把石頭轉離井口\chientien 纔可飲羊\chuan 
\bv{9}\ledleftnote{雅各遇拉結}\ProperNameC{0.5}{0}{雅各}正和他們說話的時候\chientien\ProperName{拉結}領著他父親的羊來了\chientien 因為那些羊是他牧放的\chuan 
\bv{10}\ProperNameC{0.5}{0}{雅各}看見母舅\ProperName{拉班}的女兒\ProperNameC{0}{0.5}{拉結}\chientien 和母舅\ProperName{拉班}的羊群\chientien 就上前把石頭轉離井口\chientien 飲他母舅拉班的羊群\chuan 
\bv{11}\ProperNameC{1}{0}{雅各}與\ProperName{拉結}親嘴\chientien 就放聲而哭\chuan 
\bv{12}\ProperNameC{1}{0}{雅各}告訴\ProperName{拉結}自己是他父親的外甥\chientien 是\ProperName{利百加}的兒子\yuentien\ProperName{拉結}就跑去告訴他父親\chuan\Chuan
\bv{13}\ledleftnote{拉班迎接雅各}\ProperNameC{1}{0}{拉班}聽見外甥\ProperName{雅各}的信息\chientien 就跑去迎接\chientien 抱著他與他親嘴\chientien 領他到自己的家\yuentien\ProperName{雅各}將一切的情由告訴\ProperNameC{0}{0.5}{拉班}\chuan 
\bv{14}\ProperNameC{1}{0}{拉班}對他說\chientien 你實在是我的骨肉\yuentien\ProperName{雅各}就和他同住了一個月\chuan 
\bv{15}拉\ProperName{班}對\ProperName{雅各}說\chientien 你雖是我的骨肉\chientien\chu{原文作弟兄}豈可白白的服事我\chientien 請告訴我你要甚麼為工價\chuan 
\bv{16}\ProperNameC{1}{0}{拉班}有兩個女兒\chientien 大的名叫\ProperNameC{0}{0.5}{利亞}\chientien 小的名叫\ProperNameC{0}{0.5}{拉結}\chuan 
\bv{17}\ProperNameC{1}{0}{利亞}的眼睛沒有神氣\chientien\ProperName{拉結}卻生得美貌俊秀\chuan 
\bv{18}\ProperNameC{1}{0}{雅各}愛\ProperNameC{0}{0.5}{拉結}\chientien 就說\chientien 我願為你小女兒\ProperName{拉結}服事你七年\chuan 
\bv{19}\ProperNameC{1}{0}{拉班}說\chientien 我把他給你\chientien 勝似給別人\chientien 你與我同住罷\chuan 
\bv{20}\ProperNameC{1}{0}{雅各}就為\ProperName{拉結}服事了七年\yuentien 他因為深愛\ProperNameC{0}{0.5}{拉結}\chientien 就看這七年如同幾天\chuan\Chuan
\bv{21}\ProperNameC{1}{0}{雅各}對\ProperName{拉班}說\chientien 日期已經滿了\chientien 求你把我的妻子給我\chientien 我好與他同房\chuan 
\bv{22}\ProperNameC{1}{0}{拉班}就擺設筵席\chientien 請齊了那地方的眾人\chuan 
\bv{23}\ledleftnote{拉班以利亞妻雅各}到晚上\chientien\ProperName{拉班}將女兒\ProperName{利亞}送來給\ProperNameC{0}{0.5}{雅各}\chientien\ProperName{雅各}就與他同房\chuan 
\bv{24}\ProperNameC{1}{0}{拉班}又將婢女\ProperName{悉帕}給女兒\ProperName{利亞}作使女\chuan 
\bv{25}到了早晨\chientien 雅各一看是\ProperNameC{0}{0.5}{利亞}\chientien 就對\ProperName{拉班}說\chientien 你向我作的是甚麼事呢\yuentien 我服事你\chientien 不是為\ProperName{拉結}麼\yuentien 你為甚麼欺哄我呢\chuan 
\bv{26}\ProperNameC{1}{0}{拉班}說\chientien 大女兒還沒有給人\chientien 先把小女兒給人\chientien 在我們這地方沒有這規矩\chuan 
\bv{27}你為這個滿了七日\chientien 我就把那個也給你\chientien 你再為他服事我七年\chuan 
\bv{28}\ProperNameC{1}{0}{雅各}就如此行\yuentien 滿了\ProperName{利亞}的七日\chientien\ProperName{拉班}便將女兒\ProperName{拉結}給\ProperName{雅各}為妻\chuan\ledleftnote{雅各又娶拉結}
\bv{29}\ProperNameC{1}{0}{拉班}又將婢女\ProperName{辟拉}給女兒\ProperName{拉結}作使女\chuan 
\bv{30}\ledleftnote{利亞與兩妾生子}\ProperNameC{1}{0}{雅各}也與\ProperName{拉結}同房\chientien 並且愛\ProperName{拉結}勝似愛\ProperNameC{0}{0.5}{利亞}\yuentien 於是又服事了\ProperName{拉班}七年\chuan\Chuan
\bv{31}耶和華見\ProperName{利亞}失寵\chientien\chu{原文作被恨下}\linebreak\chu{同 }就使他生育\yuentien\ProperName{拉結}卻不生育\chuan 
\bv{32}\ProperNameC{1}{0}{利亞}懷孕生子\chientien 就給他起名叫\ProperNameC{0}{0.5}{流便}\chientien\chu{就是有兒子的意思}因而說\chientien 耶和華看見我的苦情\chientien 如今我的丈夫必愛我\chuan 
\bv{33}他又懷孕生子\chientien 就說\chientien 耶和華因為聽見我失寵\chientien 所以又賜給我這個兒子\yuentien 於是給他起名叫\ProperNameC{0}{0.5}{西緬}\chuan\chu{就是聽見的意思}
\bv{34}他又懷孕生子\chientien 起名叫\ProperNameC{0}{0.5}{利未}\chientien\chu{就是聯合的意思}說\chientien 我給丈夫生了三個兒子\chientien 他必與我聯合\chuan 
\bv{35}他又懷孕生子\chientien 說\chientien 這回我要讚美耶和華\chientien 因此給他起名叫\ProperNameC{0}{0.5}{猶大}\yuentien\chu{就是讚美的意思}這纔停了生育\chuan 

\pend
\endnumbering
\beginnumbering
\pstart
\bchapter%
\bv{1}\ProperNameC{0.5}{0}{拉結}見自己不給\ProperName{雅各}生子\chientien 就嫉妒他姐姐\chientien 對\ProperName{雅各}說\chientien 你給我孩子\chientien 不然我就死了\chuan 
\bv{2}\ProperNameC{0.5}{0}{雅各}向\ProperName{拉結}生氣\chientien 說\chientien 叫你不生育的是 神\chientien 我豈能代替他\kenten{作主}呢\chuan 
\bv{3}\ProperNameC{0.5}{0}{拉結}說\chientien 有我的使女\ProperName{辟拉}在這裡\chientien 你可以與他同房\chientien 使他生子在我膝下\chientien 我便因他也得孩子\chuan\chu{得孩子原文作被建立}
\bv{4}\ProperNameC{0.5}{0}{拉結}就把他的使女\ProperName{辟拉}給丈夫為妾\chientien\ProperName{雅各}便與他同房\chuan 
\bv{5}\ProperNameC{0.5}{0}{辟拉}就懷孕給\ProperName{雅各}生了一個兒子\chuan 
\bv{6}\ProperNameC{0.5}{0}{拉結}說\chientien  神伸了我的冤\chientien 也聽了我的聲音\chientien 賜我一個兒子\chientien 因此給他起名叫\ProperNameC{0}{0.5}{但}\chuan\chu{就是伸冤的意思}
\bv{7}\ProperNameC{0.5}{0}{拉結}的使女\ProperName{辟拉}又懷孕\chientien 給\ProperName{雅各}生了第二個兒子\chuan 
\bv{8}\ProperNameC{0.5}{0}{拉結}說\chientien 我與我姐姐大大相爭\chientien 並且得勝\yuentien 於是給他起名叫\ProperNameC{0}{0.5}{拿弗他利}\chuan\chu{就是相爭的意思}\Chuan
\bv{9}\ProperNameC{0.5}{0}{利亞}見自己停了生育\chientien 就把使女\ProperName{悉帕}給\ProperName{雅各}為妾\yuentien 
\bv{10}\ProperNameC{0.5}{0}{利亞}的使女\ProperName{悉帕}給\ProperName{雅各}生了一個兒子\yuentien 
\bv{11}\ProperNameC{1}{0}{利亞}說\chientien 萬幸\chientien 於是給他起名叫\ProperNameC{0}{0.5}{迦得}\chuan\chu{就是萬幸}\linebreak\chu{的意思}
\bv{12}\ProperNameC{1}{0}{利亞}的使女\ProperName{悉帕}又給\ProperName{雅各}生了第二個兒子\yuentien 
\bv{13}\ProperNameC{1}{0}{利亞}說\chientien 我有福阿\chientien 眾女子都要稱我是有福的\yuentien 於是給他起名叫\ProperNameC{0}{0.5}{亞設}\chuan\chu{就是有福的意思}\Chuan
\bv{14}割麥子的時候\chientien\ProperName{流便}往田裡去尋見風茄\chientien 拿來給他母親\ProperNameC{0}{0.5}{利亞}\yuentien\ProperName{拉結}對\ProperName{利\allowbreak 亞}說\chientien 請你把你兒子的風茄給我些\chuan 
\bv{15}\ProperNameC{1}{0}{利亞}說\chientien 你奪了我的丈夫還算小事麼\chientien 你又要奪我兒子的風茄麼\chuan\ProperName{拉結}說\chientien 為你兒子的風茄\chientien 今夜他可以與你同寢\chuan 
\bv{16}到了晚上\chientien\ProperName{雅各}從田裡回來\chientien\ProperName{利亞}出來迎接他\chientien 說\chientien 你要與我同寢\chientien 因為我實在用我兒子的風茄\chientien 把你雇下了\chuan 那一夜\ProperName{雅各}就與他同寢\yuentien 
\bv{17} 神應允了\ProperNameC{0}{0.5}{利亞}\chientien 他就懷孕\chientien 給\ProperName{雅各}生了第五個兒子\chuan 
\bv{18}\ProperNameC{1}{0}{利亞}說\chientien  神給了我價值\chientien 因為我把使女給了我丈夫\yuentien 於是給他起名叫以\ProperNameC{0}{0.5}{薩迦}\chuan\chu{就是價值的意思}
\bv{19}\ProperNameC{1}{0}{利亞}又懷孕\chientien 給\ProperName{雅各}生了第六個兒子\chuan 
\bv{20}\ProperNameC{1}{0}{利亞}說\chientien  神賜我厚賞\chientien 我丈夫必與我同住\chientien 因我給他生了六個兒子\yuentien 於是給他起名\ProperNameC{0}{0.5}{西布倫}\chuan\chu{就是同住的意思}
\bv{21}後來又生了一個女兒\chientien 給他起名叫\ProperNameC{0}{0.5}{底拿}\chuan 
\bv{22} 神顧念\ProperNameC{0}{0.5}{拉結}\chientien 應允了他\chientien 使他能生育\chuan 
\bv{23}\ProperNameC{1}{0}{拉結}懷孕生子\chientien 說\chientien  神除去了我的羞恥\yuentien 
\bv{24}就給他起名叫\ProperNameC{0}{0.5}{約瑟}\chientien\chu{就是增添}\linebreak\chu{的意思}意思說\chientien 願耶和華再增添我一個兒子\chuan\Chuan
\bv{25}\ProperNameC{1}{0}{拉結}生\ProperName{約瑟}之後\chientien\ProperName{雅各}對\ProperName{拉班}說\chientien 請打發我走\chientien 叫我回到我本鄉本土去\chuan 
\bv{26}請你把我服事你所得的妻子\chientien 和兒女給我\chientien 讓我走\yuentien 我怎樣服事你\chientien 你都知道\chuan 
\bv{27}\ProperNameC{1}{0}{拉班}對他說\chientien 我若在你眼前蒙恩\chientien 請你\kenten{仍與我同住}\chientien\kenten{因為}我已算定\chientien 耶和華賜福與我\chientien 是為你的緣故\yuentien 
\bv{28}\ledleftnote{雅各與拉班定工價}又說\chientien 請你定你的工價\chientien 我就給你\chuan 
\bv{29}\ProperNameC{1}{0}{雅各}對他說\chientien 我怎樣服事你\chientien 你的牲畜在我手裡怎樣\chientien 是你知道的\chuan 
\bv{30}我未來之先\chientien 你所有的很少\chientien 現今卻發大眾多\chientien 耶和華隨我的腳步賜福與你\yuentien 如今\chientien 我甚麼時候纔為自己興家立業呢\chuan 
\bv{31}\ProperNameC{1}{0}{拉班}說\chientien 我當給你甚麼呢\chuan\ProperName{雅各}說\chientien 甚麼你也不必給我\chientien 只有一件事\chientien 你若應承\chientien 我便仍舊牧放你的羊群\chuan 
\bv{32}今天我要走遍你的羊群\chientien 把綿羊中凡有點的\chientien 有斑的\chientien 和黑色的\chientien 並山羊中凡有斑的\chientien 有點的\chientien 都挑出來\chientien 將來\kenten{這一等的}\chientien 就算我的工價\chuan 
\bv{33}以後你來查看我的工價\chientien 凡在我手裡的山羊不是有點有斑的\chientien 綿羊不是黑色的\chientien 那就算是我偷的\yuentien 這樣\chientien 便可證出我的公義來\chuan 
\bv{34}\ProperNameC{1}{0}{拉班}說\chientien 好阿\chuan 我情願照著你的話行\chuan 
\bv{35}當日\ProperName{拉班}把有紋的\chientien 有斑的公山羊\chientien 有點的\chientien 有斑的\chientien 有雜白紋的母山羊\chientien 並黑色的綿羊\chientien 都挑出來\chientien 交在他兒子們的手下\yuentien 
\bv{36}又使自己和\ProperName{雅各}相離三天的路程\yuentien\ProperName{雅各}就牧養\ProperName{拉班}其餘的羊\chuan\Chuan
\bv{37}\ledleftnote{雅各用策致富}\ProperNameC{1}{0}{雅各}拿楊樹\chientien 杏樹\chientien 楓樹的嫩枝\chientien 將皮剝成白紋\chientien 使枝子露出白的來\yuentien 
\bv{38}將剝了皮的枝子\chientien 對著羊群插在飲羊的水溝裡\chientien 和水槽裡\chientien 羊來喝的時候牝牡配合\chuan 
\bv{39}羊對著枝子配合\chientien 就生下有紋的\chientien 有點的\chientien 有斑的來\chuan 
\bv{40}\ProperNameC{1}{0}{雅各}把羊羔分出來\chientien 使\ProperName{拉班}的羊\chientien 與這有紋和黑色的羊相對\chientien 把自己的羊另放一處\chientien 不叫他和\ProperName{拉班}的羊混雜\chuan 
\bv{41}到羊群肥壯配合的時候\chientien\ProperName{雅各}就把枝子插在水溝裡\chientien 使羊對著枝子配合\chuan 
\bv{42}只是到羊瘦弱配合的時候\chientien 就不插枝子\yuentien 這樣\chientien 瘦弱的就歸\ProperNameC{0}{0.5}{拉班}\chientien 肥壯的就歸雅各\chuan 
\bv{43}於是\ProperName{雅各}極其發大\chientien 得了許多的羊群\chientien 僕婢\chientien 駱駝\chientien 和驢\chuan 

\pend
\endnumbering
\beginnumbering
\pstart
\bchapter%
\bv{1}\ledleftnote{雅各思歸故土}\ProperNameC{0.5}{0}{雅各}聽見\ProperName{拉班}的兒子們有話說\chientien\ProperName{雅各}把我們父親所有的都奪了去\chientien 並藉著我們父親的\chientien 得了這一切的榮耀\chuan\chu{榮耀或作財}
\bv{2}\ProperNameC{0.5}{0}{雅各}見\ProperName{拉班}的氣色向他不如從前了\chuan 
\bv{3}耶和華對\ProperName{雅各}說\chientien 你要回你祖你父之地\chientien 到你親族那裡去\chientien 我必與你同在\chuan 
\bv{4}\ProperNameC{1}{0}{雅各}就打發人\chientien 叫\ProperName{拉結}和\ProperName{利亞}到田野羊群那裡來\yuentien 
\bv{5}對他們說\chientien 我看你們父親的氣色向我不如從前了\yuentien 但我父親的 神向來與我同在\chuan 
\bv{6}你們也知道\chientien 我盡了我的力量服事你們的父親\chuan 
\bv{7}你們的父親欺哄我\chientien 十次改了我的工價\yuentien 然而 神不容他害我\chuan 
\bv{8}他若說\chientien 有點的歸你作工價\chientien 羊群所生的都有點\yuentien 他若說\chientien 有紋的歸你作工價\chientien 羊群所生的都有紋\chuan 
\bv{9}這樣\chientien  神把你們父親的牲畜奪來賜給我了\chuan 
\bv{10}羊配合的時候\chientien 我夢中舉目一看\chientien 見跳母羊的公羊\chientien 都是有紋的\chientien 有點的\chientien 有花斑的\chuan 
\bv{11} 神的使者在那夢中呼叫我說\chientien\ProperNameC{0}{0.5}{雅各}\chuan 我說\chientien 我在這裡\chuan 
\bv{12}他說\chientien 你舉目觀看\chientien 跳母羊的公羊都是有紋的\chientien 有點的\chientien 有花斑的\yuentien 凡\ProperName{拉班}向你所作的\chientien 我都看見了\chuan 
\bv{13}我是\PlaceName{伯特利}的 神\yuentien 你在那裡\kenten{用油}澆過柱子\chientien 向我許過願\yuentien 現今你起來離開這地\chientien 回你本地去罷\chuan 
\bv{14}\ProperNameC{1}{0}{拉結}和\ProperName{利亞}回答\ProperName{雅各}說\chientien 在我們父親的家裡還有我們可得的分麼\chientien 還有我們的產業麼\chuan 
\bv{15}我們不是被他當作外人麼\yuentien 因為他賣了我們\chientien 吞了我們的價值\chuan 
\bv{16} 神從我們父親所奪出來的一切財物\chientien 那就是我們\chientien 和我們孩子們的\yuentien 現今凡 神所吩咐你的\chientien 你只管去行罷\chuan\Chuan
\bv{17}\ProperNameC{1}{0}{雅各}起來\chientien 使他的兒子和妻子都騎上駱駝\yuentien 
\bv{18}又帶著他在\PlaceName{巴旦亞蘭}所得的一切牲畜和財物\chientien 往\PlaceName{迦南}地\chientien 他父親\ProperName{以撒}那裡去了\chuan 
\bv{19}當時\ProperName{拉班}剪羊毛去了\yuentien\ProperName{拉結}偷了他父親家中的神像\chuan 
\bv{20}\ledleftnote{雅各背逃}\ProperNameC{1}{0}{雅各}背著\PlaceName{亞蘭}人\ProperName{拉班}偷走了\chientien 並不告訴他\chuan 
\bv{21}就帶著所有的逃跑\yuentien 他起身過大河\chientien 面向\PlaceName{基列}山行去\chuan\Chuan
\bv{22}到第三日\chientien 有人告訴\ProperNameC{0}{0.5}{拉班}\chientien\ProperName{雅各}逃跑了\chuan 
\bv{23}\ledleftnote{拉班追之}\ProperNameC{1}{0}{拉班}帶領他的眾弟兄去追趕\chientien 追了七日\chientien 在\PlaceName{基列}山就追上了\chuan 
\bv{24}夜間 神到\PlaceName{亞蘭}人\ProperName{拉班}那裡\chientien 在夢中對他說\chientien 你要小心\chientien 不可與\ProperName{雅各}說好說歹\chuan 
\bv{25}\ProperNameC{1}{0}{拉班}追上\ProperNameC{0}{0.5}{雅各}\chientien\ProperName{雅各}在山上支搭帳棚\yuentien\ProperName{拉班}和他的眾弟兄\chientien 也在\PlaceName{基列}山上支搭帳棚\chuan 
\bv{26}\ProperNameC{1}{0}{拉班}對\ProperName{雅各}說\chientien 你作的是甚麼事呢\chientien 你背著我偷走了\chientien 又把我的女兒們帶了去\chientien 如同用刀劍擄去的一般\chuan 
\bv{27}你為甚麼暗暗的逃跑\chientien 偷著走\chientien 並不告訴我\chientien 叫我可以歡樂\chientien 唱歌\chientien 擊鼓\chientien 彈琴的送你回去\chuan 
\bv{28}又不容我與外孫和女兒親嘴\yuentien 你所行的真是愚昧\chuan 
\bv{29}我手中原有能力害你\chientien 只是你父親的 神昨夜對我說\chientien 你要小心\chientien 不可與\ProperName{雅各}說好說歹\chuan 
\bv{30}現在你雖然想你父家\chientien 不得不去\chientien 為甚麼又偷了我的神像呢\chuan\ledleftnote{拉班責雅各竊其神像}
\bv{31}\ProperNameC{1}{0}{雅各}回答\ProperName{拉班}說\chientien 恐怕你把你的女兒從我奪去\chientien 所以我逃跑\chuan 
\bv{32}至於你的神像\chientien 你在誰那裡搜出來\chientien 就不容誰存活\yuentien 當著我們的眾弟兄你認一認\chientien 在我這裡有甚麼東西是你的\chientien 就拿去\chuan 原來\ProperName{雅各}不知道\ProperName{拉結}偷了那些神像\chuan\Chuan
\bv{33}\ProperNameC{1}{0}{拉班}進了\ProperNameC{0}{0.5}{雅各}\chientien\ProperNameC{0}{0.5}{利亞}\chientien 並兩個使女的帳棚\chientien 都沒有搜出來\yuentien 就從\ProperName{利亞}的帳棚出來\chientien 進了\ProperName{拉結}的帳棚\chuan 
\bv{34}\ProperNameC{1}{0}{拉結}已經把神像藏在駱駝的馱簍裡\chientien 便坐在上頭\chientien\ProperName{拉班}摸遍了那帳棚\chientien 並沒有摸著\chuan 
\bv{35}\ProperNameC{1}{0}{拉結}對他父親說\chientien 現在我身上不便\chientien 不能在你面前起來\chientien 求我主不要生氣\yuentien 這樣\chientien\ProperName{拉\allowbreak 班}搜尋神像\chientien 竟沒有搜出來\chuan\Chuan
\bv{36}\ledleftnote{雅各斥責拉班}\ProperNameC{1}{0}{雅各}就發怒斥責\ProperName{拉班}說\chientien 我有甚麼過犯\chientien 有甚麼罪惡\chientien 你竟這樣火速的追我\chuan 
\bv{37}你摸遍了我一切的家具\chientien 你搜出甚麼來呢\chientien 可以放在你我弟兄面前\chientien 叫他們在你我中間辨別辨別\chuan 
\bv{38}我在你家這二十年\chientien 你的母綿羊\chientien 母山羊\chientien 沒有掉過胎\yuentien 你群中的公羊\chientien 我沒有喫過\yuentien 
\bv{39}被野獸撕裂的\chientien 我沒有帶來給你\chientien 是我自己賠上\chientien 無論是白日\chientien 是黑夜\chientien 被偷去的\chientien 你都向我索要\chuan 
\bv{40}我白日受盡乾熱\chientien 黑夜受盡寒霜\chientien 不得合眼睡著\chientien 我常是這樣\chuan 
\bv{41}我這二十年在你家裡\chientien 為你的兩個女兒服事你十四年\chientien 為你的羊群服事你六年\chientien 你又十次改了我的工價\chuan 
\bv{42}若不是我父親\ProperName{以撒}所敬畏的 神\chientien 就是\ProperName{亞伯拉罕}的 神與我同在\chientien 你如今必定打發我空手而去\yuentien  神看見我的苦情\chientien 和我的勞碌\chientien 就在昨夜責備你\chuan\Chuan
\bv{43}\ProperNameC{1}{0}{拉班}回答\ProperName{雅各}說\chientien 這女兒是我的女兒\chientien 這些孩子是我的孩子\chientien 這些羊群也是我的羊群\chientien 凡在你眼前的都是我的\yuentien 我的女兒\chientien 並他們所生的孩子\chientien 我今日能向他們作甚麼呢\chuan 
\bv{44}\ledleftnote{二人立約}來罷\chientien 你我二人可以立約\chientien 作你我中間的證據\chuan 
\bv{45}\ProperNameC{1}{0}{雅各}就拿一塊石頭立作柱子\chuan 
\bv{46}又對眾弟兄說\chientien 你們堆聚石頭\yuentien 他們就拿石頭來堆成一堆\chientien 大家便在旁邊喫喝\chuan 
\bv{47}\ProperNameC{1}{0}{拉班}稱那石堆為\PlaceNameC{0}{0.5}{伊迦爾撒哈杜他}\chientien\ProperName{雅各}卻稱那石堆為\PlaceNameC{0}{0.5}{迦累得}\chuan\chu{都是以石堆為證的意思}
\bv{48}\ProperNameC{1}{0}{拉班}說\chientien 今日這石堆作你我中間的證據\yuentien 因此這地方名叫\PlaceNameC{0}{0.5}{迦累得}\chientien 
\bv{49}又叫\PlaceNameC{0}{0.5}{米斯巴}\chientien 意思說\chientien 我們彼此離別以後\chientien 願耶和華在你我中間鑒察\chuan 
\bv{50}你若苦待我的女兒\chientien 又在我的女兒以外另娶妻\chientien 雖沒有人知道\chientien 卻有 神在你我中間作見證\chuan 
\bv{51}\ProperNameC{1}{0}{拉班}又說\chientien 你看我在你我中間所立的這石堆\chientien 和柱子\yuentien 
\bv{52}這石堆作證據\chientien 這柱子也作證據\chientien 我必不過這石堆去害你\chientien 你也不可過這石堆和柱子\chientien 來害我\yuentien 
\bv{53}但願\ProperName{亞伯拉罕}的 神\chientien 和\ProperName{拿鶴}的 神\chientien 就是他們父親的 神\chientien 在你我中間判斷\yuentien\ProperName{雅各}就指著他父親以撒所敬畏的 神起誓\chuan 
\bv{54}又在山上獻祭\chientien 請眾弟兄來喫飯\yuentien 他們喫了飯\chientien 便在山上住宿\chuan 
\bv{55}\ProperNameC{1}{0}{拉班}清早起來\chientien 與他外孫和女兒親嘴\chientien 給他們祝福\chientien 回往自己的地方去了\chuan 

\pend
\endnumbering
\beginnumbering
\pstart
\bchapter%
\bv{1}\ProperNameC{0.5}{0}{雅各}仍舊行路\chientien  神的使者遇見他\chuan 
\bv{2}\ProperNameC{0.5}{0}{雅各}看見他們就說\chientien 這是 神的軍兵\yuentien 於是給那地方起名叫\PlaceNameC{0}{0.5}{瑪哈念}\chuan\chu{就是二軍兵的意思}
\bv{3}\ProperNameC{0.5}{0}{雅各}打發人先往\PlaceName{西珥}地去\chientien 就是\PlaceName{以東}地\chientien 見他哥哥\ProperNameC{0}{0.5}{以掃}\chuan 
\bv{4}吩咐他們說\chientien 你們對我主\ProperName{以掃}說\chientien 你的僕人\ProperName{雅各}這樣說\chientien 我在\ProperName{拉班}那裡寄居\chientien 直到如今\chuan 
\bv{5}我有牛\chientien 驢\chientien 羊群\chientien 僕婢\chientien 現在打發人來報告我主\chientien 為要在你眼前蒙恩\chuan 
\bv{6}所打發的人回到\ProperName{雅各}那裡說\chientien 我們到了你哥哥\ProperName{以掃}那裡\chientien 他帶著四百人\chientien 正迎著你來\chuan 
\bv{7}\ledleftnote{雅各懼怕以掃}\ProperNameC{0.5}{0}{雅各}就甚懼怕\chientien 而且愁煩\chientien 便把那與他同在的人口\chientien 和羊群\chientien 牛群\chientien 駱駝\chientien 分作兩隊\yuentien 
\bv{8}說\chientien\ProperName{以掃}若來擊殺這一隊\chientien 剩下的那一隊還可以逃避\chuan 
\bv{9}\ProperNameC{0.5}{0}{雅各}說\chientien 耶和華我祖\ProperName{亞伯拉罕}的 神\chientien 我父親\ProperName{以撒}的 神阿\chientien 你曾對我說\chientien 回你本地本族去\chientien 我要厚待你\chuan 
\bv{10}你向僕人所施的一切慈愛和誠實\chientien 我一點也不配得\yuentien 我先前只拿著我的杖過這約但河\chientien 如今我卻成了兩隊了\chuan 
\bv{11}求你救我脫離我哥哥\ProperName{以掃}的手\chientien 因為我怕他來殺我\chientien 連妻子帶兒女一同殺了\chuan 
\bv{12}你曾說\chientien 我必定厚待你\chientien 使你的後裔如同海邊的沙\chientien 多得不可勝數\chuan 
\bv{13}\ledleftnote{送禮物給以掃}當夜\ProperName{雅各}在那裡住宿\chientien 就從他所有的物中拿禮物\chientien 要送給他哥哥\ProperNameC{0}{0.5}{以掃}\yuentien 
\bv{14}母山羊二百隻\chientien 公山羊二十隻\chientien 母綿羊二百隻\chientien 公綿羊二十隻\chientien 
\bv{15}奶崽子的駱駝三十隻\chientien 各帶著崽子\chientien 母牛四十隻\chientien 公牛十隻\chientien 母驢二十匹\chientien 驢駒十匹\yuentien 
\bv{16}每樣各分一群\chientien 交在僕人手下\chientien 就對僕人說\chientien 你們要在我前頭過去\chientien 使群群相離有空閒的地方\chuan 
\bv{17}又吩咐儘先走的說\chientien 我哥哥\ProperName{以掃}遇見你的時候\chientien 問你說\chientien 你是那家的人\chientien 要往那裡去\chientien 你前頭這些是誰的\yuentien 
\bv{18}你就說\chientien 是你僕人\ProperName{雅各}的\chientien 是送給我主\ProperName{以掃}的禮物\chientien 他自己也在我們後邊\chuan 
\bv{19}又吩咐第二\chientien 第三\chientien 和一切趕群畜的人說\chientien 你們遇見\ProperName{以掃}的時候\chientien 也要這樣對他說\yuentien 
\bv{20}並且你們要說\chientien 你僕人\ProperName{雅各}在我們後邊\yuentien 因\ProperName{雅各}心裡說\chientien 我藉著在我前頭去的禮物解他的恨\chientien 然後再見他的面\chientien 或者他容納我\chuan 
\bv{21}於是禮物先過去了\yuentien 那夜\ProperName{雅各}在隊中住宿\chuan\Chuan
\bv{22}\ledleftnote{雅各與天使摔跤而勝之}他夜間起來\chientien 帶著兩個妻子\chientien 兩個使女\chientien 並十一個兒子都過了\PlaceName{雅博}渡口\chuan 
\bv{23}先打發他們過河\chientien 又打發所有的都過去\yuentien 
\bv{24}只剩下\ProperName{雅各}一人\yuentien 有一個人來和他摔跤\chientien 直到黎明\chuan 
\bv{25}那人見自己勝不過他\chientien 就將他的大腿窩摸了一把\chientien\ProperName{雅各}的大腿窩\chientien 正在摔跤的時候就扭了\chuan 
\bv{26}那人說\chientien 天黎明了\chientien 容我去罷\yuentien\ProperName{雅各}說\chientien 你不給我祝福\chientien 我就不容你去\chuan 
\bv{27}那人說\chientien 你名叫甚麼\chientien 他說\chientien 我名叫\ProperNameC{0}{0.5}{雅各}\chuan 
\bv{28}\ledleftnote{ 神賜名以色列}那人說\chientien 你的名不要再叫\ProperNameC{0}{0.5}{雅各}\chientien 要叫\ProperNameC{0}{0.5}{以色列}\chientien 因為你與 神與人較力\chientien 都得了勝\chuan 
\bv{29}\ProperNameC{1}{0}{雅各}問他說\chientien 請將你的名告訴我\yuentien 那人說\chientien 何必問我的名\yuentien 於是在那裡給\ProperName{雅各}祝福\chuan 
\bv{30}\ProperNameC{1}{0}{雅各}便給那地方起名叫\PlaceNameC{0}{0.5}{毘努伊勒}\chuan\chu{就是 神之面的意思}意思說\chientien 我面對面見了 神\chientien 我的性命仍得保全\chuan 
\bv{31}日頭剛出來的時候\chientien\ProperName{雅各}經過\PlaceNameC{0}{0.5}{毘努伊勒}\chientien 他的大腿就瘸了\chuan 
\bv{32}故此\PlaceName{以色列}人不喫大腿窩的筋\chientien 直到今日\chientien 因為那人摸了雅各大腿窩的筋\chuan 

\pend
\endnumbering
\beginnumbering
\pstart
\bchapter%
\bv{1}\ledleftnote{兄弟相見}\ProperNameC{0.5}{0}{雅各}舉目觀看\chientien 見\ProperName{以掃}來了\chientien 後頭跟著四百人\yuentien 他就把孩子們分開交給\ProperNameC{0}{0.5}{利亞}\chientien\ProperNameC{0}{0.5}{拉結}\chientien 和兩個使女\yuentien 
\bv{2}並且叫兩個使女和他們的孩子在前頭\chientien\ProperName{利亞}和他的孩子在後頭\chientien\ProperName{拉結}和\ProperName{約瑟}在儘後頭\chuan 
\bv{3}他自己在他們前頭過去\chientien 一連七次俯伏在地\chientien 纔就近他哥哥\chuan 
\bv{4}\ProperNameC{0.5}{0}{以掃}跑來迎接他\chientien 將他抱住\chientien 又摟著他的頸項與他親嘴\chientien 兩個人就哭了\chuan 
\bv{5}\ProperNameC{0.5}{0}{以掃}舉目看見婦人孩子\chientien 就說\chientien 這些和你同行的是誰呢\yuentien\ProperName{雅各}說\chientien 這些孩子是 神施恩給你僕人的\chuan 
\bv{6}於是兩個使女和他們的孩子前來下拜\chuan 
\bv{7}\ProperNameC{0.5}{0}{利亞}和他的孩子也前來下拜\yuentien 隨後\ProperName{約瑟}和\ProperName{拉結}也前來下拜\chuan 
\bv{8}\ProperNameC{0.5}{0}{以掃}說\chientien 我所遇見的這些群畜是甚麼意思呢\yuentien\ProperName{雅各}說\chientien 是要在我主面前蒙恩的\chuan 
\bv{9}\ProperNameC{0.5}{0}{以掃}說\chientien 兄弟阿\chientien 我的已經夠了\chientien 你的仍歸你罷\chuan 
\bv{10}\ProperNameC{0.5}{0}{雅各}說\chientien 不然\chientien 我若在你眼前蒙恩\chientien 就求你從我手裡收下這禮物\chientien 因為我見了你的面\chientien 如同見了 神的面\chientien 並且你容納了我\chuan 
\bv{11}求你收下我帶來給你的禮物\chientien 因為 神恩待我\chientien 使我充足\yuentien\ProperName{雅各}再三的求他\chientien 他纔收下了\chuan 
\bv{12}\ProperNameC{1}{0}{以掃}說\chientien 我們可以起身前往\chientien 我在你前頭走\yuentien 
\bv{13}雅\ProperName{各}對他說\chientien 我主知道孩子們年幼嬌嫩\chientien 牛羊也正在乳養的時候\chientien 若是催趕一天\chientien 群畜都必死了\chuan 
\bv{14}求我主在僕人前頭走\yuentien 我要量著在我面前群畜和孩子的力量慢慢的前行\chientien 直走到\PlaceName{西珥}我主那裡\chuan 
\bv{15}\ProperNameC{1}{0}{以掃}說\chientien 容我把跟隨我的人留幾個在你這裡\yuentien\ProperName{雅各}說\chientien 何必呢\chientien 只要在我主眼前蒙恩就是了\chuan 
\bv{16}於是\ProperName{以掃}當日起行\chientien 回往\PlaceName{西珥}去了\chuan 
\bv{17}\ProperNameC{1}{0}{雅各}就往\PlaceName{疎割}去\chientien 在那裡為自己蓋造房屋\chientien 又為牲畜搭棚\yuentien 因此那地方名叫\PlaceNameC{0}{0.5}{疎割}\chuan\chu{就是棚的}\linebreak\chu{意思}\Chuan
\bv{18}\ledleftnote{雅各往示劍}\ProperNameC{1}{0}{雅各}從\PlaceName{巴旦亞蘭}回來的時候\chientien 平平安安的到了\PlaceName{迦南}地的\PlaceName{示劍}城\chientien 在城東支搭帳棚\chuan 
\bv{19}就用一百塊銀子向\PlaceName{示劍}的父親哈抹的子孫\chientien 買了支帳棚的那塊地\chuan 
\bv{20}在那裡築了一座壇\chientien 起名叫\PlaceName{伊利伊羅伊以色}\PlaceNameC{0}{0.5}{列}\chuan\chu{就是 神以色列 神的意思}

\pend
\endnumbering
\beginnumbering
\pstart
\bchapter%
\bv{1}\ProperNameC{0.5}{0}{利亞}給\ProperName{雅各}所生的女兒\ProperName{底拿}出去\chientien 要見那地的女子們\chuan 
\bv{2}那地的主\PlaceName{希未}人\chientien\ProperName{哈抹}的兒子\ProperNameC{0}{0.5}{示劍}\chientien 看見他\chientien 就拉住他\chientien 與他行淫\chientien 玷辱他\chuan 
\bv{3}\ProperNameC{0.5}{0}{示劍}的心繫戀雅各的女兒\ProperNameC{0}{0.5}{底拿}\chientien 喜愛這女子\chientien 甜言蜜語的安慰他\chuan 
\bv{4}\ProperNameC{0.5}{0}{示劍}對他父親\ProperName{哈抹}說\chientien 求你為我聘這女子為妻\chuan 
\bv{5}\ProperNameC{0.5}{0}{雅各}聽見\ProperName{示劍}玷污了他的女兒\ProperNameC{0}{0.5}{底拿}\yuentien 那時他的兒子們正和群畜在田野\chientien\ProperName{雅各}就閉口不言\chientien 等他們回來\chuan 
\bv{6}\ProperNameC{0.5}{0}{示劍}的父親\ProperName{哈抹}出來見\ProperNameC{0}{0.5}{雅各}\chientien 要和他商議\chuan 
\bv{7}\ProperNameC{0.5}{0}{雅各}的兒子們聽見這事\chientien 就從田野回來\chientien 人人忿恨\chientien 十分惱怒\chientien 因\ProperName{示劍}在\PlaceName{以色列}家作了醜事\chientien 與\ProperName{雅各}的女兒行淫\chientien 這本是不該作的事\chuan 
\bv{8}\ProperNameC{0.5}{0}{哈抹}和他們商議說\chientien 我兒子\ProperName{示劍}的心戀慕這女子\chientien 求你們將他給我的兒子為妻\chuan 
\bv{9}你們與我們彼此結親\chientien 你們可以把女兒給我們\chientien 也可以娶我們的女兒\chuan 
\bv{10}你們與我們同住罷\chientien 這地都在你們面前\chientien 只管在此居住\chientien 作買賣\chientien 置產業\chuan 
\bv{11}\ProperNameC{1}{0}{示劍}對女兒的父親和弟兄們說\chientien 但願我在你們眼前蒙恩\yuentien 你們向我要甚麼我必給你們\chuan 
\bv{12}任憑向我要多重的聘金和禮物\chientien 我必照你們所說的給你們\chientien 只要把女子給我為妻\chuan 
\bv{13}\ledleftnote{雅各子用計戮示劍族}\ProperNameC{1}{0}{雅各}的兒子們\chientien 因為\ProperName{示劍}玷污了他們的妹子\ProperNameC{0}{0.5}{底拿}\chientien 就用詭詐的話回答\ProperNameC{0}{0.5}{示劍}\chientien 和他父親\ProperNameC{0}{0.5}{哈抹}\yuentien 
\bv{14}對他們說\chientien 我們不能把我們的妹子給沒有受割禮的人為妻\chientien 因為那是我們的羞辱\yuentien 
\bv{15}惟有一件纔可以應允\chientien 若你們所有的男丁都受割禮\chientien 和我們一樣\chientien 
\bv{16}我們就把女兒給你們\chientien 也娶你們的女兒\chientien 我們便與你們同住\chientien 兩下成為一樣的人民\chuan 
\bv{17}倘若你們不聽從我們受割禮\chientien 我們就帶著妹子走了\chuan\Chuan
\bv{18}\ProperNameC{1}{0}{哈抹}和他的兒子\ProperName{示劍}喜歡這話\chuan 
\bv{19}那少年人作這事並不遲延\chientien 因為他喜愛\ProperName{雅各}的女兒\yuentien 他在他父親家中也是人最尊重的\chuan 
\bv{20}\ProperNameC{1}{0}{哈抹}和他兒子\ProperName{示劍}到本城的門口\chientien 對本城的人說\chientien 
\bv{21}這些人與我們和睦\chientien 不如許他們在這地居住作買賣\yuentien 這地也寬闊\chientien 足可容下他們\chientien 我們可以娶他們的女兒為妻\chientien 也可以把我們的女兒嫁給他們\chuan 
\bv{22}惟有一件事我們必須作\chientien 他們纔肯應允\chientien 和我們同住\chientien 成為一樣的人民\chientien 就是我們中間所有的男丁\chientien 都要受割禮\chientien 和他們一樣\chuan 
\bv{23}他們的群畜\chientien 貨財\chientien 和一切的牲口\chientien 豈不都歸我們麼\yuentien 只要依從他們\chientien 他們就與我們同住\chuan 
\bv{24}凡從城門出入的人\chientien 就都聽從\ProperName{哈抹}和他兒子\ProperName{示劍}的話\chientien 於是凡從城門出入的男丁\chientien 都受了割禮\chuan 
\bv{25}到第三天\chientien 眾人正在疼痛的時候\chientien\ProperName{雅各}的兩個兒子\chientien 就是\ProperName{底拿}的哥哥\chientien\ProperName{西緬}和\ProperNameC{0}{0.5}{利未}\chientien 各拿刀劍\chientien 趁著眾人想不到的時候\chientien 來到城中\chientien 把一切男丁都殺了\chuan 
\bv{26}又用刀殺了哈抹和他兒子\ProperNameC{0}{0.5}{示劍}\chientien 把\ProperName{底拿}從\ProperName{示\allowbreak 劍}家裡帶出來\chientien 就走了\chuan 
\bv{27}\ProperNameC{1}{0}{雅各}的兒子們因為他們的妹子受了玷污\chientien 就來到被殺的人那裡\chientien 擄掠那城\chientien 
\bv{28}奪了他們的羊群\chientien 牛群\chientien 和驢\chientien 並城裡田間所有的\yuentien 
\bv{29}又把他們一切貨財\chientien 孩子\chientien 婦女\chientien 並各房中所有的\chientien 都擄掠去了\chuan 
\bv{30}\ProperNameC{1}{0}{雅各}對\ProperName{西緬}和\ProperName{利未}說\chientien 你們連累我\chientien 使我在這地的居民中\chientien 就是在\PlaceName{迦南}人\chientien 和\PlaceName{比利洗}人中\chientien 有了臭名\chientien 我的人丁既然稀少\chientien 他們必聚集來擊殺我\chientien 我和全家的人\chientien 都必滅絕\chuan 
\bv{31}他們說\chientien 他豈可待我們的妹子如同妓女麼\chuan 

\pend
\endnumbering
\beginnumbering
\pstart
\bchapter%
\bv{1}\ledleftnote{雅各往伯特利築壇} 神對\ProperName{雅各}說\chientien 起來\chientien 上\PlaceName{伯特利}去\chientien 住在那裡\chientien 要在那裡築一座壇給 神\chientien 就是你逃避你哥哥\ProperName{以掃}的時候向你顯現的那位\chuan 
\bv{2}\ProperNameC{0.5}{0}{雅各}就對他家中的人\chientien 並一切與他同在的人說\chientien 你們要除掉你們中間的外邦神\chientien 也要自潔\chientien 更換衣裳\chuan 
\bv{3}我們要起來\chientien 上\PlaceName{伯特利}去\yuentien 在那裡我要築一座壇給 神\chientien 就是在我遭難的日子\chientien 應允我\kenten{的禱告}\chientien 在我行的路上保佑我的那位\chuan 
\bv{4}他們就把外邦人的神像\chientien 和他們耳朵上的環子\chientien 交給\ProperNameC{0}{0.5}{雅各}\yuentien\ProperName{雅各}都藏在示劍那裡的橡樹底下\chuan 
\bv{5}他們便起行前往\yuentien  神使那周圍城邑的人都甚驚懼\chientien 就不追趕\ProperName{雅各}的眾子了\chuan 
\bv{6}於是\ProperName{雅各}和一切與他同在的人\chientien 到了\PlaceName{迦南}地的\PlaceNameC{0}{0.5}{路斯}\chientien 就是\PlaceNameC{0}{0.5}{伯特利}\chuan 
\bv{7}他在那裡築了一座壇\chientien 就給那地方起名叫\PlaceNameC{0}{0.5}{伊勒伯特利}\chientien\chu{就是伯特利之 神的意思}因為他逃避他哥哥的時候\chientien  神在那裡向他顯現\chuan 
\bv{8}\ProperNameC{0.5}{0}{利百加}的奶母\ProperName{底波拉}死了\chientien 就葬在\PlaceName{伯特利}下邊橡樹底下\yuentien 那棵樹名叫\PlaceNameC{0}{0.5}{亞倫巴古}\chuan\Chuan
\bv{9}\ProperNameC{0.5}{0}{雅\allowbreak 各}從\PlaceName{巴旦亞蘭}回來\chientien  神又向他顯現賜福與他\yuentien 
\bv{10}且對他說\chientien 你的名原是\ProperNameC{0}{0.5}{雅各}\chientien 從今以後不要再叫\ProperNameC{0}{0.5}{雅各}\chientien 要叫\ProperNameC{0}{0.5}{以色列}\chientien 這樣\chientien 他就改名叫\ProperNameC{0}{0.5}{以色列}\chuan 
\bv{11} 神又對他說\chientien 我是全能的 神\chientien 你要生養眾多\chientien 將來有一族\chientien 和多國的民從你而生\chientien 又有君王從你而出\chuan 
\bv{12}我所賜給\ProperName{亞伯拉罕}和\ProperName{以撒}的地\chientien 我要賜給你\chientien 與你的後裔\chuan 
\bv{13} 神就從那與\ProperName{雅各}說話的地方升上去了\chuan 
\bv{14}\ProperNameC{1}{0}{雅各}便在那裡立了一根石柱\chientien 在柱子上奠酒\chientien 澆油\yuentien 
\bv{15}\ProperNameC{1}{0}{雅各}就給那地方起名叫\ProperNameC{0}{0.5}{伯特利}\chuan\Chuan
\bv{16}\ledleftnote{拉結生子而死}他們從\PlaceName{伯特利}起行\chientien 離\PlaceName{以法他}還有一段路程\chientien\ProperName{拉結}臨產甚是艱難\yuentien 
\bv{17}正在艱難的時候\chientien 收生婆對他說\chientien 不要怕\chientien 你又要得一個兒子了\chuan 
\bv{18}他將近於死\chientien 靈魂要走的時候\chientien 就給他兒子起名叫\ProperNameC{0}{0.5}{便俄尼}\chientien 他父親卻給他起名叫\ProperNameC{0}{0.5}{便雅憫}\chuan 
\bv{19}\ProperNameC{1}{0}{拉結}死了\chientien 葬在\PlaceNameC{0}{0.5}{以法他}的路旁\yuentien\PlaceName{以法他}就是\PlaceNameC{0}{0.5}{伯利恆}\chuan 
\bv{20}\ProperNameC{1}{0}{雅各}在他的墳上立了一統碑\chientien 就是\ProperName{拉結}的墓碑\chientien 到今日還在\chuan 
\bv{21}\ProperNameC{1}{0}{以色列}起行前往\chientien 在\PlaceName{以得}臺那邊支搭帳棚\chuan 
\bv{22}\ProperNameC{1}{0}{以色列}住在那地的時候\chientien\ProperName{流便}去與他父親的妾\ProperName{辟拉}同寢\chientien\ProperName{以色列}也聽見了\chuan\Chuan\ProperName{雅各}共有十二個兒子\chuan 
\bv{23}\ProperNameC{1}{0}{利亞}所生的\chientien 是\ProperName{雅各}的長子\ProperNameC{0}{0.5}{流便}\chientien 還有\ProperNameC{0}{0.5}{西緬}\chientien\ProperNameC{0}{0.5}{利未}\chientien\ProperNameC{0}{0.5}{猶大}\chientien\ProperNameC{0}{0.5}{以薩迦}\chientien\ProperNameC{0}{0.5}{西布倫}\chuan 
\bv{24}\ProperNameC{1}{0}{拉結}所生的是\ProperNameC{0}{0.5}{約瑟}\chientien\ProperNameC{0}{0.5}{便雅憫}\chuan 
\bv{25}\ProperNameC{1}{0}{拉結}的使女\ProperName{辟拉}所生的是\ProperNameC{0}{0.5}{但}\chientien\ProperNameC{0}{0.5}{拿弗他利}\chuan 
\bv{26}\ProperNameC{1}{0}{利亞}的使女\ProperName{悉帕}所生的是\ProperNameC{0}{0.5}{迦得}\chientien\ProperNameC{0}{0.5}{亞設}\chientien 這是\ProperName{雅各}在\PlaceName{巴旦亞蘭}所生的兒子\chuan 
\bv{27}\ProperNameC{1}{0}{雅各}來到他父親\ProperName{以撒}那裡\chientien 到了\PlaceName{基列亞巴}的\PlaceNameC{0}{0.5}{幔利}\chientien 乃是\ProperName{亞伯拉罕}和\ProperName{以撒}寄居的地方\yuentien\PlaceName{基列亞巴}就是\PlaceNameC{0}{0.5}{希伯崙}\chuan\Chuan
\bv{28}\ledleftnote{以撒壽終}\ProperNameC{1}{0}{以撒}共活了一百八十歲\chuan 
\bv{29}\ProperNameC{1}{0}{以撒}年紀老邁\chientien 日子滿足\chientien 氣絕而死\chientien 歸到他列祖\chu{原文作本民}那裡\yuentien 他兩個兒子\ProperName{以掃}\ProperName{雅各}把他埋葬了\chuan 

\pend
\endnumbering
\beginnumbering
\pstart
\bchapter%
\bv{1}\ledleftnote{以掃之後裔}\ProperNameC{0.5}{0}{以掃}就是\ProperNameC{0}{0.5}{以東}\chientien 他的後代\chientien 記在下面\yuentien 
\bv{2}\ProperNameC{0.5}{0}{以掃}娶\PlaceName{迦南}的女子為妻\chientien 就是\PlaceName{赫}人\ProperName{以倫}的女兒\ProperNameC{0}{0.5}{亞大}\chientien 和\ProperName{希未}人\ProperName{祭便}的孫女\chientien\ProperName{亞拿}的女兒\ProperNameC{0}{0.5}{阿何利巴瑪}\chuan 
\bv{3}又娶了\ProperName{以實瑪利}的女兒\chientien\ProperName{尼拜約}的妹子\ProperNameC{0}{0.5}{巴實抹}\chuan 
\bv{4}\ProperNameC{0.5}{0}{亞大}給\ProperName{以掃}生了\ProperNameC{0}{0.5}{以利法}\chientien\ProperName{巴實抹}生了\ProperNameC{0}{0.5}{流珥}\chuan 
\bv{5}\ProperNameC{0.5}{0}{阿何利巴瑪}生了\ProperNameC{0}{0.5}{耶烏施}\chientien\ProperNameC{0}{0.5}{雅蘭}\chientien\ProperNameC{0}{0.5}{可拉}\yuentien 這都是\ProperNameC{0}{0.5}{以掃}的兒子\chientien 是在\PlaceName{迦南}地生的\chuan 
\bv{6}\ProperNameC{0.5}{0}{以掃}帶著他的妻子\chientien 兒女\chientien 與家中一切的人口\chientien 並他的牛羊\chientien 牲畜\chientien 和一切貨財\chientien 就是他在\PlaceName{迦南}地所得的\chientien 往別處去\chientien 離了他兄弟\ProperNameC{0}{0.5}{雅各}\chuan 
\bv{7}因為二人的財物群畜甚多\chientien 寄居的地方容不下他們\chientien 所以不能同居\chuan 
\bv{8}於是\ProperName{以掃}住在\PlaceName{西珥}山裡\chientien\ProperName{以掃}就是\ProperNameC{0}{0.5}{以東}\chuan\Chuan
\bv{9}\ProperNameC{0.5}{0}{以掃}是\PlaceName{西珥}山裡\PlaceName{以東}人的始祖\yuentien 他的後代\chientien 記在下面\yuentien 
\bv{10}\ProperNameC{0.5}{0}{以掃}眾子的名字如下\yuentien\ProperName{以掃}的妻子\ProperName{亞大}生\ProperNameC{0}{0.5}{以利法}\chientien\ProperName{以掃}的妻子\ProperName{巴實抹}生\ProperNameC{0}{0.5}{流珥}\chuan 
\bv{11}\ProperNameC{1}{0}{以利法}的兒子是\ProperNameC{0}{0.5}{提幔}\chientien\ProperNameC{0}{0.5}{阿抹}\chientien\ProperNameC{0}{0.5}{洗玻}\chientien\ProperNameC{0}{0.5}{迦坦}\chientien\ProperNameC{0}{0.5}{基納斯}\chuan 
\bv{12}\ProperNameC{1}{0}{亭納}是\ProperName{以掃}兒子\ProperName{以利法}的妾\yuentien 他給\ProperName{以利法}生了\ProperNameC{0}{0.5}{亞瑪力}\yuentien 這是\ProperName{以掃}的妻子\linebreak\ProperName{亞大}的子孫\chuan 
\bv{13}\ProperNameC{1}{0}{流珥}的兒子是\ProperNameC{0}{0.5}{拿哈}\chientien\ProperNameC{0}{0.5}{謝拉}\chientien\ProperNameC{0}{0.5}{沙瑪}\chientien\ProperNameC{0}{0.5}{米撒}\yuentien 這是\ProperName{以掃}妻子\ProperName{巴實抹}的子孫\chuan 
\bv{14}\ProperNameC{1}{0}{以掃}的妻子\ProperNameC{0}{0.5}{阿何利巴\allowbreak 瑪}\chientien 是\ProperName{祭便}的孫女\chientien\ProperName{亞拿}的女兒\chientien 他給\ProperName{以掃}生了\ProperNameC{0}{0.5}{耶烏施}\chientien\ProperNameC{0}{0.5}{雅蘭}\chientien\ProperNameC{0}{0.5}{可拉}\chuan\Chuan
\bv{15}\ProperNameC{1}{0}{以掃}子孫中作族長的\chientien 記在下面\yuentien\ProperName{以\allowbreak 掃}的長子\chientien\ProperName{以利法}的子孫中\chientien 有\ProperName{提幔}族長\chientien\ProperName{阿抹}族長\chientien\ProperName{洗玻}族長\chientien\ProperNameC{0}{0.5}{基納斯}族長\chientien 
\bv{16}\ProperNameC{1}{0}{可拉}族長\chientien\ProperNameC{0}{0.5}{迦坦}族長\chientien\ProperName{亞瑪力}族長\yuentien 這是在\PlaceName{以東}地從\ProperName{以利法}所出的族長\chientien 都是\ProperName{亞大}的子孫\chuan 
\bv{17}\ProperNameC{1}{0}{以掃}的兒子\ProperName{流珥}的子孫中\chientien 有\ProperName{拿哈}族長\chientien\ProperName{謝拉}族長\chientien\ProperName{沙瑪}族長\chientien\ProperName{米撒}族長\yuentien 這是在\PlaceName{以東}地從\ProperName{流珥}所出的族長\chientien 都是\ProperName{以掃}妻子\ProperName{巴實抹}的子孫\chuan 
\bv{18}\ProperNameC{1}{0}{以掃}的妻子\chientien\ProperName{阿何利巴瑪}的子孫中\chientien 有\ProperName{耶烏施}族長\chientien\ProperName{雅蘭}族長\chientien\ProperName{可拉}族長\yuentien 這是從\ProperName{以掃}妻子\chientien\ProperName{亞拿}的女兒\chientien\ProperName{阿何\allowbreak 利巴瑪}子孫中\chientien 所出的族長\chuan 
\bv{19}以上的族長\chientien 都是\ProperName{以掃}的子孫\chientien\ProperName{以掃}就是以東\chuan\Chuan
\bv{20}\ledleftnote{西珥之後裔}那地原有的居民\PlaceName{何利}人\chientien\linebreak\ProperName{西珥}的子孫\chientien 記在下面\yuentien 就是\ProperNameC{0}{0.5}{羅坍}\chientien\ProperNameC{0}{0.5}{朔巴}\chientien\ProperNameC{0}{0.5}{祭便}\chientien\ProperNameC{0}{0.5}{亞拿}\chientien 
\bv{21}\ProperNameC{1}{0.4}{底順}\chientien\ProperNameC{0}{0.5}{以察}\chientien\ProperNameC{0}{0.5}{底珊}\chientien 這是從\PlaceName{以東}地的\PlaceName{何利}人\yuentien\PlaceName{西珥}子孫中\chientien 所出的族長\chuan 
\bv{22}\ProperNameC{1}{0}{羅坍}的兒子是\ProperNameC{0}{0.5}{何利}\chientien\ProperNameC{0}{0.5}{希幔}\yuentien\ProperName{羅坍}的妹子是\ProperNameC{0}{0.5}{亭納}\chuan 
\bv{23}\ProperNameC{1}{0}{朔巴}的兒子是\ProperNameC{0}{0.5}{亞勒文}\chientien\ProperNameC{0}{0.5}{瑪拿轄}\chientien\ProperNameC{0}{0.5}{以巴錄}\chientien\ProperNameC{0}{0.5}{示玻}\chientien\ProperNameC{0}{0.5}{阿南}\chuan 
\bv{24}\ProperNameC{1}{0}{祭便}的兒子是\ProperNameC{0}{0.5}{亞雅}\chientien\ProperNameC{0}{0.5}{亞拿}\chientien 當時在曠野\chientien 放他父親\ProperName{祭便}的驢\chientien 遇著溫泉的\chientien 就是這\ProperNameC{0}{0.5}{亞拿}\chuan 
\bv{25}\ProperNameC{1}{0}{亞拿}的兒子是\ProperNameC{0}{0.5}{底順}\chientien\ProperName{亞拿}的女兒是\ProperNameC{0}{0.5}{阿何利巴瑪}\chuan 
\bv{26}\ProperNameC{1}{0}{底順}的兒子是\ProperNameC{0}{0.5}{欣但}\chientien\ProperNameC{0}{0.5}{伊是班}\chientien\ProperNameC{0}{0.5}{益蘭}\chientien\ProperNameC{0}{0.5}{基蘭}\chuan 
\bv{27}\ProperNameC{1}{0}{以察}的兒子是\ProperNameC{0}{0.5}{辟罕}\chientien\ProperNameC{0}{0.5}{撒番}\chientien\ProperNameC{0}{0.5}{亞\allowbreak 干}\chuan 
\bv{28}\ProperNameC{1}{0}{底珊}的兒子是\ProperNameC{0}{0.5}{烏斯}\chientien\ProperNameC{0}{0.5}{亞蘭}\chuan 
\bv{29}從\PlaceNameC{0.1}{0}{何利}人所出的族長\chientien 記在下面\yuentien 就是\ProperName{羅坍}族長\chientien\ProperName{朔巴}族長\chientien\ProperName{祭便}族長\chientien\ProperName{亞拿}族長\chientien 
\bv{30}\ProperNameC{1}{0}{底順}族長\chientien\ProperName{以察}族長\chientien\ProperName{底珊}族長\yuentien 這是從\PlaceName{何利}人所出的族長\chientien 都在\PlaceName{西珥}地\chientien 按著宗族作族長\chuan\Chuan
\bv{31}\ledleftnote{記以東諸王}\PlaceNameC{1}{0}{以色}\PlaceName{列}人未有君王治理以先\chientien 在\PlaceName{以東}地作王的\chientien 記在下面\yuentien 
\bv{32}\ProperNameC{1}{0}{比珥}的兒子\ProperName{比拉}在\PlaceName{以東}作王\chientien 他的\kenten{京}城名叫\PlaceName{亭}\PlaceNameC{0}{0.5}{哈巴}\chuan 
\bv{33}\ProperNameC{1}{0}{比拉}死了\chientien\PlaceName{波斯拉}人\chientien\ProperName{謝拉}的兒子\ProperName{約巴}接續他作王\chuan 
\bv{34}\ProperNameC{1}{0}{約巴}死了\chientien\PlaceName{提幔}地的人\ProperName{戶珊}接續他作王\chuan 
\bv{35}\ProperNameC{1}{0}{戶珊}死了\chientien\ProperName{比達}的兒子\ProperName{哈達}接續他作王\yuentien 這\ProperName{哈達}就是在\PlaceName{摩押}地殺敗\PlaceName{米甸}人的\chientien 他的\kenten{京}城名叫\PlaceNameC{0}{0.5}{亞未得}\chuan 
\bv{36}\ProperNameC{1}{0}{哈達}死了\chientien\PlaceName{瑪士利加}人\ProperName{桑拉}接續他作王\chuan 
\bv{37}\ProperNameC{1}{0}{桑拉}死了\chientien 大河邊的\PlaceName{利河伯}人\ProperName{掃羅}接續他作王\chuan 
\bv{38}\ProperNameC{1}{0}{掃羅}死了\chientien\ProperName{亞革波}的兒子\ProperName{巴勒哈南}接續他作王\chuan 
\bv{39}\ProperNameC{1}{0}{亞革波}的兒子\ProperName{巴勒哈南}死了\chientien\ProperName{哈達}接續他作王\chientien 他的\kenten{京}城名叫\PlaceNameC{0}{0.5}{巴烏}\yuentien 他的妻子名叫\ProperNameC{0}{0.5}{米希他別}\chientien 是\ProperName{米薩合}的孫女\chientien\ProperName{瑪特列}的女兒\chuan\Chuan
\bv{40}從\ProperName{以掃}所出的族長\chientien 按著他們的宗族\chientien 住處\chientien 名字\chientien 記在下面\yuentien 就是\ProperName{亭納}族長\chientien\ProperName{亞勒瓦}族長\chientien\ProperName{耶帖}族長\chientien 
\bv{41}\ProperNameC{1}{0}{阿何利巴瑪}族長\chientien\ProperName{以拉}族長\chientien\ProperName{比嫩}族長\chientien 
\bv{42}\ProperNameC{1}{0}{基納斯}族長\chientien\ProperName{提幔}族長\chientien\ProperName{米比薩}族長\chientien 
\bv{43}\ProperNameC{1}{0}{瑪基疊}族長\chientien\ProperName{以蘭}族長\yuentien 這是\PlaceName{以東}人在所得為業的地上\chientien 按著他們的住處\chientien 所有的族長\chientien 都是\PlaceName{以東}人的始祖\chientien \ProperName{以掃}的後代\chuan 

\pend
\endnumbering
\beginnumbering
\pstart
\bchapter%
\bv{1}\ledleftnote{約瑟以兄過告父}\ProperNameC{0.5}{0}{雅各}住在\PlaceName{迦南}地\chientien 就是他父親寄居的地\chuan 
\bv{2}\ProperNameC{0.5}{0}{雅各}的記略如下\yuentien\ProperName{約瑟}十七歲與他哥哥們一同牧羊\chientien 他是個童子\chientien 與他父親的妾\chientien\ProperName{辟拉}\ProperName{悉帕}的兒子們常在一處\yuentien\ProperName{約瑟}將他哥哥們的惡行\chientien 報給他們的父親\chuan 
\bv{3}\ProperNameC{0.5}{0}{以色列}原來愛\ProperName{約瑟}過於愛他的眾子\chientien 因為\ProperName{約瑟}是他年老生的\chientien 他給\ProperName{約瑟}作了一件彩衣\chuan 
\bv{4}\ProperNameC{0.5}{0}{約\allowbreak 瑟}的哥哥們見父親愛\ProperName{約瑟}過於愛他們\chientien 就恨\ProperNameC{0}{0.5}{約瑟}\chientien 不與他說和睦的話\chuan\Chuan
\bv{5}\ledleftnote{約瑟得夢述於諸兄}\ProperNameC{0.5}{0}{約瑟}作了一夢\chientien 告訴他哥哥們\chientien 他們就越發恨他\chuan 
\bv{6}\ProperNameC{0.5}{0}{約瑟}對他們說\chientien 請聽我所作的夢\yuentien 
\bv{7}我們在田裡捆禾稼\chientien 我的捆起來站著\chientien 你們的捆來圍著我的捆下拜\chuan 
\bv{8}他的哥哥們回答說\chientien 難道你真要作我們的王麼\chientien 難道你真要管轄我們麼\yuentien 他們就因為他的夢\chientien 和他的話\chientien 越發恨他\chuan 
\bv{9}後來他又作了一夢\chientien 也告訴他的哥哥們說\chientien 看哪\chientien 我又作了一夢\chientien 夢見太陽\chientien 月亮\chientien 與十一個星\chientien 向我下拜\chuan 
\bv{10}\ProperNameC{0.5}{0}{約瑟}將這夢告訴他父親\chientien 和他哥哥們\chientien 他父親就責備他說\chientien 你作的這是甚麼夢\chientien 難道我和你母親\chientien 你弟兄\chientien 果然要來俯伏在地\chientien 向你下拜麼\chuan 
\bv{11}他哥哥們都嫉妒他\yuentien 他父親卻把這話存在心裡\chuan\Chuan
\bv{12}\ProperNameC{1}{0}{約瑟}的哥哥們往\PlaceName{示劍}去\chientien 放他們父親的羊\chuan 
\bv{13}\ProperNameC{1}{0}{以色列}對\ProperName{約瑟}說\chientien 你哥哥們不是在\PlaceName{示劍}放羊麼\chientien 你來\chientien 我要打發你往他們那裡去\chuan\ProperName{約瑟}說\chientien 我在這裡\chuan 
\bv{14}\ProperNameC{1}{0}{以色列}說\chientien 你去看看你哥哥們平安不平安\chientien 群羊平安不平安\chientien 就回來報信給我\yuentien 於是打發他出\PlaceName{希伯崙}谷\chientien 他就往\PlaceName{示劍}去了\chuan 
\bv{15}有人遇見他在田野走迷了路\chientien 就問他說\chientien 你找甚麼\chuan 
\bv{16}他說\chientien 我找我的哥哥們\chientien 求你告訴我\chientien 他們在何處放羊\chuan 
\bv{17}那人說\chientien 他們已經走了\chientien 我聽見他們說\chientien 要往\PlaceName{多坍}去\yuentien\ProperName{約瑟}就去追趕他哥哥們\chientien 遇見他們在\PlaceNameC{0}{0.5}{多坍}\chuan\Chuan
\bv{18}他們遠遠的看見他\chientien 趁他還沒有走到跟前\chientien 大家就同謀要害死他\chuan 
\bv{19}彼此說\chientien 你看\chientien 那作夢的來了\chuan 
\bv{20}來罷\chientien 我們將他殺了\chientien 丟在一個坑裡\chientien 就說有惡獸把他喫了\chientien 我們且看他的夢將來怎麼樣\chuan 
\bv{21}\ProperNameC{1}{0}{流便}聽見了要救他脫離他們的手\chientien 說\chientien 我們不可害他的性命\chuan 
\bv{22}又說\chientien 不可流\kenten{他的}血\chientien 可以把他丟在這野地的坑裡\chientien 不可下手害他\yuentien\ProperName{流便}的意思是要救他脫離他們的手\chientien 把他歸還他的父親\chuan 
\bv{23}\ProperNameC{1}{0}{約瑟}到了他哥哥們那裡\chientien 他們就剝了他的外衣\chientien 就是他穿的那件彩衣\yuentien 
\bv{24}把他丟在坑裡\yuentien 那坑是空的\chientien 裡頭沒有水\chuan\Chuan
\bv{25}他們坐下喫飯\chientien 舉目觀看\chientien 見有一夥\PlaceName[0.8zw]{\kenten{米}\kenten{甸}}\kenten{的}\PlaceName{以實瑪}\PlaceName{利}人\chientien 從\PlaceName{基列}來\chientien 用駱駝馱著香料\chientien 乳香\chientien 沒藥\chientien 要帶下\PlaceName{埃及}去\chuan 
\bv{26}\ProperNameC{1}{0}{猶大}對眾弟兄說\chientien 我們殺我們的兄弟\chientien 藏了他的血\chientien 有甚麼益處呢\chuan 
\bv{27}我們不如將他賣給\PlaceName{以實瑪利}人\chientien 不可下手害他\chientien 因為他是我們的兄弟\chientien 我們的骨肉\yuentien 眾弟兄就聽從了他\chuan 
\bv{28}\ledleftnote{賣約瑟給以實瑪利人}有些\PlaceName{米甸}的商人\chientien 從那裡經過\chientien 哥哥們就把\ProperName{約瑟}從坑裡拉上來\chientien 講定二十舍客勒銀子\chientien 把\ProperName{約瑟}賣給\PlaceName{以實瑪利}人\yuentien 他們就把\ProperName{約瑟}帶到\PlaceName{埃及}去了\chuan\Chuan
\bv{29}\ProperNameC{1}{0}{流便}回到坑邊\chientien 見\ProperName{約瑟}不在坑裡\chientien 就撕裂衣服\chuan 
\bv{30}回到兄弟們那裡\chientien 說\chientien 童子沒有了\chientien 我往那裡去纔好呢\chuan 
\bv{31}他們宰了一隻公山羊\chientien 把\ProperName{約瑟}的那件彩衣染了血\yuentien 
\bv{32}打發人送到他們的父親那裡說\chientien 我們撿了這個\chientien 請認一認\chientien 是你兒子的外衣不是\chuan 
\bv{33}他認得\chientien 就說\chientien 這是我兒子的外衣\chientien 有惡獸把他喫了\chientien\ProperName{約瑟}被撕碎了\chientien 撕碎了\chuan 
\bv{34}\ledleftnote{雅各裂衣爲子悲哀}\ProperNameC{1}{0}{雅各}便撕裂衣服\chientien 腰間圍上麻布\chientien 為他兒子悲哀了多日\chuan 
\bv{35}他的兒女都起來安慰他\yuentien 他卻不肯受安慰\chientien 說\chientien 我必悲哀著下陰間到我兒子那裡\yuentien\ProperName{約瑟}的父親就為他哀哭\chuan 
\bv{36}\PlaceNameC{1}{0}{米甸}人帶\ProperName{約瑟}到\PlaceNameC{0}{0.5}{埃及}\chientien 把他賣給\ProperName{法老}的內臣\chientien 護衛長\ProperNameC{0}{0.5}{波提乏}\chuan 

\pend
\endnumbering
\beginnumbering
\pstart
\bchapter%
\bv{1}那時\ProperName{猶大}離開他弟兄下去\chientien 到一個\PlaceName{亞杜蘭}人名叫\ProperName{希拉}的家裡去\chuan 
\bv{2}\ProperNameC{0.5}{0}{猶大}在那裡看見一個\PlaceName{迦南}人名叫\ProperName{書亞}的女兒\chientien 就娶他為妻\chientien 與他同房\chientien 
\bv{3}他就懷孕生了兒子\chientien\ProperName{猶大}給他起名叫珥\chuan 
\bv{4}他又懷孕生了兒子\chientien 母親給他起名叫\ProperNameC{0}{0.5}{俄南}\chuan 
\bv{5}他復又生了兒子\chientien 給他起名叫\ProperNameC{0}{0.5}{示拉}\yuentien 他生\ProperName{示拉}的時候\chientien\ProperName{猶大}正在\PlaceName{基}\PlaceNameC{0}{0.4}{悉}\chuan 
\bv{6}\ProperNameC{0.5}{0}{猶大}為長子\ProperName{珥}娶妻\chientien 名叫\ProperNameC{0}{0.5}{他瑪}\chuan 
\bv{7}\ProperNameC{0.5}{0}{猶大}的長子\ProperName{珥}在耶和華眼中看為惡\chientien 耶和華就叫他死了\chuan 
\bv{8}\ProperNameC{0.5}{0}{猶大}對\ProperName{俄\allowbreak 南}說\chientien 你當與你哥哥的妻子同房\chientien 向他盡你為弟的本分\chientien 為你哥哥生子立後\chuan 
\bv{9}\ProperNameC{0.5}{0}{俄南}知道生子不歸自己\chientien 所以同房的時候\chientien 便遺在地\chientien 免得給他哥哥留後\chuan 
\bv{10}\ProperNameC{0.5}{0}{俄南}所作的\chientien 在耶和華眼中看為惡\chientien 耶和華也就叫他死了\chuan 
\bv{11}\ProperNameC{1}{0}{猶大}心裡說\chientien 恐怕\ProperName{示拉}也死\chientien 像他兩個哥哥一樣\chientien 就對他兒婦\ProperName{他瑪}說\chientien 你去\chientien 在你父親家裡守寡\chientien 等我兒子\ProperName{示拉}長大\yuentien\ProperName{他瑪}就回去住在他父親家裡\chuan\Chuan
\bv{12}過了許久\chientien\ProperName{猶大}的妻子\ProperName{書亞}的女兒死了\chientien\ProperName{猶大}得了安慰\chientien 就和他朋友\PlaceName{亞杜蘭}人\ProperName{希拉}上\PlaceName{亭拿}去\chientien 到他剪羊毛的人那裡\chuan 
\bv{13}有人告訴\ProperName{他瑪}說\chientien 你的公公上亭拿剪羊毛去了\chuan 
\bv{14}\ProperNameC{1}{0}{他瑪}見\ProperName{示拉}已經長大\chientien 還沒有娶他為妻\chientien 就脫了他作寡婦的衣裳\chientien 用帕子蒙著臉\chientien 又遮住身體\chientien 坐在\PlaceName{亭拿}路上的\PlaceName{伊拿印}城門口\chuan 
\bv{15}\ProperNameC{1}{0}{猶大}看見他\chientien 以為是妓女\chientien 因為他蒙著臉\chuan 
\bv{16}\ProperNameC{1}{0}{猶大}就轉到他那裡去說\chientien 來罷\chientien 讓我與你同寢\yuentien 他原不知道是他的兒婦\chuan\ProperName{他瑪}說\chientien 你要與我同寢\chientien 把甚麼給我呢\chuan 
\bv{17}\ProperNameC{1}{0}{猶大}說\chientien 我從羊群裡取一隻山羊羔打發人送來給你\yuentien\ProperName{他瑪}說\chientien 在未送以先\chientien 你願意給我一個當頭麼\chuan 
\bv{18}他說\chientien 我給你甚麼當頭呢\chientien\ProperName{他瑪}說\chientien 你的印\chientien 你的帶子\chientien 和你手裡的杖\chuan\ProperName{猶大}就給了他\chientien 與他同寢\yuentien 他就從\ProperName{猶大}懷了孕\chuan 
\bv{19}\ProperNameC{1}{0}{他瑪}起來走了\chientien 除去帕子\chientien 仍舊穿上作寡婦的衣裳\chuan 
\bv{20}\ProperNameC{1}{0}{猶大}託他朋友\PlaceName{亞杜蘭}人\chientien 送一隻山羊羔去\chientien 要從那女人手裡取回當頭來\chientien 卻找不著他\chientien 
\bv{21}就問那地方的人說\chientien\PlaceName{伊拿印}路旁的妓女在那裡\chientien 他們說\chientien 這裡並沒有妓女\chuan 
\bv{22}他回去見\ProperName{猶大}說\chientien 我沒有找著他\chientien 並且那地方的人說\chientien 這裡沒有妓女\chuan 
\bv{23}\ProperNameC{1}{0}{猶大}說\chientien 我把這山羊羔送去了\chientien 你竟找不著他\chientien 任憑他拿去罷\chientien 免得我們被羞辱\chuan\Chuan
\bv{24}約過了三個月\chientien 有人告訴\ProperName{猶大}說\chientien 你的兒婦\ProperName{他瑪}作了妓女\chientien 且因行淫有了身孕\yuentien\ProperName{猶大}說\chientien 拉出他來把他燒了\chuan 
\bv{25}\ProperNameC{1}{0}{他瑪}被拉出來的時候\chientien 便打發人去見他公公\chientien 對他說\chientien 這些東西是誰的\chientien 我就是從誰懷的孕\chientien 請你認一認\chientien 這印\chientien 和帶子\chientien 並杖\chientien 都是誰的\chuan 
\bv{26}\ProperNameC{1}{0}{猶大}承認說\chientien 他比我更有義\chientien 因為我沒有將他給我的兒子\ProperNameC{0}{0.5}{示拉}\yuentien 從此\ProperName{猶大}不再與他同寢了\chuan 
\bv{27}\ledleftnote{他瑪孿生二子}\ProperNameC{1}{0}{他瑪}將要生產\chientien 不料他腹裡是一對雙生\yuentien 
\bv{28}到生產的時候\chientien 一個孩子伸出一隻手來\chientien 收生婆拿紅線拴在他手上\chientien 說\chientien 這是頭生的\chuan 
\bv{29}隨後這孩子把手收回去\chientien 他哥哥生出來了\chientien 收生婆說\chientien 你為甚麼搶著來呢\chientien 因此給他起名叫\ProperNameC{0}{0.5}{法勒斯}\chuan 
\bv{30}後來他兄弟那手上有紅線的\chientien 也生出來\chientien 就給他起名叫\ProperNameC{0}{0.5}{謝拉}\chuan 

\pend
\endnumbering
\beginnumbering
\pstart
\bchapter%
\bv{1}\ProperNameC{0.5}{0}{約瑟}被帶下\PlaceName{埃及}去\yuentien 有一個\PlaceName{埃及}人是\ProperName{法老}的內臣\chientien 護衛長\ProperNameC{0}{0.5}{波提乏}\chientien 從那些帶下他來的\PlaceName{以實瑪利}人手下買了他去\chuan 
\bv{2}\ProperNameC{0.5}{0}{約瑟}住在他主人\PlaceName{埃及}人的家中\yuentien 耶和華與他同在\chientien 他就百事順利\chuan 
\bv{3}他主人見耶和華與他同在\chientien 又見耶和華使他手裡所辦的盡都順利\chuan 
\bv{4}\ProperNameC{0.5}{0}{約瑟}就在主人眼前蒙恩\chientien 伺候他主人\chientien 並且主人派他管理家務\chientien 把一切所有的都交在他手裡\chuan 
\bv{5}自從主人派\ProperName{約瑟}管理家務\chientien 和他一切所有的\chientien 耶和華就因\ProperName{約瑟}的緣故\chientien 賜福與那\PlaceName{埃及}人的家\yuentien 凡家裡和田間一切所有的\chientien 都蒙耶和華賜福\chuan 
\bv{6}\ProperNameC{0.5}{0}{波提乏}將一切所有的\chientien 都交在\ProperName{約瑟}的手中\chientien 除了自己所喫的飯\chientien 別的事一概不知\yuentien\ProperName{約瑟}原來秀雅俊美\chuan\Chuan
\bv{7}這事以後\chientien\ProperName{約瑟}主人的妻以目送情給\ProperNameC{0}{0.5}{約瑟}\chientien 說\chientien 你與我同寢罷\chuan 
\bv{8}\ProperNameC{0.5}{0}{約瑟}不從\chientien 對他主人的妻說\chientien 看哪\chientien 一切家務\chientien 我主人都不知道\chientien 他把所有的都交在我手裡\chuan 
\bv{9}在這家裡沒有比我大的\chientien 並且他沒有留下一樣不交給我\chientien 只留下了你\chientien 因為你是他的妻子\chientien 我怎能作這大惡\chientien 得罪 神呢\chuan 
\bv{10}後來他天天和\ProperName{約瑟}說\chientien\ProperName{約瑟}卻不聽從他\chientien 不與他同寢\chientien 也不和他在一處\chuan 
\bv{11}有一天\chientien\ProperName{約瑟}進屋裡去辦事\chientien 家中人沒有一個在那屋裡\chuan 
\bv{12}婦人就拉住他的衣裳說\chientien 你與我同寢罷\yuentien\ProperName{約瑟}把衣裳丟在婦人手裡\chientien 跑到外邊去了\chuan 
\bv{13}婦人看見\ProperName{約瑟}把衣裳丟在他手裡跑出去了\chientien 
\bv{14}\ledleftnote{主母誣約瑟}就叫了家裡的人來\chientien 對他們說\chientien 你們看\chientien 他帶了一個\PlaceName{希伯來}人\chientien 進入我們家裡\chientien 要戲弄我們\yuentien 他到我這裡來\chientien 要與我同寢\chientien 我就大聲喊叫\yuentien 
\bv{15}他聽見我放聲喊起來\chientien 就把衣裳丟在我這裡\chientien 跑到外邊去了\chuan 
\bv{16}婦人把\ProperName{約瑟}的衣裳放在自己那裡\chientien 等著他主人回家\chientien 
\bv{17}就對他如此如此說\chientien 你所帶到我們這裡的那\PlaceName{希伯來}僕人\chientien 進來要戲弄我\yuentien 
\bv{18}我放聲喊起來\chientien 他就把衣裳丟在我這裡跑出去了\chuan\Chuan
\bv{19}\ProperNameC{1}{0}{約瑟}的主人聽見他妻子對他所說的話\chientien 說\chientien 你的僕人如此如此待我\chientien 他就生氣\yuentien 
\bv{20}\ledleftnote{波提乏囚約瑟於監}把\ProperName{約瑟}下在監裡\chientien 就是王的囚犯被囚的地方\yuentien 於是\ProperName{約瑟}在那裡坐監\chuan 
\bv{21}但耶和華與\ProperName{約瑟}同在\chientien 向他施恩\chientien 使他在司獄的眼前蒙恩\chuan 
\bv{22}司獄就把監裡所有的囚犯\chientien 都交在\ProperName{約瑟}的手下\chientien 他們在那裡所辦的事\chientien 都是經他的手\chuan 
\bv{23}凡在\ProperName{約瑟}手下的事\chientien 司獄一概不察\chientien 因為耶和華與\ProperName{約瑟}同在\chientien 耶和華使他所作的盡都順利\chuan 

\pend
\endnumbering
\beginnumbering
\pstart
\bchapter%
\bv{1}\ledleftnote{酒政膳長同夜得夢}這事以後\chientien\PlaceName{埃及}王的酒政和膳長\chientien 得罪了他們的主\PlaceName{埃及}王\chientien 
\bv{2}\ProperNameC{0.5}{0}{法老}就惱怒酒政和膳長這二臣\yuentien 
\bv{3}把他們下在護衛長府內的監裡\chientien 就是\ProperName{約瑟}被囚的地方\chuan 
\bv{4}護衛長把他們交給\ProperNameC{0}{0.5}{約瑟}\chientien\ProperName{約瑟}便伺候他們\yuentien 他們有些日子在監裡\chuan 
\bv{5}被囚在監之\PlaceName{埃及}王的酒政和膳長\chientien 二人同夜各作一夢\chientien 各夢都有講解\chuan 
\bv{6}到了早晨\chientien\ProperName{約瑟}進到他們那裡\chientien 見他們有愁悶的樣子\chuan 
\bv{7}他便問\ProperName{法老}的二臣\chientien 就是與他同囚在他主人府裡的\chientien 說\chientien 他們今日為甚麼面帶愁容呢\chuan 
\bv{8}他們對他說\chientien 我們各人作了一夢\chientien 沒有人能解\yuentien\ProperName{約瑟}說\chientien 解夢不是出於 神麼\chientien 請你們將夢告訴我\chuan\Chuan
\bv{9}酒政便將他的夢告訴\ProperName{約瑟}說\chientien 我夢見在我面前有一棵葡萄樹\chientien 
\bv{10}樹上有三根枝子\chientien 好像發了芽\chientien 開了花\chientien 上頭的葡萄都成熟了\chuan 
\bv{11}\ProperNameC{1}{0}{法老}的杯在我手中\chientien 我就拿葡萄擠在\ProperName{法老}的杯裡\chientien 將杯遞在他手中\chuan 
\bv{12}\ledleftnote{約瑟解酒政之夢}\ProperNameC{1}{0}{約瑟}對他說\chientien 他所作的夢是這樣解\chientien 三根枝子就是三天\yuentien 
\bv{13}三天之內\chientien\ProperName{法老}必提你出監\chientien 叫你官復原職\chientien 你仍要遞杯在法老的手中\chientien 和先前作他的酒政一樣\chuan 
\bv{14}但你得好處的時候\chientien 求你記念我\chientien 施恩與我\chientien 在\ProperName{法老}面前提說我\chientien 救我出這監牢\chuan 
\bv{15}我實在是從\PlaceName{希伯來}人之地被拐來的\chientien 我在這裡也沒有作過甚麼\chientien 叫他們把我下在監裡\chuan\Chuan
\bv{16}膳長見夢解得好\chientien 就對\ProperName{約瑟}說\chientien 我在夢中見我頭上頂著三筐白餅\yuentien 
\bv{17}極上的筐子裡\chientien 有為\ProperName{法老}烤的各樣食物\chientien 有飛鳥來喫我頭上筐子裡的食物\chuan 
\bv{18}\ledleftnote{解膳長之夢}\ProperNameC{1}{0}{約瑟}說\chientien 你的夢是這樣解\chientien 三個筐子就是三天\yuentien 
\bv{19}三天之內\chientien\ProperName{法老}必斬斷你的頭\chientien 把你挂在木頭上\chientien 必有飛鳥來喫你身上的肉\chuan 
\bv{20}到了第三天\chientien 是\ProperName{法老}的生日\chientien 他為眾臣僕設擺筵席\chientien 把酒政和膳長提出監來\yuentien 
\bv{21}使酒政官復原職\chientien 他仍舊遞杯在法老手中\chuan 
\bv{22}但把膳長挂起來\chientien 正如\ProperName{約瑟}向他們所解的話\chuan 
\bv{23}酒政卻不記念\ProperNameC{0}{0.5}{約瑟}\chientien 竟忘了他\chuan 

\pend
\endnumbering
\beginnumbering
\pstart
\bchapter%
\bv{1}\ledleftnote{法老連得二夢}過了兩年\ProperName{法老}作夢\yuentien 夢見自己站在河邊\chuan 
\bv{2}有七隻母牛從河裡上來\chientien 又美好\chientien 又肥壯在蘆荻中喫草\chuan 
\bv{3}隨後又有七隻母牛從河裡上來\chientien 又醜陋\chientien 又乾瘦\chientien 與那七隻母牛一同站在河邊\chuan 
\bv{4}這又醜陋\chientien 又乾瘦的七隻母牛\chientien 喫盡了那又美好\chientien 又肥壯的七隻母牛\yuentien\ProperName{法老}就醒了\chuan 
\bv{5}他又睡著\chientien 第二回作夢\yuentien 夢見一棵麥子長了七個穗子\chientien 又肥大\chientien 又佳美\chuan 
\bv{6}隨後又長了七個穗子\chientien 又細弱\chientien 又被東風吹焦了\chuan 
\bv{7}這細弱的穗子\chientien 吞了那七個又肥大又飽滿的穗子\yuentien\ProperName{法老}醒了\chientien 不料是個夢\chuan 
\bv{8}到了早晨\chientien\ProperName{法老}心裡不安\chientien 就差人召了\PlaceName{埃及}所有的術士和博士來\yuentien\ProperName{法老}就把所作的夢告訴他們\chientien 卻沒有人能給\ProperName{法老}圓解\chuan\Chuan
\bv{9}\ledleftnote{酒政薦約瑟}那時酒政對\ProperName{法老}說\chientien 我今日想起我的罪來\yuentien 
\bv{10}從前\ProperName{法老}惱怒臣僕\chientien 把我和膳長下在護衛長府佈的監裡\chuan 
\bv{11}我們二人同夜各作一夢\chientien 各夢都有講解\chuan 
\bv{12}在那裡同著我們有一個\PlaceName{希伯來}的少年人\chientien 是護衛長的僕人\chientien 我們告訴他\chientien 他就把我們的夢圓解\chientien 是按著各人的夢圓解的\chuan 
\bv{13}後來正如他給我們圓解的成就了\yuentien 我官復原職\chientien 膳長被挂起來了\chuan\Chuan
\bv{14}\ProperNameC{1}{0}{法老}遂即差人去召\ProperNameC{0}{0.5}{約瑟}\yuentien 他們便急忙帶他出監\chientien 他就剃頭\chientien 刮臉\chientien 換衣裳\chientien 進到\ProperName{法老}面前\chuan 
\bv{15}\ProperNameC{1}{0}{法老}對\ProperName{約瑟}說\chientien 我作了一夢沒有人能解\chientien 我聽見人說\chientien 你聽了夢就能解\chuan 
\bv{16}\ProperNameC{1}{0}{約瑟}回答\ProperName{法老}說\chientien 這不在乎我\chientien  神必將平安的話回答\ProperNameC{0}{0.5}{法老}\chuan 
\bv{17}\ProperNameC{1}{0}{法老}對\ProperName{約瑟}說\chientien 我夢見我站在河邊\yuentien 
\bv{18}有七隻母牛從河裡上來\chientien 又肥壯\chientien 又美好\chientien 在蘆荻中喫草\chuan 
\bv{19}隨後又有七隻母牛上來\chientien 又軟弱\chientien 又醜陋\chientien 又乾瘦\chientien 在\PlaceName{埃及}遍地\chientien 我沒有見過這樣不好的\chuan 
\bv{20}這又乾瘦\chientien 又醜陋的母牛\chientien 喫盡了那以先的七隻肥母牛\chuan 
\bv{21}喫了以後\chientien 卻看不出是喫了\chientien 那醜陋的樣子仍舊和先前一樣\yuentien 我就醒了\chuan 
\bv{22}我又夢見一棵麥子\chientien 長了七個穗子\chientien 又飽滿\chientien 又佳美\yuentien 
\bv{23}隨後又長了七個穗子\chientien 枯槁細弱\chientien 被東風吹焦了\chuan 
\bv{24}這些細弱的穗子\chientien 吞了那七個佳美的穗子\yuentien 我將這夢告訴了術士\chientien 卻沒有人能給我解說\chuan\Chuan
\bv{25}\ledleftnote{約瑟爲法老解夢}\ProperNameC{1}{0}{約瑟}對\ProperName{法老}說\chientien\ProperName{法老}的夢乃是一個\chientien  神已將所要作的事指示\ProperName{法老}了\chuan 
\bv{26}七隻好母牛是七年\yuentien 七個好穗子也是七年\yuentien 這夢乃是一個\chuan 
\bv{27}那隨後上來的七隻又乾瘦\chientien 又醜陋的母牛是七年\chientien 那七個虛空被東風吹焦的穗子也是七年\yuentien 都是七個荒年\chuan 
\bv{28}這就是我對\ProperName{法老}所說\chientien  神已將所要作的事顯明給\ProperName{法\allowbreak 老}了\chuan 
\bv{29}\PlaceNameC{1}{0}{埃及}遍地必來七個大豊年\yuentien 
\bv{30}隨後又要來七個荒年\chientien 甚至在\PlaceName{埃及}地都忘了先前的豊收\chientien 全地必被饑荒所滅\chuan 
\bv{31}因那以後的饑荒甚大\chientien 便不覺得先前的豊收了\chuan 
\bv{32}至於\ProperName{法老}兩回作夢\chientien 是因 神命定這事\chientien 而且必速速成就\chuan 
\bv{33}所以\ProperName{法老}當揀選一個有聰明有智慧的人\chientien 派他治理\PlaceName{埃及}地\chuan 
\bv{34}\ProperNameC{1}{0}{法老}當這樣行\chientien 又派官員管理這地\yuentien 當七個豊年的時候\chientien 征收\PlaceName{埃及}地的五分之一\yuentien 
\bv{35}叫他們把將來豊年一切的糧食聚斂起來\chientien 積蓄五穀\chientien 收存在各城裡作食物\chientien 歸於\ProperName{法老}的手下\chuan 
\bv{36}所積蓄的糧食\chientien 可以防備\PlaceName{埃及}地將來的七個荒年\chientien 免得這地被饑荒所滅\chuan\Chuan
\bv{37}\ProperNameC{1}{0}{法老}和他一切臣僕\chientien 都以這事為妙\chuan 
\bv{38}\ProperNameC{1}{0}{法老}對臣僕說\chientien 像這樣的人\chientien 有 神的靈在他裡頭\chientien 我們豈能找得著呢\chuan 
\bv{39}\ledleftnote{法老立約瑟爲埃及宰}\ProperNameC{1}{0}{法老}對\ProperName{約瑟}說\chientien  神既將這事都指示你\chientien 可見沒有人像你這樣有聰明有智慧\chuan 
\bv{40}你可以掌管我的家\chientien 我的民都必聽從你的話\chientien 惟獨在寶座上我比你大\chuan 
\bv{41}\ProperNameC{1}{0}{法老}又對\ProperName{約瑟}說\chientien 我派你治理\PlaceName{埃及}全地\chuan 
\bv{42}\ProperNameC{1}{0}{法老}就摘下手上打印的戒指\chientien 戴在\ProperName{約瑟}的手上\yuentien 給他穿上細麻衣\chientien 把金鍊戴在他的頸項上\chuan 
\bv{43}又叫\ProperName{約瑟}坐他的副車\chientien 喝道的在前呼叫說\chientien 跪下\yuentien 這樣\chientien\ProperName{法老}派他治理\PlaceName{埃及}全地\chuan 
\bv{44}\ProperNameC{1}{0}{法老}對\ProperName{約瑟}說\chientien 我是\ProperNameC{0}{0.5}{法老}\chientien 在\PlaceName{埃及}全地\chientien 若沒有你的命令\chientien 不許人擅自辦事\chuan\chu{原文作動手動腳}
\bv{45}\ProperNameC{1}{0}{法老}賜名給\ProperNameC{0}{0.5}{約瑟}\chientien 叫\ProperNameC{0}{0.5}{撒發那忒巴內\allowbreak 亞}\yuentien 又將\PlaceName{安}城的祭司\ProperName{波提非拉}的女兒\ProperNameC{0}{0.5}{亞西納}\chientien 給他為妻\chuan\ProperName{約瑟}就出去巡行\PlaceName{埃及}地\chuan\Chuan
\bv{46}\ProperNameC{1}{0}{約瑟}見\PlaceName{埃及}王\ProperName{法\allowbreak 老}的時候\chientien 年三十歲\yuentien 他從\ProperName{法老}面前出去遍行\PlaceName{埃及}全地\chuan 
\bv{47}七個豊年之內\chientien 地的出產極豊極盛\chuan\chu{原文作一把一把的}
\bv{48}\ProperNameC{1}{0}{約瑟}聚歛\PlaceName{埃及}地七個豐年一切的糧食\chientien 把糧食積存在各城裡\chientien 各城周圍田地的糧食\chientien 都積存在本城裡\chuan 
\bv{49}\ProperNameC{1}{0}{約瑟}積蓄五穀甚多\chientien 如同海邊的沙\chientien 無法計算\chientien 因為穀不可勝數\chuan 
\bv{50}荒年未到以前\chientien\PlaceName{安}城的祭司\ProperName{波提非\allowbreak 拉}的女兒\ProperName{亞西納}給\ProperName{約瑟}生了兩個兒子\yuentien 
\bv{51}\ProperNameC{1}{0}{約瑟}給長子起名叫\ProperNameC{0}{0.5}{瑪拿西}\chientien\chu{就是使之忘了的意思}因為\kenten{他說}\chientien  神使我忘了一切的困苦\chientien 和我父的全家\chuan 
\bv{52}他給次子起名叫\ProperNameC{0}{0.5}{以法蓮}\chientien\chu{就是使之昌盛的意思}因為\kenten{他說}\chientien  神使我在受苦的地方昌盛\chuan 
\bv{53}\PlaceNameC{1.1}{0}{埃及}地的七個豊年一完\chientien 
\bv{54}七個荒年就來了\chientien 正如\ProperName{約瑟}所說的\yuentien 各地都有饑荒\chientien 惟獨\PlaceName{埃及}全地有糧食\chuan 
\bv{55}及至\PlaceName{埃及}全地有了饑荒\chientien 眾民向\ProperName{法老}哀求糧食\chientien\ProperName{法老}對他們說\chientien 你們往\ProperName{約瑟}那裡去\chientien 凡他所說的你們都要作\chuan 
\bv{56}當時饑荒遍滿天下\chientien\ProperName{約瑟}開了各處的倉\chientien 糶糧給\PlaceName{埃及}人\yuentien 在\PlaceName{埃及}地饑荒甚大\chuan 
\bv{57}各地的人都往\PlaceName{埃及}去\chientien 到\ProperName{約瑟}那裡糴糧\chientien 因為天下的饑荒甚大\chuan 

\pend
\endnumbering
\beginnumbering
\pstart
\bchapter%
\bv{1}\ledleftnote{雅各遣子往埃及糴糧}\ProperNameC{0.5}{0}{雅各}見\PlaceName{埃及}有糧\chientien 就對兒子們說\chientien 你們為甚麼彼此觀望呢\chuan 
\bv{2}我聽見\PlaceName{埃及}有糧\chientien 你們可以下去從那裡為我們糴些來\chientien 使我們可以存活\chientien 不至於死\chuan 
\bv{3}於是\ProperName{約瑟}的十個哥哥都下\PlaceName{埃及}糴糧去了\chuan 
\bv{4}但\ProperName{約瑟}的兄弟\ProperNameC{0}{0.5}{便雅憫}\chientien\ProperName{雅各}沒有打發他和哥哥們同去\chientien 因為\ProperName{雅各}說\chientien 恐怕他遭害\chuan 
\bv{5}來糴糧的人中有\ProperName{以\allowbreak 色列}的兒子們\chientien 因為\PlaceName{迦南}地也有饑荒\chuan 
\bv{6}當時治理\PlaceName{埃及}地的是\ProperNameC{0}{0.5}{約瑟}\chientien 糶糧給那地眾民的就是他\chuan\ProperName{約瑟}的哥哥們來了\chientien 臉伏於地\chientien 向他下拜\chuan 
\bv{7}\ProperNameC{0.5}{0}{約瑟}看見他哥哥們\chientien 就認得他們\chientien 卻裝作生人\chientien 向他們說些嚴厲話\chientien 問他們說\chientien 你們從那裡來\yuentien 他們說\chientien 我們從\PlaceName{迦南}地來糴糧\chuan 
\bv{8}\ProperNameC{0.5}{0}{約瑟}認得他哥哥們\chientien 他們卻不認得他\chuan 
\bv{9}\ProperNameC{0.5}{0}{約瑟}想起從前所作的那兩個夢\chientien 就對他們說\chientien 你們是奸細\chientien 來窺探這地的虛實\chuan 
\bv{10}他們對他說\chientien 我主阿\chientien 不是的\chientien 僕人們是糴糧來的\chientien 
\bv{11}我們都是一個人的兒子\chientien 是誠實人\chientien 僕人們並不是奸細\chuan 
\bv{12}\ProperNameC{1}{0}{約瑟}說\chientien 不然\chientien 你們必是窺探這地的虛實來的\chuan 
\bv{13}他們說\chientien 僕人們本是弟兄十二人\chientien 是\PlaceName{迦南}地一個人的兒子\chientien 頂小的現今在我們的父親那裡\chientien 有一個沒有了\chuan 
\bv{14}\ProperNameC{1}{0}{約瑟}說\chientien 我纔說你們是奸細\chientien 這話實在不錯\chientien 
\bv{15}我指著\ProperName{法老}的性命起誓\chientien 若是你們的小兄弟\chientien 不到這裡來\chientien 你們就不得出這地方\chientien 從此就可以把你們證驗出來了\yuentien\ledleftnote{約瑟囚凶令攜弟至}
\bv{16}須要打發你們中間一個人去\chientien 把你們的兄弟帶來\chientien 至於你們\chientien 都要囚在這裡\chientien 好證驗你們的話真不真\chientien 若不真\chientien 我指著\ProperName{法老}的性命起誓\chientien 你們一定是奸細\chuan 
\bv{17}於是\ProperName{約瑟}把他們都下在監裡三天\chuan\Chuan
\bv{18}到第三天\chientien\ProperName{約瑟}對他們說\chientien 我是敬畏 神的\chientien 你們照我的話行\chientien 就可以存活\yuentien 
\bv{19}你們如果是誠實人\chientien 可以留你們中間的一個人囚在監裡\chientien 但你們可以帶著糧食回去\chientien 救你們家裡的饑荒\yuentien 
\bv{20}把你們的小兄弟帶到我這裡來\chientien 如此\chientien 你們的話便有證據\chientien 你們也不至於死\yuentien 他們就照樣而行\chuan 
\bv{21}他們彼此說\chientien 我們在兄弟身上實在有罪\chientien 他哀求我們的時候\chientien 我們見他心裡的愁苦\chientien 卻不肯聽\chientien 所以這場苦難臨到我們身上\chuan 
\bv{22}\ProperNameC{1}{0}{流便}說\chientien 我豈不是對你們說過\chientien 不可傷害那孩子麼\chientien 只是你們不肯聽\chientien 所以流他血的罪向我們追討\chuan 
\bv{23}他們不知道\ProperName{約瑟}聽得出來\chientien 因為在他們中間用通事傳話\chuan 
\bv{24}\ProperNameC{1}{0}{約瑟}轉身退去\chientien 哭了一場\chientien 又回來對他們說話\chientien 就從他們中間挑出\ProperName{西緬}來\chientien 在他們眼前把他捆綁\chuan 
\bv{25}\ProperNameC{1}{0}{約瑟}吩咐人把糧食裝滿他們的器具\chientien 把各人的銀子歸還在各人的口袋裡\chientien 又給他們路上用的食物\chientien 人就照他的話辦了\chuan\Chuan
\bv{26}\ledleftnote{見金在囊則懼}他們就把糧食馱在驢上\chientien 離開那裡去了\chuan 
\bv{27}到了住宿的地方\chientien 他們中間有一個人打開口袋\chientien 要拿料餵驢\chientien 纔看見自己的銀子仍在口袋裡\yuentien 
\bv{28}就對弟兄們說\chientien 我的銀子歸還了\chientien 看哪\chientien 仍在我口袋裡\yuentien 他們就提心吊膽\chientien 戰戰兢兢的彼此說\chientien 這是 神向我們作甚麼呢\chuan 
\bv{29}他們來到\PlaceName{迦南}地他們的父親\ProperName{雅各}那裡\chientien 將所遭遇的事都告訴他\chientien 說\chientien 
\bv{30}那地的主對我們說嚴厲的話\chientien 把我們當作窺探那地的奸細\chuan 
\bv{31}我們對他說\chientien 我們是誠實人\chientien 並不是奸細\yuentien 
\bv{32}我們本是弟兄十二人\chientien 都是一個父親的兒子\chientien 有一個沒有了\chientien 頂小的如今同我們的父親在\PlaceName{迦南}地\chuan 
\bv{33}那地的主對我們說\chientien 若要我知道你們是誠實人\chientien 可以留下你們中間的一個人在我這裡\chientien 你們可以帶著\kenten{糧食}回去\chientien 救你們家裡的饑荒\yuentien 
\bv{34}把你們的小兄弟帶到我這裡來\chientien 我便知道你們不是奸細\chientien 乃是誠實人\yuentien 這樣\chientien 我就把你們的弟兄交給你們\chientien 你們也可以在這地作買賣\chuan\Chuan
\bv{35}\ledleftnote{雅各不容攜幼子去}後來他們倒口袋\chientien 不料各人的銀包都在口袋裡\chientien 他們和父親看見銀包就都害怕\chuan 
\bv{36}他們的父親\ProperName{雅各}對他們說\chientien 你們使我喪失我的兒子\yuentien\ProperName{約瑟}沒有了\chientien\ProperName{西緬}也沒有了\chientien 你們又要將\ProperName{便雅憫}帶去\yuentien 這些事都歸到我身上了\chuan 
\bv{37}\ProperNameC{1}{0}{流便}對他父親說\chientien 我若不帶他回來交給你\chientien 你可以殺我的兩個兒子\chientien 只管把他交在我手裡\chientien 我必帶他回來交給你\chuan 
\bv{38}\ProperNameC{1}{0}{雅各}說\chientien 我的兒子不可與你們一同下去\yuentien 他哥哥死了\chientien 只剩下他\chientien 他若在你們所行的路上遭害\chientien 那便是你們使我白髮蒼蒼\chientien 悲悲慘慘的下陰間去了\chuan 

\pend
\endnumbering
\beginnumbering
\pstart
\bchapter%
\bv{1}\ledleftnote{雅各遣子再往糴糧}那地的饑荒甚大\chuan 
\bv{2}他們從\PlaceName{埃及}帶來的糧食喫盡了\chientien 他們的父親就對他們說\chientien 你們再去給我糴些糧來\chuan 
\bv{3}\ProperNameC{0.5}{0}{猶大}對他說\chientien 那人諄諄的誥誡我們\chientien 說\chientien 你們的兄弟若不與你們同來\chientien 你們就不得見我的面\chuan 
\bv{4}你若打發我們的兄弟與我們同去\chientien 我們就下去給你糴糧\chuan 
\bv{5}你若不打發他去\chientien 我們就不下去\chientien 因為那人對我們說\chientien 你們的兄弟若不與你們同來\chientien 你們就不得見我的面\chuan 
\bv{6}\ProperNameC{0.5}{0}{以色列}說\chientien 你們為甚麼這樣害我\chientien 告訴那人你們還有兄弟呢\chuan 
\bv{7}他們回答說\chientien 那人詳細問到我們\chientien 和我們的親屬\chientien 說\chientien 你們的父親還在麼\yuentien 你們還有兄弟麼\chuan 我們就按著他所問的告訴他\chientien 焉能知道他要說\chientien 必須把你們的兄弟帶下來呢\chuan 
\bv{8}\ProperNameC{0.5}{0}{猶大}又對他父親\ProperName{以色列}說\chientien 你打發童子與我同去\chientien 我們就起身下去\chientien 好叫我們和你\chientien 並我們的\kenten{婦人}孩子都得存活\chientien 不至於死\chuan 
\bv{9}我為他作保\chientien 你可以從我手中追討\chientien 我若不帶他回來交在你面前\chientien 我情願永遠擔罪\chuan 
\bv{10}我們若沒有耽擱\chientien 如今第二次都回來了\chuan 
\bv{11}他們的父親\ProperName{以色列}說\chientien 若必須如此\chientien 你們就當這樣行\chientien 可以將這地\chientien 土產中最好的乳香\chientien 蜂蜜\chientien 香料\chientien 沒藥\chientien 榧子\chientien 杏仁\chientien 都取一點收在器具裡\chientien 帶下去送給那人作禮物\chuan 
\bv{12}又要手裡加倍的帶銀子\chientien 並將歸還在你們口袋內的銀子\chientien 仍帶在手裡\yuentien 那或者是錯了\chuan 
\bv{13}也帶著你們的兄弟\chientien 起身去見那人\chuan 
\bv{14}但願全能的 神使你們在那人面前蒙憐憫\chientien 釋放你們的那弟兄和\ProperName{便雅憫}回來\yuentien 我若喪了兒子\chientien 就喪了罷\chuan 
\bv{15}於是他們拿著那禮物\chientien 又手裡加倍的帶銀子\chientien 並且帶著便雅憫\chientien 起身下到\PlaceName{埃及}站在約瑟面前\chuan\Chuan
\bv{16}\ledleftnote{約瑟爲弟兄設席}\ProperNameC{1}{0}{約瑟}見\ProperName{便雅憫}和他們同來\chientien 就對家宰說\chientien 將這些人領到屋裡\chientien 要宰殺牲畜\chientien 預備\kenten{筵席}\chientien 因為晌午這些人同我喫飯\chuan 
\bv{17}家宰就遵著\ProperName{約瑟}的命去行\chientien 領他們進\ProperName{約瑟}的屋裡\chuan 
\bv{18}他們因為被領到\ProperName{約瑟}的屋裡\chientien 就害怕\chientien 說\chientien 領我們到這裡來\chientien 必是因為頭次歸還在我們口袋裡的銀子\chientien 找我們的錯縫\chientien 下手害我們\chientien 強取我們為奴僕\chientien 搶奪我們的驢\chuan 
\bv{19}他們就挨進\ProperName{約瑟}的家宰\chientien 在屋門口和他說話\chientien 
\bv{20}說\chientien 我主阿\chientien 我們頭次下來實在是要糴糧\chuan 
\bv{21}後來到了住宿的地方\chientien 我們打開口袋\chientien 不料各人的銀子分量足數\chientien 仍在各人的口袋內\chientien 現在我們手裡又帶回來了\chuan 
\bv{22}另外又帶下銀子來糴糧\chientien 不知道先前誰把銀子放在我們的口袋裡\chuan 
\bv{23}家宰說\chientien 你們可以放心\chientien 不要害怕\chientien 是你們的 神\chientien 和你們父親的 神\chientien 賜給你們財寶在你們的口袋裡\yuentien 你們的銀子我早已收了\yuentien 他就把\ProperName{西緬}帶出來交給他們\chuan 
\bv{24}家宰就領他們進\ProperName{約瑟}的屋裡\chientien 給他們水洗腳\chientien 又給他們草料餵驢\chuan 
\bv{25}他們就預備那禮物\chientien 等候\ProperName{約瑟}晌午來\chientien 因為他們聽見要在那裡喫飯\chuan\Chuan
\bv{26}\ProperNameC{1}{0}{約瑟}來到家裡\chientien 他們就把手中的禮物拿進屋去給他\chientien 又俯伏在地向他下拜\chuan 
\bv{27}\ProperNameC{1}{0}{約瑟}問他們好\chientien 又問\chientien 你們的父親\chientien 就是你們所說的那老人家平安麼\yuentien 他還在麼\chuan 
\bv{28}他們回答說\chientien 你僕人我們的父親平安\chientien 他還在\yuentien 於是他們低頭下拜\chuan 
\bv{29}\ProperNameC{1}{0}{約瑟}舉目看見他同母的兄弟\ProperNameC{0}{0.5}{便雅憫}\chientien 就說\chientien 你們向我所說那頂小的兄弟\chientien 就是這位麼\yuentien 又說\chientien 小兒阿\chientien 願 神賜恩給你\chuan 
\bv{30}\ProperNameC{1}{0}{約瑟}愛弟之情發動\chientien 就急忙尋找可哭之地\chientien 進入自己的屋裡\chientien 哭了一場\chuan 
\bv{31}他洗了臉出來\chientien 勉強隱忍\chientien 吩咐人擺飯\chuan 
\bv{32}他們就為\ProperName{約瑟}單擺了一席\chientien 為那些人又擺了一席\chientien 也為和\ProperName{約瑟}同喫飯的\PlaceName{埃及}人另擺了一席\yuentien 因為\PlaceName{埃及}人不可和\PlaceName{希伯來}人一同喫飯\yuentien 那原是\PlaceName{埃及}人所厭惡的\chuan 
\bv{33}\ProperNameC{1}{0}{約瑟}使眾弟兄在他面前排列坐席\chientien 都按著長幼的次序\yuentien 眾弟兄就彼此詫異\chuan 
\bv{34}\ProperNameC{1}{0}{約瑟}把他面前的食物分出來\chientien 送給他們\yuentien 但\ProperName{便雅憫}所得的\chientien 比別人多五倍\yuentien 他們就飲酒\chientien 和\ProperName{約瑟}一同宴樂\chuan 

\pend
\endnumbering
\beginnumbering
\pstart
\bchapter%
\bv{1}\ledleftnote{約瑟用策留弟以試諸兄}\ProperNameC{0.5}{0}{約瑟}吩咐家宰說\chientien 把糧食裝滿這些人的口袋\chientien 儘著他們的驢所能馱的\chientien 又把各人的銀子放在各人的口袋裡\chuan 
\bv{2}並將我的銀杯\chientien 和那少年人糴糧的銀子\chientien 一同裝在他的口袋裡\yuentien 家宰就照\ProperName{約\allowbreak 瑟}所說的話行了\chuan 
\bv{3}天一亮就打發那些人帶著驢走了\chuan 
\bv{4}他們出城走了不遠\chientien\ProperName{約瑟}對家宰說\chientien 起來追那些人去\chientien 追上了就對他們說\chientien 你們為甚麼以惡報善呢\chuan 
\bv{5}這不是我主人飲酒的杯麼\chientien 豈不是他占卜用的麼\chientien 你們這樣行是作惡了\chuan 
\bv{6}家宰追上他們\chientien 將這些話對他們說了\chuan 
\bv{7}他們回答說\chientien 我主為甚麼說這樣的話呢\chientien 你僕人斷不能作這樣的事\chuan 
\bv{8}你看\chientien 我們從前在口袋裡所見的銀子\chientien 尚且從\PlaceName{迦南}地帶來還你\chientien 我們怎能從你主人家裡偷竊金銀呢\chuan 
\bv{9}你僕人中\chientien 無論在誰那裡搜出來\chientien 就叫他死\chientien 我們也作我主的奴僕\chuan 
\bv{10}家宰說\chientien 現在就照你們的話行罷\chientien 在誰那裡搜出來\chientien 誰就作我的奴僕\yuentien 其餘的都沒有罪\chuan 
\bv{11}於是他們各人急忙把口袋卸在地下\chientien 各人打開口袋\chuan 
\bv{12}家宰就搜查\chientien 從年長的起\chientien 到年幼的為止\chientien 那杯竟在\ProperName{便雅憫}的口袋裡搜出來\chuan 
\bv{13}他們就撕裂衣服\chientien 各人把馱子抬在驢上\chientien 回城去了\chuan\Chuan
\bv{14}\ProperNameC{1}{0}{猶大}和他弟兄們來到\ProperName{約瑟}的屋中\chientien\ProperName{約瑟}還在那裡\chientien 他們就在他面前俯伏於地\chuan 
\bv{15}\ProperNameC{1}{0}{約瑟}對他們說\chientien 你們作的是甚麼事呢\chientien 你們豈不知像我這樣的人必能占卜嗎\chuan 
\bv{16}\ProperNameC{1}{0}{猶大}說\chientien 我們對我主說甚麼呢\yuentien 還有甚麼話可說呢\yuentien 我們怎能自己表白出來呢\yuentien  神已經查出僕人的罪孽了\yuentien 我們與那在他手中搜出杯來的都是我主的奴僕\chuan 
\bv{17}\ProperNameC{1}{0}{約瑟}說\chientien 我斷不能這樣行\chientien 在誰的手中搜出杯來\chientien 誰就作我的奴僕\yuentien 至於你們可以平平安安的上你們父親那裡去\chuan\Chuan
\bv{18}\ProperNameC{1}{0}{猶大}挨近他說\chientien 我主阿\chientien 求你容僕人說一句話給我主聽\chientien 不要向僕人發烈怒\chientien 因為你如同\ProperName{法老}一樣\chuan 
\bv{19}我主曾問僕人們說\chientien 你們有父親有兄弟沒有\chuan 
\bv{20}我們對我主說\chientien 我們有父親\chientien 已經年老\chientien 還有他老年所生的一個小孩子\chientien 他哥哥死了\chientien 他母親只撇下他一人\chientien 他父親疼愛他\chuan 
\bv{21}你對僕人說\chientien 把他帶到我這裡來\chientien 叫我親眼看看他\chuan 
\bv{22}我們對我主說\chientien 童子不能離開他父親\chientien 若是離開\chientien 他父親必死\chuan 
\bv{23}你對僕人說\chientien 你們的小兄弟若不與你們一同下來\chientien 你們就不得再見我的面\chuan 
\bv{24}我們上到你僕人我們父親那裡\chientien 就把我主的話告訴了他\chuan 
\bv{25}我們的父親說\chientien 你們再去給我糴些糧來\chuan 
\bv{26}我們就說\chientien 我們不能下去\chientien 我們的小兄弟若和我們同往\chientien 我們就可以下去\chientien 因為小兄弟若不與我們同往\chientien 我們必不得見那人的面\chuan 
\bv{27}你僕人我父親對我們說\chientien 你們知道我的妻子給我生了兩個兒子\chuan 
\bv{28}一個離開我出去了\chientien 我說\chientien 他必是被撕碎了\chientien 直到如今我也沒有見他\yuentien 
\bv{29}現在你們又要把這個帶去離開我\chientien 倘若他遭害\chientien 那便是你們使我白髮蒼蒼\chientien 悲悲慘慘的下陰間去了\chuan
\bv{30}我父親的命與這童子的命相連\chientien 如今我回到你僕人我父親那裡\chientien 若沒有童子與我們同在\chientien 
\bv{31}我們的父親見沒有童子\chientien 他就必死\yuentien 這便是我們使你僕人我們的父親白髮蒼蒼\chientien 悲悲慘慘的下陰間去了\chuan 
\bv{32}因為僕人曾向我父親為這童子作保\chientien 說\chientien 我若不帶他回來交給父親\chientien 我便在父親面前永遠擔罪\chuan 
\bv{33}現在求你容僕人住下\chientien 替這童子作我主的奴僕\chientien 叫童子和他哥哥們一同上去\chuan 
\bv{34}若童子不和我同去\chientien 我怎能上去見我父親呢\chientien 恐怕我看見災禍臨到我父親身上\chuan 

\pend
\endnumbering
\beginnumbering
\pstart
\bchapter%
\bv{1}\ledleftnote{約瑟與弟兄相認}\ProperNameC{0.5}{0}{約瑟}在左右站著的人面前\chientien 情不自禁\chientien 吩咐一聲說\chientien 人都要離開我出去\yuentien\ProperName{約瑟}和弟兄們相認的時候\chientien 並沒有一人站在他面前\chuan 
\bv{2}他就放聲大哭\chientien\PlaceName{埃及}人\chientien 和\ProperName{法老}家中的人都聽見了\chuan 
\bv{3}\ProperNameC{0.5}{0}{約瑟}對他弟兄們說\chientien 我是\ProperNameC{0}{0.5}{約瑟}\chientien 我的父親還在麼\yuentien 他弟兄不能回答\chientien 因為在他面前都驚惶\chuan 
\bv{4}\ProperNameC{0.5}{0}{約瑟}又對他弟兄們說\chientien 請你們近前來\chientien 他們就近前來\yuentien 他說\chientien 我是你們的兄弟\ProperNameC{0}{0.5}{約瑟}\chientien 就是你們所賣到\PlaceName{埃及}的\chuan 
\bv{5}現在不要因為把我賣到這裡\chientien 自憂自恨\chientien 這是 神差我在你們以先來\chientien 為要保全生命\chuan 
\bv{6}現在這地的饑荒已經二年了\chientien 還有五年不能耕種\chientien 不能收成\chuan 
\bv{7} 神差我在你們以先來\chientien 為要給你們存留餘種在世上\chientien 又要大施拯救\chientien 保全你們的生命\chuan 
\bv{8}這樣看來\chientien 差我到這裡來的不是你們\chientien 乃是 神\chientien 他又使我如\ProperName{法老}的父\chientien 作他全家的主\chientien 並\PlaceName{埃及}全地的宰相\chuan 
\bv{9}你們要趕緊上到我父親那裡\chientien 對他說\chientien 你兒子\ProperName{約瑟}這樣說\chientien  神使我作全\PlaceName{埃及}的主\chientien 請你下到我這裡來\chientien 不要耽延\chuan 
\bv{10}你和你我兒子\chientien 孫子\chientien 連牛群\chientien 羊群\chientien 並一切所有的\chientien 都可以住在\PlaceName{歌珊}地\chientien 與我相近\yuentien 
\bv{11}我要在那裡奉養你\yuentien 因為還有五年的饑荒\chientien 免得你和你的眷屬\chientien 並一切所有的\chientien 都敗落了\chuan 
\bv{12}況且你們的眼和我兄弟\ProperName{便雅憫}的眼\chientien 都看見是我親口對你們說話\chuan 
\bv{13}你們也要將我在\PlaceName{埃及}一切的榮耀\chientien 和你們所看見的事\chientien 都告訴我父親\chientien 又要趕緊的將我父親搬到我這裡來\chuan 
\bv{14}於是\ProperName{約瑟}伏在他兄弟\ProperName{便雅\allowbreak 憫}的頸項上哭\chientien\ProperName{便雅憫}也在他的頸項上哭\chuan 
\bv{15}他又與眾弟兄親嘴\chientien 抱著他們哭\chientien 隨後他弟兄們就和他說話\chuan\Chuan
\bv{16}\ledleftnote{法老命約瑟迎父}這風聲傳到\ProperName{法老}的宮裡\chientien 說\chientien\ProperName{約瑟}的弟兄們來了\yuentien\ProperName{法老}和他的臣僕都很喜歡\chuan 
\bv{17}\ProperNameC{1}{0}{法老}對\ProperName{約瑟}說\chientien 你吩咐你的弟兄們說\chientien 你們要這樣行\chientien 把馱子抬在牲口上\chientien 起身往\PlaceName{迦南}地去\chuan 
\bv{18}將你們的父親和你們的眷屬\chientien 都搬到我這裡來\chientien 我要把\PlaceName{埃及}地的美物賜給你們\chientien 你們也要喫這地肥美的出產\chuan 
\bv{19}現在我吩咐你們要這樣行\chientien 從\PlaceName{埃及}地帶著車輛去\chientien 把你們的孩子\chientien 和妻子\chientien 並你們的父親都搬來\chuan 
\bv{20}你們眼中不要愛惜你們的家具\chientien 因為\PlaceName{埃及}全地的美物\chientien 都是你們的\chuan\Chuan
\bv{21}\ProperNameC{1}{0}{以色列}的兒子們就如此行\chientien\ProperName{約瑟}照著\ProperName{法老}的吩咐給他們車輛\chientien 和路上用的食物\chuan 
\bv{22}又給他們各人一套衣服\chientien 惟獨給\ProperName{便雅憫}三百銀子\chientien 五套衣服\yuentien 
\bv{23}送給他父親公驢十匹\chientien 馱著\PlaceName{埃及}的美物\chientien 母驢十匹\chientien 馱著糧食與餅\chientien 和菜\chientien 為他父親路上用\chuan 
\bv{24}於是\ProperName{約瑟}打發他弟兄們回去\chientien 又對他們說\chientien 你們不要在路上相爭\chuan 
\bv{25}他們從\PlaceName{埃及}上去\chientien 來到\PlaceName{迦南}地\chientien 他們的父親\ProperName{雅各}那裡\yuentien 
\bv{26}告訴他說\chientien\ProperName{約瑟}還在\chientien 並且作\PlaceName{埃及}全地的宰相\yuentien \ProperName{雅各}心裡冰涼\chientien 因為不信他們\chuan 
\bv{27}他們便將\ProperName{約瑟}對他們說的一切話\chientien 都告訴了他\yuentien 他們父親\ProperNameC{0}{0.5}{雅各}\chientien 又看見\ProperName{約瑟}打發來接他的車輛\chientien 心就甦醒了\chuan 
\bv{28}\ProperNameC{1}{0}{以色列}說\chientien 罷了\chientien 罷了\chientien 我的兒子\ProperName{約瑟}還在\chientien 趁我未死以先\chientien 我要去見他一面\chuan 

\pend
\endnumbering
\beginnumbering
\pstart
\bchapter%
\bv{1}\ProperNameC{0.5}{0}{以色列}帶著一切所有的\chientien 起身來到\PlaceNameC{0}{0.5}{別是巴}\chientien 就獻祭給他父親\ProperName{以撒}的 神\chuan 
\bv{2}夜間 神在異象中對\ProperName{以色列}說\chientien\ProperNameC{0}{0.5}{雅各}\chientien\ProperNameC{0}{0.5}{雅各}\yuentien 他說\chientien 我在這裡\chuan 
\bv{3} 神說\chientien 我是 神\chientien 就是你父親的 神\chientien 你下\PlaceName{埃及}去不要害怕\chientien 因為我必使你在那裡成為大族\chuan 
\bv{4}我要和你同下\PlaceName{埃及}去\chientien 也必定帶你上來\yuentien\ProperName{約瑟}必給你送終\chuan\chu{原文}\linebreak\chu{作將手按在你的眼睛上}
\bv{5}\ProperNameC{0.5}{0}{雅各}就從\PlaceName{別是巴}起行\yuentien\ProperName{以色列}的兒子們使他們的父親\ProperNameC{0}{0.5}{雅各}\chientien 和他們的妻子\chientien 兒女\chientien 都坐在\ProperName{法老}為\ProperName{雅各}送來的車上\chuan 
\bv{6}他們又帶著在\PlaceName{迦南}地所得的牲畜\chientien 貨財\chientien 來到\PlaceNameC{0}{0.5}{埃及}\yuentien\ProperName{雅各}和他的一切子孫都一同來了\chuan 
\bv{7}\ProperNameC{0.5}{0}{雅各}把他的兒子\chientien 孫子\chientien 女兒\chientien 孫女\chientien 並他的子子孫孫\chientien 一同帶到\PlaceNameC{0}{0.5}{埃及}\chuan\Chuan
\bv{8}\ledleftnote{記雅各家下埃及者之名}來到\PlaceName{埃及}的\PlaceName{以色列}人\chientien 名字記在下面\yuentien\ProperName{雅各}和他的兒孫\chientien\ProperName{雅各}的長子是\ProperNameC{0}{0.5}{流便}\chuan 
\bv{9}\ProperNameC{0.5}{0}{流便}的兒子是\ProperNameC{0}{0.5}{哈諾}\chientien\ProperNameC{0}{0.5}{法路}\chientien\ProperNameC{0}{0.5}{希斯倫}\chientien\ProperNameC{0}{0.5}{迦米}\chuan 
\bv{10}\ProperNameC{0.5}{0}{西緬}的兒子是\ProperNameC{0}{0.5}{耶母利}\chientien\ProperNameC{0}{0.5}{雅憫}\chientien\ProperNameC{0}{0.5}{阿轄}\chientien\ProperNameC{0}{0.5}{雅斤}\chientien\ProperNameC{0}{0.5}{瑣轄}\yuentien 還有\PlaceName{迦南}女子所生的\ProperNameC{0}{0.5}{掃羅}\chuan 
\bv{11}\ProperNameC{1}{0}{利未}的兒子是\ProperNameC{0}{0.5}{革順}\chientien\ProperNameC{0}{0.5}{哥轄}\chientien\ProperNameC{0}{0.5}{米拉利}\chuan 
\bv{12}\ProperNameC{1}{0}{猶大}的兒子是\ProperNameC{0}{0.5}{珥}\chientien\ProperNameC{0}{0.5}{俄南}\chientien\ProperNameC{0}{0.5}{示拉}\chientien\ProperNameC{0}{0.5}{法勒斯}\chientien\ProperNameC{0}{0.5}{謝拉}\yuentien 惟有\ProperName{珥}與\ProperNameC{0}{0.5}{俄南}死在\PlaceName{迦南}地\chuan\ProperName{法勒斯}的兒子是\ProperNameC{0}{0.5}{希斯倫}\chientien\ProperNameC{0}{0.5}{哈母勒}\chuan 
\bv{13}\ProperNameC{1}{0}{以薩迦}\linebreak 的兒子是\ProperNameC{0}{0.5}{陀拉}\chientien\ProperNameC{0}{0.5}{普瓦}\chientien\ProperNameC{0}{0.5}{約伯}\chientien\ProperNameC{0}{0.5}{伸崙}\chuan 
\bv{14}\ProperNameC{1}{0}{西布倫}的兒子是\ProperNameC{0}{0.5}{西烈}\chientien\ProperNameC{0}{0.5}{以倫}\chientien\ProperNameC{0}{0.5}{雅利}\chuan 
\bv{15}這是\ProperName{利亞}在\PlaceName{巴旦亞蘭}給\ProperName{雅各}所生的兒子\chientien 還有女兒\ProperNameC{0}{0.5}{底拿}\yuentien 兒孫共有三十三人\chuan 
\bv{16}\ProperNameC{1}{0}{迦得}的兒子是\ProperNameC{0}{0.5}{洗非芸}\chientien\ProperNameC{0}{0.5}{哈基}\chientien\ProperNameC{0}{0.5}{書尼}\chientien\ProperNameC{0}{0.5}{以斯本}\chientien\ProperNameC{0}{0.5}{以利}\chientien\ProperNameC{0}{0.5}{亞羅底}\chientien\ProperNameC{0}{0.5}{亞列利}\chuan 
\bv{17}\ProperNameC{1}{0}{亞設}的兒子是\ProperNameC{0}{0.5}{音拿}\chientien\ProperNameC{0}{0.5}{亦施瓦}\chientien\ProperNameC{0}{0.5}{亦施韋}\chientien\ProperNameC{0}{0.5}{比利亞}\chientien 還有他們的妹子\ProperNameC{0}{0.5}{西拉}\chuan\ProperNameC{0}{0.5}{比利亞}的兒子是\ProperNameC{0}{0.5}{希別}\chientien\ProperNameC{0}{0.5}{瑪結}\chuan 
\bv{18}這是\ProperName{拉\allowbreak 班}給他女兒\ProperName{利亞}的婢女\ProperNameC{0}{0.5}{悉帕}\chientien 從\ProperName{雅各}所生的兒孫\chientien 共有十六人\chuan 
\bv{19}\ProperNameC{1}{0}{雅各}之妻\ProperName{拉結}的兒子是\ProperNameC{0}{0.5}{約瑟}\chientien 和\ProperNameC{0}{0.5}{便雅\allowbreak 憫}\chuan 
\bv{20}\ProperNameC{1}{0}{約瑟}在\PlaceName{埃及}地生了\ProperName{瑪拿西}和\ProperNameC{0}{0.5}{以法蓮}\chientien 就是\PlaceName{安}城的祭司\ProperName{波提非拉}的女兒\ProperName{亞西納}給\ProperName{約瑟}生的\chuan 
\bv{21}\ProperNameC{1}{0}{便雅\allowbreak 憫}的兒子是\ProperNameC{0}{0.5}{比拉}\chientien\ProperNameC{0}{0.5}{比結}\chientien\ProperNameC{0}{0.5}{亞實別}\chientien \ProperNameC{0}{0.5}{基拉}\chientien\ProperNameC{0}{0.5}{乃幔}\chientien\ProperNameC{0}{0.5}{以希}\chientien\ProperNameC{0}{0.5}{羅實}\chientien\ProperNameC{0}{0.5}{母平}\chientien\ProperNameC{0}{0.5}{戶平}\chientien\ProperNameC{0}{0.5}{亞勒}\yuentien 
\bv{22}這是\ProperName{拉結}給\ProperName{雅各}所生的兒孫\chientien 共有十四人\chuan 
\bv{23}\ProperNameC{1}{0}{但}的兒子是\ProperNameC{0}{0.5}{戶伸}\chuan 
\bv{24}\ProperNameC{1}{0}{拿弗他利}的兒子是\ProperNameC{0}{0.5}{雅薛}\chientien\ProperNameC{0}{0.5}{沽尼}\chientien\ProperNameC{0}{0.5}{耶色}\chientien\ProperNameC{0}{0.5}{示冷}\chuan 
\bv{25}這是\ProperName{拉班}給他女兒\ProperNameC{0}{0.5}{拉結}的婢女\ProperNameC{0}{0.5}{辟\allowbreak 拉}\chientien 從\ProperName{雅各}所生的兒孫\chientien 共有七人\chuan 
\bv{26}那與\ProperName{雅各}同到\PlaceName{埃及}的\chientien 除了他兒婦之外\chientien 凡從他所生的\chientien 共有六十六人\chuan 
\bv{27}還有\ProperName{約瑟}在\PlaceName{埃及}所生的兩個兒子\yuentien\ProperName{雅各}家來到\PlaceName{埃及}的共有七十人\chuan\Chuan
\bv{28}\ledleftnote{約瑟往歌珊迎父}\ProperNameC{1}{0}{雅各}打發\ProperName{猶大}先去見\ProperNameC{0}{0.5}{約瑟}\chientien 請派人引路往\PlaceName{歌珊}去\yuentien 於是他們來到\PlaceName{歌珊}地\chuan 
\bv{29}\ProperNameC{1}{0}{約瑟}套車往\PlaceName{歌珊}去\chientien 迎接他父親\ProperNameC{0}{0.5}{以色列}\yuentien 及至見了面\chientien 就伏在父親的頸項上\chientien 哭了許久\chuan 
\bv{30}\ProperNameC{1}{0}{以色列}對\ProperName{約瑟}說\chientien 我既得見你的面\chientien 知道你還在\chientien 就是死我也甘心\chuan 
\bv{31}\ProperNameC{1}{0}{約瑟}對他的弟兄\chientien 和他父的全家說\chientien 我要上去告訴\ProperNameC{0}{0.5}{法老}\chientien 對他說\chientien 我的弟兄和我父的全家\chientien 從前在\PlaceName{迦南}地\chientien 現今都到我這裡來了\chuan 
\bv{32}他們本是牧羊的人\chientien 以養牲畜為業\chientien 他們把羊群\chientien 牛群\chientien 和一切所有的都帶來了\chuan 
\bv{33}等\ProperName{法老}召你們的時候\chientien 問你們說\chientien 你們以何事為業\chientien 
\bv{34}你們要說\chientien 你的僕人從幼年直到如今\chientien 都以養牲畜為業\chientien 連我們的祖宗也都以此為業\chuan 這樣\chientien 你們可以住在\PlaceName{歌珊}地\chientien 因為凡牧羊的都被\PlaceName{埃及}人所厭惡\chuan 

\pend
\endnumbering
\beginnumbering
\pstart
\bchapter%
\bv{1}\ledleftnote{約瑟簡兄弟見法老}\ProperNameC{0.5}{0}{約瑟}進去告訴\ProperName{法老}說\chientien 我的父親和我的弟兄帶著羊群\chientien 牛群\chientien 並一切所有的\chientien 從\PlaceName{迦南}地來了\chientien 如今在\PlaceName{歌珊}地\chuan 
\bv{2}\ProperNameC{0.5}{0}{約瑟}從他弟兄中挑出五個人來\chientien 引他們去見\ProperNameC{0}{0.5}{法老}\chuan 
\bv{3}\ProperNameC{0.5}{0}{法老}問\ProperName{約瑟}的弟兄說\chientien 你們以可事為業\yuentien 他們對\ProperName{法老}說\chientien 你僕人是牧羊的\chientien 連我們的祖宗\chientien 也是牧羊的\chuan 
\bv{4}他們又對\ProperName{法老}說\chientien\PlaceName{迦南}地的饑荒甚大\chientien 僕人的羊群沒有草喫\chientien 所以我們來到這地寄居\yuentien 現在求你容僕人住在\PlaceName{歌珊}地\chuan 
\bv{5}\ProperNameC{0.5}{0}{法老}對\ProperName{約瑟}說\chientien 你父親和你弟兄到你這裡來了\yuentien 
\bv{6}\PlaceNameC{0.5}{0}{埃及}地都在你面前\chientien 只管叫你父親和你弟兄住在國中最好的地\chientien 他們可以住在\PlaceName{歌珊}地\yuentien 你若知道他們中間有甚麼能人\chientien 就派他們看管我的牲畜\chuan 
\bv{7}\ledleftnote{約瑟引雅各見法老}\ProperNameC{0.5}{0}{約瑟}領他父親\ProperName{雅各}進到\ProperName{法老}面前\chientien\ProperName{雅各}就給\ProperName{法老}祝福\chuan 
\bv{8}\ProperNameC{0.5}{0}{法老}問\ProperName{雅各}說\chientien 你平生的年日是多少呢\chuan 
\bv{9}\ProperNameC{0.5}{0}{雅各}對\ProperName{法老}說\chientien 我寄居在世的年日是一百三十歲\chientien 我平生的年日又少\chientien 又苦\chientien 不及我列祖在世寄居的年日\chuan 
\bv{10}\ProperNameC{0.5}{0}{雅各}又給\ProperName{法老}祝福\chientien 就從\ProperName{法老}面前出去了\chuan 
\bv{11}\ProperNameC{1}{0}{約瑟}遵著\ProperName{法老}的命\chientien 把\PlaceName{埃及}國最好的地\chientien 就是\PlaceName{蘭塞}境內的地\chientien 給他父親和弟兄居住\chientien 作為產業\chuan 
\bv{12}\ProperNameC{1}{0}{約瑟}用糧食奉養他父親\chientien 和他弟兄\chientien 並他父親全家的眷屬\chientien 都是照各家的人口奉養他們\chuan\Chuan
\bv{13}饑荒甚大\chientien 全地都絕了糧\chientien 甚至\PlaceName{埃及}地和\PlaceName{迦南}地\kenten{的人}\chientien 因那饑荒的緣故\chientien 都餓昏了\chuan 
\bv{14}\ProperNameC{1}{0}{約瑟}收聚了\PlaceName{埃}\PlaceName{及}地和\PlaceName{迦南}地所有的銀子\chientien 就是眾人糴糧的銀子\chientien 約瑟就把那銀子帶到\ProperName{法老}的宮裡\chuan 
\bv{15}\PlaceNameC{1.1}{0}{埃及}地和\PlaceName{迦南}地的銀子都花盡了\chientien\PlaceName{埃及}眾人都來見約瑟說\chientien 我們的銀子都用盡了\chientien 求你給我們糧食\chientien 我們為甚麼死在你面前呢\chuan 
\bv{16}\ledleftnote{約瑟以糧易畜}\ProperNameC{1}{0}{約瑟}說\chientien 若是銀子用盡了\chientien 可以把你們的牲畜給我\chientien 我就為你們的牲畜給你們糧食\chuan 
\bv{17}於是他們把牲畜趕到\ProperName{約瑟}那裡\chientien\ProperName{約瑟}就拿糧食換了他們的牛\chientien 羊\chientien 驢\chientien 馬\chientien 那一年因換他們一切的牲畜\chientien 就用糧食養活他們\chuan 
\bv{18}那一年過去\chientien 第二年他們又來見\ProperName{約瑟}說\chientien 我們不瞞我主\chientien 我們的銀子都花盡了\chientien 牲畜也都歸了我主\chientien 我們在我主眼前除了我們的身體和田地之外\chientien 一無所剩\chuan 
\bv{19}你何忍見我們人死地荒呢\yuentien 求你用糧食買我們的我們的地\chientien 我們和我們的地就要給\ProperName{法老}效力\yuentien 又求你給我們種子\chientien 使我們得以存活\chientien 不至死亡\chientien 地土也不至荒涼\chuan 
\bv{20}\ledleftnote{以糧易地}於是\ProperName{約瑟}為\ProperName{法老}買了\PlaceName{埃及}所有的地\chientien\PlaceName{埃及}人因被饑荒所迫\chientien 各都賣了自己的田地\yuentien 那地就都歸了\ProperNameC{0}{0.5}{法老}\chuan 
\bv{21}至於百姓\chientien\ProperName{約瑟}叫他們從\PlaceName{埃及}這邊\chientien 直到\PlaceName{埃及}那邊\chientien 都各歸各城\chuan 
\bv{22}惟有祭司的地\chientien\ProperName{約瑟}沒有買\chientien 因為祭司有從\ProperName{法老}所得的常俸\chientien 他們喫\ProperName{法老}所給的常俸\chientien 所以他們不賣自己的地\chuan 
\bv{23}\ProperNameC{1}{0}{約瑟}對百姓說\chientien 我今日為\ProperName{法老}買了你們\chientien 和你們的地\chientien 看哪\chientien 這裡有種子給你們\chientien 你們可以種地\chuan 
\bv{24}後來打糧食的時候\chientien 你們要把五分之一納給\ProperNameC{0}{0.5}{法老}\chientien 四分可以歸你們作地裡的種子\chientien 也作你們和你們家口孩童的食物\chuan 
\bv{25}他們說\chientien 你救了我們的性命\chientien 但願我們在我主眼前蒙恩\chientien 我們就作\ProperName{法老}的僕人\chuan 
\bv{26}於是\ProperName{約\allowbreak 瑟}為\PlaceName{埃及}地定下常例直到今日\chientien\ProperName{法老}必得五分之一\chientien 惟獨祭司的地不歸\ProperNameC{0}{0.5}{法老}\chuan\Chuan
\bv{27}\PlaceNameC{1.1}{0}{以色列}人住在\PlaceName{埃及}的\PlaceName{歌珊}地\chientien 他們在那裡置了產業\chientien 並且生育甚多\chuan 
\bv{28}\ProperNameC{1}{0}{雅各}住在\PlaceName{埃及}地十七年\chientien\ProperName{雅各}平生的年日是一百四十七歲\chuan 
\bv{29}\ledleftnote{雅各遣命於約瑟}\ProperNameC{1}{0}{以色列}的死期臨近了\chientien 他就叫了他兒子\ProperName{約瑟}來\chientien 說\chientien 我若在你眼前蒙恩\chientien 請你把手放在我大腿底下\chientien 用慈愛和誠實待我\chientien 請你不要將我葬在\PlaceNameC{0}{0.5}{埃及}\yuentien 
\bv{30}我與我祖我父同睡的時候\chientien 你要將我帶出\PlaceNameC{0}{0.5}{埃及}\chientien 葬在他們所葬的地方\chuan\ProperName{約瑟}說\chientien 我必遵著你的命而行\chuan 
\bv{31}\ProperNameC{1}{0}{雅各}說\chientien 你要向我起誓\chientien\ProperName{約瑟}就向他起了誓\chientien 於是\ProperName{以\allowbreak 色列}在床頭上\chu{或作扶著杖頭}敬拜 神\chuan 

\pend
\endnumbering
\beginnumbering
\pstart
\bchapter%
\bv{1}\ledleftnote{約瑟攜兒子見雅各}這事以後\chientien 有人告訴\ProperName{約瑟}說\chientien 你的父親病了\yuentien 他就帶著兩個兒子\ProperName{瑪拿西}和\ProperName{以法蓮}同去\chuan 
\bv{2}有人告訴\ProperName{雅各}說\chientien 請看\chientien 你兒子\ProperName{約瑟}到你這裡來了\yuentien\ProperName{以色列}就勉強在床上坐起來\chuan 
\bv{3}\ProperNameC{0.5}{0}{雅各}對\ProperName{約瑟}說\chientien 全能的 神曾在\PlaceName{迦南}地的\PlaceName{路斯}向我顯現\chientien 賜福與我\chuan 
\bv{4}對我說\chientien 我必使你生養眾多\chientien 成為多民\chientien 又要把這地賜給你的後裔\chientien 永遠為業\chuan 
\bv{5}我未到\PlaceName{埃及}見你之先\chientien 你在\PlaceName{埃及}地所生的\ProperName{以法蓮}和\ProperName{瑪拿西}\chientien 這兩個兒子是我的\yuentien 正如\ProperName{流便}和\ProperName{西緬}是我的一樣\chuan 
\bv{6}你在他們以後所生的\chientien 就是你的\chientien 他們可以歸於他們弟兄的名下得產業\chuan 
\bv{7}至於我\chientien 我從\PlaceName{巴旦}來的時候\chientien 拉結死在我眼前\chientien 在\PlaceName{迦南}地的路上\chientien 離\PlaceName{以法他}還有一段路程\chientien 我就把他葬在\PlaceName{以法他}的路上\yuentien\PlaceName{以法他}就是\PlaceNameC{0}{0.5}{伯利恆}\chuan\Chuan
\bv{8}\ProperNameC{0.5}{0}{以色列}看見\ProperName{約瑟}的兩個兒子\chientien 就說\chientien 這是誰\chuan 
\bv{9}\ProperNameC{0.5}{0}{約瑟}對他父親說\chientien 這是 神在這裡賜給我的兒子\yuentien\ProperName{以色列}說\chientien 請你領他們到我跟前\chientien 我要給他們祝福\chuan 
\bv{10}\ProperNameC{0.5}{0}{以色列}年紀老邁\chientien 眼睛昏花\chientien 不能看見\chientien\ProperName{約瑟}領他們到他跟前\chientien 他就和他們親嘴\chientien 抱著他們\chuan 
\bv{11}\ProperNameC{1}{0}{以色列}對\ProperName{約瑟}說\chientien 我想不到得見你的面\chientien 不料\chientien  神又使我得見你的兒子\chuan 
\bv{12}\ProperNameC{1}{0}{約瑟}把兩個兒子從\ProperName{以色列}兩膝中領出來\chientien 自己就臉伏於地下拜\chuan 
\bv{13}隨後\ProperName{約瑟}又拉著他們兩個\chientien\ProperName{以法蓮}在他的右手裡\chientien 對著\ProperName{以色列}的左手\chientien\ProperName{瑪拿西}在他的左手裡\chientien 對著\ProperName{以色列}的右手\chientien 領他們到\ProperName{以色列}的跟前\chuan 
\bv{14}\ProperNameC{1}{0}{以色列}伸出右手來\chientien 按在\ProperName{以法蓮}的頭上\yuentien\ProperName{以法蓮}乃是次子\chientien 又剪搭過左手來\chientien 按在\ProperName{瑪拿西}的頭上\chientien\ProperName{瑪拿西}原是長子\yuentien 
\bv{15}\ledleftnote{雅各給約瑟祝福}他就給\ProperName{約瑟}祝福\chientien 說\chientien 願我祖\ProperNameC{0}{0.5}{亞伯拉\allowbreak 罕}\chientien 和我父\ProperName{以撒}所事奉的 神\chientien 就是一生牧養我直到今日的 神\chientien 
\bv{16}救贖我脫離一切患難的那使者\chientien 賜福與這兩個童子\chientien 願他們歸在我的名下\chientien 和我祖\ProperNameC{0}{0.5}{亞伯拉罕}\chientien 我父\ProperName{以撒}的名下\chientien 又願他們在世界中生養眾多\chuan 
\bv{17}\ProperNameC{1}{0}{約瑟}見他父親把右手按在\ProperName{以法蓮}的頭上\chientien 就不喜悅\chientien 便提起他父親的手\chientien 要從\ProperName{以法蓮}頭上挪到\ProperName{瑪拿西}的頭上\chuan 
\bv{18}\ProperNameC{1}{0}{約瑟}對他父親說\chientien 我父不是這樣\chientien 這本是長子\chientien 求你把右手按在他的頭上\chuan 
\bv{19}他父親不從\chientien 說\chientien 我知道\chientien 我兒\chientien 我知道\chientien 他也必成為一族\chientien 也必昌大\chuan 只是他的兄弟將來比他還大\chientien 他兄弟的後裔要成為多族\chuan 
\bv{20}當日就給他們祝福\chientien 說\chientien\ProperName{以色列}人要指著你們祝福\chientien 說\chientien 願 神使你如\ProperName{以法蓮}\ProperName{瑪拿西}一樣\yuentien 於是立\ProperName{以法蓮}在\ProperName{瑪拿西}以上\chuan 
\bv{21}\ProperNameC{1}{0}{以色列}又對\ProperName{約瑟}說\chientien 我要死了\chientien 但 神必與你們同在\chientien 領你們回到你們列祖之地\chuan 
\bv{22}並且我從前用弓用刀\chientien 從\PlaceName{亞摩利}人手下奪的那塊地\chientien 我都賜給你\chientien 使你比眾弟兄多得一分\chuan 

\pend
\endnumbering
\beginnumbering
\pstart
\bchapter%
\bv{1}\ledleftnote{雅各作歌豫言將來之事}\ProperName{雅各}叫了他的兒子們來\chientien 說\chientien 你們都來聚集\chientien 我好把你們日後必遇的事告訴你們\chuan 
\bv{2}\ProperNameC{0.5}{0}{雅\allowbreak 各}的兒子們\chientien 你們要聚集而聽\chientien 要聽你們父親\ProperName{以色列}的話\chuan\Chuan
\bv{3}\ProperNameC{0.5}{0}{流便}哪\chientien 你是我的長子\chientien 是我力量強壯的時候生的\chientien 本當大有尊榮\chientien 權力超眾\yuentien 
\bv{4}但你\kenten{放縱情慾}\chientien 滾沸如水\chientien 必不得居首位\chientien 因為你上了你父親的床\chientien 污穢了我的榻\chuan\Chuan
\bv{5}\ProperNameC{0.5}{0}{西緬}和\ProperName{利未}是弟兄\chientien 他們的刀劍是殘忍的器具\chuan 
\bv{6}我的靈阿\chientien 不要與他們同謀\chientien 我的心哪\chientien 不要與他們聯絡\chientien 因為他們趁怒殺害人命\chientien 任意砍斷牛腿大筋\yuentien 
\bv{7}他們的怒氣暴烈可咒\yuentien 他們的忿恨殘忍可詛\yuentien 我要使他們分居在\ProperName{雅各}\kenten{家裡}\chientien 散住在\PlaceName{以色列}\kenten{地中}\chuan\Chuan
\bv{8}\ProperNameC{0.5}{0}{猶大}阿\chientien 你弟兄們必讚美你\chientien 你手必掐住仇敵的頸項\chientien 你父親的兒子們必向你下拜\chuan 
\bv{9}\ProperNameC{0.5}{0}{猶大}是個小獅子\yuentien 我兒阿\chientien 你抓了食便上去\yuentien 你屈下身去\chientien 臥如公獅\chientien 蹲如母獅\chientien 誰敢惹你\chuan 
\bv{10}圭必不離\ProperNameC{0}{0.5}{猶大}\chientien 杖必不離他兩腳之間\chientien 直等\ProperName{細羅}\chu{就是賜平安者}來到\chientien 萬民都必歸順\chuan 
\bv{11}\ProperNameC{1}{0}{猶大}把小驢拴在葡萄樹上\chientien 把驢駒拴在美好的葡萄樹上\chientien 他在葡萄酒中洗了衣服\chientien 在葡萄汁中洗了袍褂\chuan 
\bv{12}他的眼睛必因酒紅潤\chientien 他的牙齒必因奶白亮\chuan\Chuan
\bv{13}\ProperNameC{1}{0}{西布倫}必住在海口\chientien 必成為停船的海口\yuentien 他的境界必延到西頓\chuan\Chuan
\bv{14}\ProperNameC{1}{0}{以薩迦}是個強壯的驢\chientien 臥在羊圈之中\yuentien 
\bv{15}他以安靜為佳\chientien 以肥地為美\chientien 便低肩背重\chientien 成為服苦的僕人\chuan\Chuan
\bv{16}\ProperNameC{1}{0}{但}必判斷他的民\chientien 作\PlaceName{以色列}支派之一\chuan 
\bv{17}\ProperNameC{1}{0}{但}必作道上的蛇\chientien 路中的虺\chientien 咬傷馬蹄\chientien 使騎馬的墜落於後\chuan 
\bv{18}耶和華阿\chientien 我向來等候你的救恩\chuan\Chuan
\bv{19}\ProperNameC{1}{0}{迦得}必被敵軍追逼\chientien 他卻要追逼他們的腳跟\chuan\Chuan
\bv{20}亞\ProperName{設}之地必出肥美的糧食\chientien 且出君王的美味\chuan\Chuan
\bv{21}\ProperNameC{1}{0}{拿弗他利}是被釋放的母鹿\chientien 他出嘉美的言語\chuan\Chuan
\bv{22}\ProperNameC{1}{0}{約瑟}是多結果子的樹枝\chientien 是泉旁多結果的枝子\chientien 他的枝條探出牆外\chuan 
\bv{23}弓箭手將他苦害\chientien 向他射箭\chientien 逼迫他\yuentien 
\bv{24}但他的弓仍舊堅硬\chientien 他的手健壯敏捷\yuentien 這是因\ProperName{以色列}的牧者\chientien\ProperName{以色列}的磐石\chientien 就是\ProperName{雅各}的大能者\yuentien 
\bv{25}你父親的 神\chientien 必幫助你\chientien 那全能者\chientien 必將天上所有的福\chientien 地裡所藏的福\chientien 以及生產乳養的福\chientien 都賜給你\chuan 
\bv{26}你父親所祝的福\chientien 勝過我祖先所祝的福\chientien 如永世的山嶺\chientien 至極的邊界\chientien 這些福必降在\ProperName{約瑟}的頭上\chientien 臨列那與弟兄迥別之人的頂上\chuan\Chuan
\bv{27}\ProperNameC{1}{0}{便雅憫}是個撕掠的狼\chientien 早晨要喫他所抓的\chientien 晚上要分他所奪的\chuan\Chuan
\bv{28}這一切是\ProperName{以色\allowbreak 列}的十二支派\yuentien 這也是他們的父親對他們所說的話\chientien 為他們所祝的福\chientien 都是按著各人的福分\chientien 為他們祝福\chuan 
\bv{29}\ledleftnote{遺囑葬事}他又囑咐他們說\chientien 我將要歸到我列祖\chu{原文作本民}那裡\chientien 你們要將我葬在\PlaceName{赫}人\ProperName{以弗崙}田間的洞裡\chientien 與我祖我父在一處\chientien 
\bv{30}就是在\PlaceName{迦南}地\PlaceName{幔利}前\chientien\PlaceName{麥比拉}田間的洞\yuentien 那洞和田\chientien 是\ProperName{亞伯拉罕}向\PlaceName{赫}人\ProperName{以弗崙}買來為業\chientien 作墳地的\chuan 
\bv{31}他們在那裡葬了\ProperName{亞伯拉罕}和他妻子\ProperNameC{0}{0.5}{撒拉}\yuentien 又在那裡葬了\ProperNameC{0}{0.5}{以撒}\chientien 和他妻子\ProperNameC{0}{0.5}{利百加}\yuentien 我也在那裡葬了\ProperNameC{0}{0.5}{利亞}\chuan 
\bv{32}那塊田和田間的洞\chientien 原是向\PlaceName{赫}人買的\chuan 
\bv{33}\ProperNameC{1}{0}{雅各}囑咐眾子已畢\chientien 就把腳收在床上\chientien 氣絕而死\chientien 歸到列祖\chu{原文作本民}那裡去了\chuan 

\pend
\endnumbering
\beginnumbering
\pstart
\bchapter%
\bv{1}\ProperNameC{0.5}{0}{約瑟}伏在他父親的面上哀哭\chientien 與他親嘴\chuan 
\bv{2}\ProperNameC{0.5}{0}{約瑟}吩咐伺候他的醫生\chientien 用香料薰他父親\chientien 醫生就用香料薰了\ProperNameC{0}{0.5}{以色列}\chuan
\bv{3}薰尸的常例是四十天\chientien 那四十天滿了\chientien\PlaceName{埃及}人為他哀哭了七十天\chuan\Chuan
\bv{4}\ledleftnote{約瑟歸葬其父}為他哀哭的日子過了\chientien\ProperName{約瑟}對\ProperName{法老}家中的人說\chientien 我若在你們眼前蒙恩\chientien 請你們報告\ProperName{法老}說\chientien 
\bv{5}我父親要死的時候\chientien 叫我起誓\chientien 說\chientien 你要將我葬在\PlaceName{迦南}地\chientien 在我為自己所掘的墳墓裡\yuentien 現在求你讓我上去葬我父親\chientien 以後我必回來\chuan 
\bv{6}\ProperNameC{0.5}{0}{法老}說\chientien 你可以上去\chientien 照著你父親叫你起的誓\chientien 將他葬埋\chuan 
\bv{7}於是\ProperName{約瑟}上去葬他父親\yuentien 與他一同上去的\chientien 有\ProperName{法老}的臣僕\chientien 和\ProperName{法老}家中的長老\chientien 並\PlaceName{埃及}國的長老\yuentien 
\bv{8}還有\ProperName{約瑟}的全家\chientien 和他的弟兄們\chientien 並他父親的眷屬\chientien 只有他們的\kenten{婦人}\chientien 孩子\chientien 和羊群\chientien 牛群\chientien 都留在\PlaceName{歌珊}地\yuentien 
\bv{9}又有車輛\chientien 馬兵\chientien 和他一同上去\yuentien 那一幫人甚多\chuan 
\bv{10}他們到了\PlaceName{約但}河外\chientien\ProperName{亞達}的禾場\chientien 就在那裡大大的號咷痛哭\yuentien\ProperName{約瑟}為他父親哀哭了七天\chuan 
\bv{11}\PlaceNameC{1}{0}{迦南}的居民\chientien 見\ProperName{亞達}禾場上的哀哭\chientien 就說\chientien 這是\PlaceName{埃及}人一場極大的哀哭\chientien 因此那地方名叫\PlaceNameC{0}{0.5}{亞伯麥西}\chientien 是在\PlaceName{約但}河東\chuan 
\bv{12}\ProperNameC{1}{0}{雅各}的兒子們\chientien 就遵著他父親所吩咐的辦了\chientien 
\bv{13}把他搬到\PlaceName{迦南}地\chientien 葬在幔利前\chientien 麥比拉田間的洞裡\yuentien 那洞和田\chientien 是亞伯拉罕向赫人以弗崙買來為業作墳地的\chuan 
\bv{14}\ProperNameC{1}{0}{約瑟}葬了他父親以後\chientien 就和眾弟兄\chientien 並一切同他上去葬他父親的人\chientien 都回\PlaceName{埃及}去了\chuan\Chuan
\bv{15}\ProperNameC{1}{0}{約瑟}的哥哥們見父親死了\chientien 就說\chientien 或者\ProperName{約瑟}懷恨我們\chientien 照著我們從前待他一切的惡\chientien 足足的報復我們\chuan 
\bv{16}他們就打發人去見\ProperName{約瑟}說\chientien 你父親未死以先\chientien 吩咐說\chientien 
\bv{17}你們要對\ProperName{約瑟}這樣說\chientien 從前你哥哥們惡待你\chientien 求你饒恕他們的過犯\chientien 和罪惡\yuentien 如今求你饒恕你父親 神之僕人的過犯\yuentien 他們對\ProperName{約瑟}說這話\chientien\ProperName{約瑟}就哭了\chuan 
\bv{18}他的哥哥們又來俯伏在他面前說\chientien 我們是你的僕人\chuan 
\bv{19}約\ProperName{瑟}對他們說\chientien 不要害怕\chientien 我豈能代替 神呢\chuan 
\bv{20}從前你們的意思是要害我\chientien 但 神的意思原是好的\chientien 要保全許多人的性命\chientien 成就今日的光景\yuentien 
\bv{21}現在你們不要害怕\chientien 我必養活你們\chientien 和你們的\kenten{婦人}\chientien 孩子\yuentien 於是\ProperName{約瑟}用親愛的話安慰他們\chuan\Chuan
\bv{22}\ProperNameC{1}{0}{約瑟}和他父親的眷屬\chientien 都住在\PlaceNameC{0}{0.5}{埃及}\yuentien\ProperName{約瑟}活了一百一十歲\chuan 
\bv{23}\ProperNameC{1}{0}{約瑟}得見\ProperName{以法蓮}第三代的子孫\yuentien\ProperName{瑪拿西}的孫子\ProperName{瑪吉}的兒子\chientien 也養在\ProperName{約瑟}的膝上\chuan 
\bv{24}\ProperNameC{1}{0}{約瑟}對他弟兄們說\chientien 我要死了\chientien 但 神必定看顧你們\chientien 領你們從這地上去\chientien 到他起誓所應許給\ProperNameC{0}{0.5}{亞伯拉罕}\chientien\ProperNameC{0}{0.5}{以撒}\chientien \ProperName{雅各}之地\chuan 
\bv{25}\ledleftnote{約瑟遺囑弟兄將其骸骨攜歸故土}\ProperNameC{1}{0}{約瑟}叫\ProperName{以色列}的子孫起誓\chientien 說\chientien  神必定看顧你們\chientien 你們要把我的骸骨從這裡搬上去\chuan 
\bv{26}\ProperNameC{1}{0}{約瑟}死了\chientien 正一百一十歲\yuentien 人用香料將他薰了\chientien 把他收殮在棺材裡\chientien 停在\PlaceNameC{0}{0.5}{埃及}\chuan 

\pend
\endnumbering

% \bbook{出埃及記}
% \input{tex/exo/exo.tex}
% \bbook{利未記}
% \input{tex/lev/lev.tex}
% \bbook{民數記}
% \input{tex/num/num.tex}
% \bbook{申命記}
% \input{tex/deu/deu.tex}
% \bbook{約書亞記}
% \input{tex/jos/jos.tex}
% \bbook{士師記}
% \input{tex/jug/jug.tex}
% \bbook{路得記}
% \input{tex/rut/rut.tex}
% \bbook{撒母爾記上}
% \input{tex/1sa/1sa.tex}
% \bbook{撒母爾記下}
% \input{tex/2sa/2sa.tex}
% \bbook{列王記上}
% \input{tex/1ki/1ki.tex}
% \bbook{列王記下}
% \input{tex/2ki/2ki.tex}
% \bbook{歷代志上}
% \input{tex/1ch/1ch.tex}
% \bbook{歷代志下}
% \input{tex/2ch/2ch.tex}
% \bbook{以斯拉記}
% \input{tex/ezr/ezr.tex}
% \bbook{尼希米記}
% \input{tex/neh/neh.tex}
% \bbook{以斯帖記}
% \input{tex/est/est.tex}
% \bbook{約伯記}
% \input{tex/job/job.tex}
% \bbook{詩篇}
% \input{tex/psm/psm.tex}
% \bbook{箴言}
% \input{tex/pro/pro.tex}
% \bbook{傳道書}
% \input{tex/ecc/ecc.tex}
% \bbook{雅歌}
% \input{tex/son/son.tex}
% \bbook{以賽亞書}
% \input{tex/isa/isa.tex}
% \bbook{耶利米書}
% \input{tex/jer/jer.tex}
% \bbook{耶利米哀歌}
% \input{tex/lam/lam.tex}
% \bbook{以西結書}
% \input{tex/eze/eze.tex}
% \bbook{但以理書}
% \input{tex/dan/dan.tex}
% \bbook{何西阿書}
% \input{tex/hos/hos.tex}
% \bbook{約珥書}
% \input{tex/joe/joe.tex}
% \bbook{阿摩司書}
% \input{tex/amo/amo.tex}
% \bbook{俄巴底亞書}
% \input{tex/oba/oba.tex}
% \bbook{約拿書}
% \input{tex/jon/jon.tex}
% \bbook{彌迦書}
% \input{tex/mic/mic.tex}
% \bbook{那鴻書}
% \input{tex/nah/nah.tex}
% \bbook{哈巴谷書}
% \input{tex/hab/hab.tex}
% \bbook{西番雅書}
% \input{tex/zep/zep.tex}
% \bbook{哈該書}
% \input{tex/hag/hag.tex}
% \bbook{撒迦利亞書}
% \input{tex/zec/zec.tex}
% \bbook{瑪拉基書}
% \input{tex/mal/mal.tex}

% \part{新約全書}
% \bbook{馬太福音}
% \input{tex/mat/mat.tex}
% \bbook{馬可福音}
% \input{tex/mak/mak.tex}
% \bbook{路加福音}
% \input{tex/luk/luk.tex}
% \bbook{約翰福音}
% \input{tex/jhn/jhn.tex}
% \bbook{使徒行傳}
% \input{tex/act/act.tex}
% \bbook{羅馬書}
% \input{tex/rom/rom.tex}
% \bbook{哥林多前書}
% \input{tex/1co/1co.tex}
% \bbook{哥林多後書}
% \input{tex/2co/2co.tex}
% \bbook{加拉太書}
% \input{tex/gal/gal.tex}
% \bbook{以弗所書}
% \input{tex/eph/eph.tex}
% \bbook{腓立比書}
% \input{tex/phl/phl.tex}
% \bbook{歌羅西書}
% \input{tex/col/col.tex}
% \bbook{帖撒羅尼迦前書}
% \input{tex/1ts/1ts.tex}
% \bbook{帖撒羅尼迦後書}
% \input{tex/2ts/2ts.tex}
% \bbook{提摩太前書}
% \input{tex/1ti/1ti.tex}
% \bbook{提摩太後書}
% \input{tex/2ti/2ti.tex}
% \bbook{提多書}
% \input{tex/tit/tit.tex}
% \bbook{腓利門書}
% \input{tex/phm/phm.tex}
% \bbook{希伯來書}
% \input{tex/heb/heb.tex}
% \bbook{雅各書}
% \input{tex/jas/jas.tex}
% \bbook{彼得前書}
% \input{tex/1pe/1pe.tex}
% \bbook{彼得後書}
% \input{tex/2pe/2pe.tex}
% \bbook{約翰一書}
% \input{tex/1jn/1jn.tex}
% \bbook{約翰二書}
% \input{tex/2jn/2jn.tex}
% \bbook{約翰三書}
% \input{tex/3jn/3jn.tex}
% \bbook{猶大書}
% \input{tex/jud/jud.tex}
% \bbook{啟示錄}
% \input{tex/rev/rev.tex}

\end{document}